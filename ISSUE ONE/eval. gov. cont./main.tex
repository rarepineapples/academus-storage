\documentclass[12pt, a4paper, twoside]{article}
\usepackage{format}
% Do not alter above

% Metadata: put your article information here 
\newcommand{\jtitle}{Evaluating Governmental Control on Social Media Speech}
\newcommand{\jauthor}{Joseph Huang}
\newcommand{\jaffiliation}{Acalanes High School}

% Editors will change these fields after acceptance 
\newcommand{\jvolume}{1}
\newcommand{\jyear}{2025}
% \newcommand{\jdoi}{N/A}  


% References should be placed in refs.bib and cited with \autocite{<source>}
% Quotations can be placed in quote environments: \begin{quote}<your quote>\end{quote}
% Footnotes can be added with \footnote{<your footnote>}

% Your Content

\begin{document}

\maketitle{}

In recent decades, fake news—whether spread intentionally or unintentionally—has emerged as a significant challenge for countries around the world. Widespread claims of voter fraud on social media following the 2020 U.S. election undermined trust in the electoral process, resulting in reduced voter turnout in subsequent elections (Sanchez and Middlemass). Similarly, social media disinformation has fueled dangerous and violent behavior during events such as the Rohingya genocide, Southport murders, and COVID-19 (Booth; Ferreira Caceres et al.; Ortutay). Accordingly, many have argued for government regulation aimed at suppressing disinformation. However, given the practical, ethical, and economic risks of such an approach, fake news is better addressed through independent mechanisms such as third-party fact-checkers and media literacy initiatives. 

The foremost consideration when evaluating the efficacy of government regulations is their feasibility, namely, the choice of enforcement mechanisms. In Germany, the 2017 Network Enforcement Act, which relied on content removal via user complaints, prompted platforms to over‐block content, often erring on the side of deletion to avoid potential fines, while producing little reduction in disinformation (Griffin). Utilizing AI to evaluate content is likewise difficult, as AI-based takedowns yield an “enormous number of false positives”; further, algorithmic fact-checking is largely limited to English, lacking sufficient multilingual training data (Marsden et al.). Another concern is that regulation may push platforms to prioritize legal compliance over responding to evolving disinformation threats, especially as misinformation outpaces slow regulatory frameworks (Bateman and Jackson). Ultimately, the technical and structural limitations of government enforcement render it an inflexible and inefficient means of combating the spread of disinformation. 

Another concern relates to classification of misinformation: who gets to decide what “fake news” constitutes? As historical examples illustrate, especially in countries with weaker political institutions, governments have used misinformation laws to silence dissent under the guise of public safety or national unity. In Bangladesh, India, Indonesia, Malaysia, the Philippines, and Thailand, “fake news” regulation extends to government criticism; “malicious intent” clauses within such regulation justify arbitrary arrests, extended pretrial detentions, and excessive fines (Anansaringkarn and Neo; Mahapatra et al.). Singapore’s regulations allow government officials to simply declare the falsity of online information, or even the removal of said information if doing so is “deemed to be in the public interest” (“Singapore”). Often, regulation in democratic countries is used as pretext by other, illiberal states to curb press freedoms (Henley; Mahapatra et al.). Even if well-intentioned, laws enabling government regulation of social media risk centralizing undue power under state control. Thus, empowering governments to define and police fake news carries inherent risks of overreach. 

An often-neglected consideration is the effect of government regulation on the overall economy. Given that regulation increases compliance costs for platforms, the bar of entry into the social media industry rises, reducing competition and consolidating incumbent power (Sperry). The effects of government regulation on the overall economy can be illustrated by the EU’s General Data Protection Regulation (GDPR), a law designed to protect user privacy online by requiring companies obtain explicit consent for data collection and use. The GDPR serves as an effective proxy for government regulation of social media since both impose high compliance costs. As expected, following the launch of GDPR, market concentration in web tracking increased by 17\%, and websites dropped smaller third-party tracking and advertising companies who could not ensure compliance (Prasad). In addition, economic modelling indicates that repealing Section 230—the U.S. law shielding platforms from liability for user content—would eliminate roughly 425,000 American jobs, illustrating how regulation aimed at policing fake news can stifle growth and employment across an entire economy (6 Myths About Section 230). In this broader context, regulation not only undermines innovation and market competition but also imposes widespread economic costs. 

Evidently, government regulation is an inadequate solution for the issue of fake news. Combating such an issue requires an alternative approach: third-party fact-checkers. This is supported by empirical evidence favoring third-party involvement—contrasted with the lack of empirical evidence favoring government regulation. While third party fact checkers consistently produce a 10-14\% decrease in false belief and a roughly 60\% reduction in the resharing of debunked posts, rigorous studies of government regulation are scarce, most likely due to the logistical and ethical challenges of testing national laws through randomized controlled trials (Chuai et al.; Porter and Wood). Although less rigorous than randomized controlled trials and given the aforementioned challenges, public-opinion surveys offer an alternative to gauge the perceived effectiveness of regulation. One such survey conducted in Singapore indicated that only approximately 34\% of respondents reported that the 2019 Protection from Online Falsehoods and Manipulation Act stopped the creation of fake news (Ang and Zhang). This data underscores a major weakness of government regulation: not only is there a lack of rigorous evidence supporting its effectiveness, but even the sparse survey-based evidence that exists suggests it may have little real-world impact. Fact-checkers are only a temporary solution, however; a far more sustainable and preventive approach is media literacy. Although media literacy programs yield a smaller drop in belief in fake news (~0.27 SD), they produce a significant drop in the sharing of fake news (~1.1 SD) (Huang et al.). Furthermore, fact-checking requires ongoing human effort and is less effective when dealing with fake news in non-English languages, leading to higher long-term costs. In contrast, media literacy can be applied more broadly, requires fewer recurring resources, and equips individuals to assess the reliability of new information on their own (Nygren and Efimova). Together, fact-checking and media literacy provide a more effective response to fake news than government regulation. 

In essence, while the harms of fake news on social media are pressing, the challenges of enforcement, the threat of political abuse, and the suppression of competition make government regulation an ill-advised solution. Instead, independent fact-checking and scalable media literacy programs offer empirically supported, rights-preserving alternatives. The true solution lies not in restricting speech, but in equipping individuals with the tools to critically evaluate information, ensuring a more informed and resilient society. 

\nocite{*}
\printbibliography
\end{document}
