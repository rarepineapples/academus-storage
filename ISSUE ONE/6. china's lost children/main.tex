\documentclass[12pt, a4paper, twoside]{article}
\usepackage{format}
% Do not alter above

% Metadata: put your article information here 
\newcommand{\jtitle}{China's Lost Children, its Dying Economy, and the Economic Implications}
\newcommand{\jauthor}{Kevin Nguyen}
\newcommand{\jaffiliation}{California High School}

% Editors will change these fields after acceptance 
\newcommand{\jvolume}{1}
\newcommand{\jyear}{2025}
% \newcommand{\jdoi}{N/A}  


% References should be placed in refs.bib and cited with \autocite{<source>}
% Quotations can be placed in quote environments: \begin{quote}<your quote>\end{quote}
% Footnotes can be added with \footnote{<your footnote>}

% Your Content

\begin{document}

\maketitle{}

First implemented beginning 1980, the One Child Policy (OCP) was a policy formulated by the Chinese Communist Party (CCP) with the goal of reducing the birth rates in a quickly industrializing China. By reducing population growth, the government hoped to avoid a Malthusian catastrophe and maintain an “optimal” population for China (L.\ Jiang) Throughout the existence of the policy, it saw a general trend in the gradual loosening of enforcement and growing number of exceptions to the policy. Despite this, the policy still managed to cause an enormous impact upon population demographics, preventing the birth of 100 to 400 million children (Q.\ Jiang), and radically skewing the gender ratio (Brittanica), causing an estimate of 3-4\% more males than females in China. Part of the consideration during the inception of the policy was that the OCP would allow for an increase in the ratio of working age adults to children, which would incentivize saving and increase productivity. While this was the case in the short term, in the long term (Feng), the OCP would cause a shrinkage and expansion (Silver \& Huang) of the working age population and the elderly population respectively, augmenting the strain upon welfare programs and taxpayers. These varieties of factors would not only have a profound effect on the Chinese economy, but consequentially, on the international economy. 

There were some benefits that came with the OCP. Mainly, lowering the fertility rate can result in demographic dividends (International Monetary Fund) that spur economic growth, a result of the diminished dependency rate (i.e., the ratio of the working populace to the non-working populace). Savings rates also increased by 7.5\% (Q. Liang), contributing 25\% to the growth of the Chinese economy, though this disputed, with some sources stating that the contribution of the OCP to saving rates is negligible (Song). The OCP was advantageous in other ways as well, increasing output per capita, narrowing the wealth and gap, and decreasing skill premium (Liao). However, there were also downsides presented by the OCP.  Families with more than one child spend more (Keyu), thus contributing to the economy. Often, this spending goes into education for the child, which serves to further the economic development in the future. The rise in dependency rates as a result of a rise in the proportion of the elderly population and legal obligation of China’s youth to support its elder will cause a contraction of the economy (Mansharamani). Unlike other Asian countries facing the same problems of a declining population of the working age, China has one of the highest female labor-participation rates in the world, being 60.5\% in 2023 (World Bank). This means that the potential effects for increasing female participation in labor is limited. According to some estimates, by 2035, 32.7\% of the population is expected to be 60+ (Economist Intelligence), 60 being the retirement age for males in China, females being even lower. As a result of the skewed gender ratios, men in China used the ownership of property as a sign of affluence, causing the formation of a housing crisis as men rush to purchase property and caused an oversaturation of the housing market. Construction has fallen by 60\% relative to pre-COVID levels (Hoyle), seriously impacting the Chinese economy. Further compounding the problem is the lingering cultural impact of the OCP, as the current generation does not want more children (Stanway), exasperating labor problems in the Chinese economy for the future. 

First, the shrinkage of the working age population in China is beginning to cause, its imports and exports will also begin to weaken, impacting the global economy. Just last month, China’s exports and imports declined by 7.5\% and 1.9\% respectively (Soo). Last year, China’s imports dropped by a little less than 9\%. This recent change reflects a growing trend of falling exports and imports, which can grow to have a profound effect on the global economy (Marsh). Companies such as Apple, Volkswagen, and Burberry rely heavily upon the consumer market in China (Marsh), had begun and will continually be impacted by the atrophying Chinese economy. There is evidence that in addition to the OCP, the apparent economic downturn wMarshas a result of the trade war between the United States and China, which has led to dramatically reductions in trade between the two countries (Bown).  

Second, the massive increase of saving rate, rising 20\% from 1983 to 2011 (Coeurdacier), was a result of many factors including low amounts of urbanization, lack of a social safety net, the strong growth rate of the economy, and the OCP (Hung). The result of this is that the money, instead of going back into the economy as money spent by consumers, instead wound up in peoples savings accounts. A high saving rate causes a litany of problems, including increased pressure on interest rates, causing excess investments, and leading to decreases in exchange rates (Song). A more balanced gender ratio may resolve this issue (Wei), and furthermore, reduce tensions between China and its trading partners.  

Third, the aforementioned gender imbalance brought about by the OCP in China has also caused a housing crisis, effecting the global housing market in the process. China is experiencing a decline in the demand for steel, cement, glass, etc. (Nunley), which is detrimental to the export economy of other nations. In addition to this, as a result of property owners’ unwillingness to sell their property at low rates, commercial property deals have sunk to record low levels globally (Real Estate Markets to Be Invigorated as Growing Pool of Buyers and Sellers Look to Make
a Splash.). Chinese investors who had invested aggressively in decades past have mostly resorted to off-loading projects in the face of the housing crisis (Callanan). In China, the fact that real-estate contributes to 22\% of the total GDP, combined with the fact that a drop of ~4.25\% in the Chinese GDP would cause a ~1.5\% drop in the GDP of advanced economies and a ~2.75\% drop in the GDP of emerging economies certainly makes the situation seem grim (Rogoff). 

The effects of the OCP, despite being highly controversial and contentious, will nonetheless remain highly relevant to not only the Chinese, but the global economy. It seems yet uncertain whether the recent policies implemented by the CCP encouraging births will reverse the current trend of the economy, which seems to be beginning to overturn its past record of phenomenal growth. It is also uncertain whether the rest of the world will also be dragged along with China into an economic slump. 


\nocite{*}
\printbibliography[title=References]
\end{document}
