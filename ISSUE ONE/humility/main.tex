\documentclass[12pt, a4paper, twoside]{article}
\usepackage{format}
\usepackage{tikz}
\usepackage{pgfplots}
\pgfplotsset{compat=1.18} 
% Do not alter above

% Metadata: put your article information here 
\newcommand{\jtitle}{Humility}
\newcommand{\jauthor}{Yichen Wang}
\newcommand{\jaffiliation}{Hong Kong International School}

% Editors will change these fields after acceptance 
\newcommand{\jvolume}{1}
\newcommand{\jyear}{2025}
% \newcommand{\jdoi}{N/A}  

% References should be placed in refs.bib and cited with \autocite{<source>}
% Quotations can be placed in quote environments: \begin{quote}<your quote>\end{quote}
% Footnotes can be added with \footnote{<your footnote>}

% Your Content

\begin{document}

\maketitle{}

When there is a case of peer disagreement, where two equally informed and rational individuals arrive at opposing beliefs about \emph{p}, a question arises: what should each person do once learning about the difference in opinion? Suppose A believes that \emph{p} is true, while B believes not-\emph{p}. After sharing evidence, A wonders if they should hold their judgment or reconsider. This puzzle is important in philosophy because it challenges the debate about humility or tenacity. Humility, as explained by Feldman, argues that A should suspend judgment in light of peer disagreement if fulfilling the following conditions: defeat, if an epistemic peer disagrees with p; equal weight, if your peer’s opinion is as strong as your own; and independence, if there are no reasons to discount your peer outside of the disagreement itself. In contrast, Kelly argues in defense of tenacity, which explains that A can reasonably maintain her belief by right. Richard Feldman proposes humility based on the premise that rationality requires consideration of epistemic peers wholeheartedly, acknowledging that we may be wrong. Thomas Kelly argues against this and states that if there is disagreement between epistemic peers, it does not mean that the opposing view or evidence has not been considered. Based on these two arguments, one is inclined to support Feldman. A rational person should suspend judgment in the face of peer epistemic disagreement because of symmetry in reasoning capacity. This paper will defend Feldman’s position by evaluating theoretical scenarios under either lens and discussing two of Kelly’s main objections to show how problematic assumptions should not interfere with the nature of evidence, belief, and rational disagreement. 

The advantages of Feldman’s humility over Kelly’s tenacity are illustrated when the two philosophies are used to analyze theoretical cases. An example of such is the \emph{Dean on the Quad Case}, where you and I, epistemic peers, disagree on seeing a dean on the quad, given the same evidence. In such a scenario, where you believe the dean is on the quad and I don’t, Feldman’s philosophy would have you suspend your judgment on the situation. Since you disagree with me and have been established to be as observant, honest, and capable as I am, the most rational course of action would be to withdraw any conclusions and wait for more evidence. On the other hand, under Kelly’s philosophy, you would maintain your beliefs. Your evidence suggests that the dean is on the quad, but I disagree. Since my opinion does not constitute evidence for the dean’s absence, the rational conclusion is to continue believing this until further proof is provided. By suspending judgment based on Feldman’s philosophy, further evidence must be viewed to conclude. Maintaining a difference of opinion in this case, under Kelly’s philosophy, is not helpful because it does not lead to a conclusion, only a difference of opinion. 

Another theoretical example is the \emph{Restaurant Check Case}, where, in the process of splitting a bill’s cost, you conclude that each person’s share is \$45, while your friend believes that it is \$43. Similar to the last case, both parties are epistemic peers with equal math abilities and evidence. In this situation, Feldman would once again have you suspend judgment, reasoning that your friend’s math abilities are identical to yours, which warrants less confidence in your conclusion. Therefore, Feldman would once again withhold conclusions before confirming either answer with a calculator. Conversely, Kelly would maintain that since you have no evidence to believe that your total is wrong, the correct action would be to pay the \$45. Suspending judgment in this case would be best pending further review of evidence, which supports Feldman’s theory of humility. 

Thomas Kelly presents a challenge to the humility view, contending that one should not suspend judgment when faced with peer disagreement. Kelly argues that “when people give reasons for their beliefs, they do not typically say things like 3. Normally, the fact that someone believes \emph{p} is a result of the evidence for \emph{p}, not the evidence itself for p.” Since we are epistemic peers, with equal intelligence, well informed, and capable of rational judgment, it is possible to form beliefs based on a body of evidence. If we were to counter someone’s belief as evidence, it is considered double-counting. For Kelly, treating B’s belief as evidence would essentially be the same as counting the underlying evidence. Both A and B have evaluated the evidence and come to opposing conclusions. If A were to decide that B’s evidence is true, then the same evidence is counted twice, since B already concluded the fact in the evidence on their own.

In addition to double-counting evidence, Thomas Kelly objects to the humility approach in an epistemic disagreement because of the asymmetry objection. Kelly contends, “suppose that, as it turns out, you and I disagree. From my perspective, of course, this means that you have misjudged the probative force of the evidence. The question then is this: why shouldn’t I take this difference between us as a relevant difference, one which effectively breaks the otherwise perfect symmetry? … From my vantage point–as one of the parties within the dispute, as opposed to some on-looking third party– it is just this undeniably relevant difference that divides us on this particular occasion” (15). Based on this premise, it seems as if A and B need to have their evaluation of the evidence. 

Given their evaluation of evidence, suspending judgment is not the appropriate course of action here. Given that A and B are true epistemic peers, neither should have reason to doubt the assessment that led to their conclusion, even if there is disagreement. Humility does not work in this context because it suggests that A and B should be doubtful of their evidence when they should instead trust their judgment. However, appealing to one’s own perspective does not break the symmetry in any meaningful way. If both are true epistemic peers, simply believing the other is mistaken adds nothing beyond restating the disagreement. From a neutral standpoint, the disagreement remains balanced, and the most rational response is to suspend judgment until further evidence is available.Such an approach implies circular reasoning, which is derived from A and B both believing that they are right because of their evidence. This does not work in this context because, contrary to what Kelly argues, Feldman supports the idea that it is okay to suspend judgment pending evaluation of evidence so that either A or B can be right.

Overall, while the two philosophies have their strengths and weaknesses, Feldman’s humility ultimately proves better than Kelly’s tenacity. Firstly, the practice of humility is typically the safer option, as seen in the scenarios. Feldman preaches not jumping to conclusions, opting to suspend judgment until further evidence is given. This is especially true for the Restaurant Check Case, where waiting for a confirmed total makes the most sense to avoid overpaying. Furthermore, the idea of humility is formed around inherent respect for epistemic peers. This not only protects the opinions of others but also fosters practicality in collaboration. Instead of picking a side, humility instead respects the judgment of others and opts for a shared search for the truth, disregarding personal opinions. 

When faced with the dilemma of disagreement between epistemic peers, the most rational course of action is to suspend judgment following Feldman’s view of humility. Although Thomas Kelly raises objections against Feldman, such as the risk of double-counting and asymmetry in evaluation, his argument falls short due to circular reasoning and shortsightedness. On the contrary, Feldman’s approach fosters open-mindedness and a genuine search for the truth over simply prevailing in an argument. His philosophy leads to more accurate conclusions in situations portrayed in the \emph{Dean in the Quad} and the \emph{Restaurant Check Cases}, and offers a perspective of mutual respect and collaboration. 








\printbibliography

\end{document}