
\documentclass[12pt]{article}
\usepackage[utf8]{inputenc}
\usepackage[T1]{fontenc}
\usepackage{geometry}
\usepackage{setspace}
\usepackage{microtype}
\usepackage{graphicx}
\usepackage{hyperref}
\usepackage{caption}
\usepackage{enumitem}
\usepackage{url}
\usepackage{titlesec}
\usepackage{ifthen}
\usepackage{amsmath}
\geometry{margin=1in}
\hypersetup{colorlinks=true,linkcolor=black,citecolor=black,urlcolor=blue}
\setlength{\parskip}{6pt}
\setlength{\parindent}{0pt}
\titleformat{\section}{\normalfont\Large\bfseries}{\thesection.}{0.5em}{}
\titleformat{\subsection}{\normalfont\large\bfseries}{\thesubsection.}{0.5em}{}

\title{A Binary Theory of Collapse}
\author{}
\date{}

\begin{document}
\maketitle

\section{Introduction}

To begin our inquiry into the nature of civilizational collapse, a definition of relevant terms is obligatory.

For the purposes of this essay, ``civilizational collapse'' shall be defined as a substantial decrease of human populations, and/or of political, scientific, or cultural complexity in some local area.\textsuperscript{1} Here, decreasing complexity is defined as the abandoning of advanced technology, economic regression, simplification of social bonds, loss of territory, increased decentralization, reduced trade, and/or curtailed information exchange.\textsuperscript{2}

In this essay, I contend that civilizations collapse due to anthropological factors, environmental factors, and anthropogenic climate change (ACC), otherwise known as ``global warming''. Moreover, I argue that our civilization will collapse since anthropogenic factors are immutable, environmental factors are unreliable and unpreventable, and ACC exhibits both properties.

\section{Binary Theory of Collapse}

The binary theory of collapse (BTC) comprises two major constituents (see fig.\ 1): anthropological and environmental. The anthropological factors are further divided into two components, what I will term the collective and individual factors, the former stems from biological necessity,\textsuperscript{3} and the latter from genetic circumstance.\textsuperscript{4} Environmental factors are of two causally connected elements: natural disasters and natural climate fluctuations (NCFs). The former can cause the latter, or, in rarer instances, be civilization-ending in and of itself. ACC, a uniquely modern product, will be discussed in section~IV.

\subsection{Anthropological Factors}

The anthropological factors are divided into two strands: collective and individual. Collective factors have the property of being universally applicable, while individual factors are restricted to, as the name would suggest, individuals. There are two basic types of collective factors: sustentative-reproductive and myopic.

Sustentative-reproductive factors are ones directed towards the maintenance and nourishment of human beings and the biological want to reproduce. Examples of this include food production, construction, and manufacturing of health-related products, all of which are crucial for reproduction. However, we effectuate unintended destruction of the natural environment, overhunting, overfishing, soil depletion, salinization, erosion, and so forth.\textsuperscript{5} This leads to a Malthusian population crisis, where population rises above the productive output of a civilization.

Myopic factors refer to ``present bias'' within the human decision-making process, i.e., we tend to prefer smaller rewards now, than larger rewards later.\textsuperscript{6} Evolutionarily, uncertainty regarding future rewards have adapted our brain to prioritize immediate gratification over delayed gratification.\textsuperscript{7} Myopic factors are especially prominent in societies with high Gini ratios. In such societies, the powerful are potentially unaffected by actions whence they benefit, while the lower classes suffer the brunt of the negative ramifications.\textsuperscript{8} For instance, while oil barons would certainly support fracking practices, the environmental consequences have the greatest effect for the impoverished.\textsuperscript{9}

Individual factors are similarly most applicable for individuals wielding power. There are two categories of individual factors: egoistic and irrational. Historic examples of individual factors are scarce when compared with evidence for the other factors, though this is not because they are less common. In many circumstances, records are scant or even non-existent, simply due to the great magnitude of time which has passed since the collapse of said civilization. Another possibility is that of revisionist historiography fueled by some ideological values, perhaps some concerted attempt at removing a figure from the historical record.\textsuperscript{10}

Egoistic factors pertain to selfishness and indifference to the plight of others. Selfishness, when combined with substantial amounts of power, can lead to unwarranted risk taking, abuse of subordinates, aggression, and duplicity, all of which produces instability and volatility.\textsuperscript{11}

Irrational factors denote acting seemingly without reason and foresight. Often, this may simply be the result of misinformation, inexperience, or misfortune, not necessarily an inherent illogicality. Collapse caused by irrational factors can be likened to the Red Queen Hypothesis, in that such collapse would be irregular and erratic.\textsuperscript{12}

\subsection{Environmental Factors}

Environmental factors are twofold: natural disasters and natural climate fluctuations.

Natural disasters have been recorded, though only on rare occasions, to be the direct cause of civilizational collapse. There have been examples of invasive diseases, immense floods and volcanic eruptions leading to the destruction of civilization.\textsuperscript{13} Nonetheless, it should be noted that natural disasters typically only have a small area of effect, which nullifies it ability to directly injure large civilizations.

More frequently, however, it is the NCFs caused by some natural disaster which carries the most potent destructive power. For example, a declining, plague-stricken population could cause reforestation due to their impaired ability to engage in logging. Accordingly, the natural carbon sink grows, reducing the global temperature, leading to crop failures.\textsuperscript{14}

NCFs could also occur independently, such as through El Ni\~no and La Ni\~na, altered ocean currents, Milankovitch cycles, or changes in solar activity.\textsuperscript{15} These have been recorded to produce adverse effects in civilizations.\textsuperscript{16}

\section{Examples}

The collapse of Pacific Island societies provides a striking illustration of the impacts of the substantive-reproductive factors. The M\=aori peoples, who, around the 14th century, settled New Zealand, hunted many native species to extinction, including the moa and the New Zealand swan.\textsuperscript{17} They also extensively reduced the population of species such as fur seals and sea lions.\textsuperscript{18} These factors, combined further with comprehensive deforestation, caused a drastic decline in the local population. Similar trends were observed in other locations, including the Mangareva, Henderson, and Kaho\char`‘olawe Islands.\textsuperscript{19} Additionally, myopic and anthropogenic factors could have been at play. However, due to the lack of information regarding the pre-European history of the Polynesian islands, this is difficult to ascertain.

From the Late Bronze Age Collapse, we can find numerous examples showcasing the impact of natural disasters on the environment. The Hekla 3 eruption, occurring around the 12th century BCE, caused famines and compounded preexisting droughts.\textsuperscript{20} Consequently, human immune systems were weakened, and bubonic plague became widespread.\textsuperscript{21} The Sea Peoples, often cited as the primary cause for the Late Bronze Age Collapse, began their exodus to the major Mediterranean civilizations, driven by droughts of their own.\textsuperscript{22} All the aforementioned factors, combined with the increasing wealth gap of the Late Bronze Age, led to the downfall of those once mighty pillars of civilization.\textsuperscript{23}

The Byzantine historian Procopius describes egoistic factors at play in the collapse of the Western Roman Empire. As he describes, when emperor Honorius was informed of the sack of Rome in 410 CE, he cried, ``And yet it has just eaten from my hands!'', referring to his favorite pet chicken, coincidentally named \textit{Roma}. When it was explained that the city had fallen, and not his pet chicken, Honorius supposedly sighed with relief.\textsuperscript{24} Though the story is widely believed to be apocryphal, it aptly demonstrates the incompetence and self-absorption of the ruling elite.\textsuperscript{25} Many of the final Western Roman Emperors in the 5th century shared in these qualities.\textsuperscript{26} Environmental factors likewise contributed greatly to the fall of the empire. In the third century, the Northwestern provinces saw climate fluctuations; a century later, severe droughts forced the Huns to migrate into Roman territory and ultimately weakened Rome.\textsuperscript{27} Extensive deforestation for arable land and air pollution from incinerated lumber was also recorded.\textsuperscript{28} Once again, we find that wealth inequality can be observed, as records show that \textasciitilde1.4\% of the population controlled around 16--29\% of the total wealth.\textsuperscript{29}

For another example of individual factors at play in civilizational collapse, we can turn to the Mughal Empire. Akbar the Great, the founder of the empire, promoted a policy of religious toleration, in the process improving the governing bureaucracy and ensuring stability.\textsuperscript{30} However, though the system was carefully maintained by the next two emperors, emperor Aurangzeb, the third after Akbar, increasingly favored Muslims in his rulings, abandoning the policy of toleration.\textsuperscript{31} Thus, the Mughal empire became progressively more fractured and debilitated until it was finally subsumed by the British East India Company in the mid-18th century.\textsuperscript{32}

\section{Our collapse}

Based on the metrics proposed in the BTC, it is more than likely that human civilization will collapse. Let us analyze each factor of the BTC individually and delineate the way in which they are applicable today:

First, the sustentative-reproductive factor; namely, the conjoined problems of overpopulation and food scarcity. Currently, 10\% of the world faces chronic hunger.\textsuperscript{33} By 2050, we would need to double our current agricultural output in order to meet consumption demands.\textsuperscript{34} However, in attempting to increase agricultural output, we would cause the toxification and pollution of the environment.\textsuperscript{35} Yet, said pollution cyclically contributes to the stunting of crop yields.\textsuperscript{36} This may result in a positive feedback loop, resulting in both further environmental degradation and instances of persistent hunger and famine.

Second, myopic factors, which can be seen in many of the world governments and international organizations today. For instance, despite progress made against overconsumption by American President Jimmy Carter in the late 1970s, the Reagan administration reversed this trend in support of their ideological values and to curry favor with the oil industry.\textsuperscript{37} More recently, despite the optimistic objectives set during the 2015 Paris Agreement, none of the top four emitters -- the United States, China, the EU, and India -- have met their emission reduction targets.\textsuperscript{38}

Third, individual factors, which can be difficult to identify in the short-term. However, this factor can be found in the conducts of American President Donald Trump, who has taken an ambiguous stance towards ACC despite clear evidence of its existence.\textsuperscript{39} Furthermore, Trump’s Affordable Clean Energy policy, which supplanted the Clean Power Plan, increased emissions despite aiming to do the opposite.\textsuperscript{40}

Fourth, NCFs. Although mostly overshadowed by ACC, NCFs still play a key role in global climate variation.\textsuperscript{41} Climate patterns such as El Ni\~no and La Ni\~na will continue to impact precipitation and temperature in the Americas and other parts of the world.\textsuperscript{42} Additionally, solar output levels and Milankovitch cycles will remain, albeit to a minimal degree, a consideration in climate trends.\textsuperscript{43}

Lastly, we must contend with the greatest existential menace facing humanity today: ACC.\textsuperscript{44} Occupying the divide between the anthropological and environmental factors, ACC is the most concrete and substantial of the previously considered factors of collapse in our modern day. While originating with human activity, ACC impacts us in ways more akin to environmental factors.\textsuperscript{45}

Beginning with the industrial revolution, global temperature has risen by around 1.1~\textdegree C or 2~\textdegree F (see fig.\ 2), which, despite seemingly being insignificant, has momentous consequences for the future of civilizations.\textsuperscript{46} The increased temperature leads to more frequent wildfires, increased destructiveness and regularity of storms, higher frequency of drought, rising sea levels, magnified health risks, etc.\textsuperscript{47}

These factors, all interconnected and reciprocally collaborating, can contribute to civilizational collapse, as demonstrated by figure~3.\textsuperscript{48} When compared with past climate events, ACC is of a much greater magnitude.\textsuperscript{49} Despite modern scientific advancements in reducing emissions and green energy, global temperatures are expected to continue to rise well into the future.\textsuperscript{50} Taking into account the effect climate change events have had on human civilizations hitherto, the future seems bleak.\textsuperscript{51}

Lastly, with the advent of globalization, the collapse of one civilization could spread to other nations at an alarming rate. If a pandemic on the scale of the 1917 Spanish Flu were to occur in the modern day, it would kill nearly 33 million people in just the first six months.\textsuperscript{52} We can reasonably conclude that the magnitude of civilizational collapses in the contemporary era would surpass all previous collapses witnessed in the Anthropocene.\textsuperscript{53}

\section{Conclusion}

Supported by the aforementioned factors, I thus assert that civilizations collapse due to anthropological factors, environmental factors, and, more recently, ACC. In addition, as the evidence compiled in section~IV would suggest, I contend that our civilization will collapse.

\clearpage
\section*{Appendix}

\begin{figure}[htbp]
  \centering
  % Uncomment the next line if the PDF file is available alongside this .tex file.
  %\includegraphics[page=9,width=\textwidth]{A Binary Theory of Collapse.pdf}
  \caption{Venn diagram representing the binary theory of collapse. (Chart by author)}
\end{figure}

\begin{figure}[htbp]
  \centering
  %\includegraphics[page=10,width=\textwidth]{A Binary Theory of Collapse.pdf}
  \caption{The change in global temperature since 0~AD, chart ``Global temperature change over the last 2019 years,'' from Ed Hawkins, \textit{2019 years}.}
\end{figure}

\begin{figure}[htbp]
  \centering
  %\includegraphics[page=11,width=\textwidth]{A Binary Theory of Collapse.pdf}
  \caption{Causal loop diagram illustrating global climate failure, chart ``Cascading global climate failure,'' from Kemp et al., ``Climate Endgame: Exploring catastrophic climate change scenarios'', \textit{PNAS}, 119, no.\ 34 (2022), fig.\ 3, \url{https://doi.org/10.1073/pnas.2108146119}.}
\end{figure}

\clearpage
\section*{Endnotes}
\begin{enumerate}[leftmargin=*]
\item Diamond, ``Ecological Collapses of Past Civilizations,'' 363.
\item \textit{The Collapse of Complex Societies} - Professor Joseph Tainter; Dourado, ``A Beginner’s Guide to Sociopolitical Collapse''; Tainter, \textit{The Collapse of Complex Societies}, chap.\ 1.
\item Santos and Rosati, ``The Evolutionary Roots of Human Decision Making,'' 13.
\item Goriounova and Mansvelder, ``Genes, Cells and Brain Areas of Intelligence,'' 8; Wu et al., ``The Genetic Mechanism of Selfishness and Altruism in Parent-Offspring Coadaptation,'' 1--3.
\item Jared Diamond- \textit{Collapse}.
\item Chakraborty, ``Present Bias,'' 1.
\item Albrecht et al., ``What Do I Want and When Do I Want It,'' 1.
\item Paulson Jr., ``Short-Termism and the Threat from Climate Change | McKinsey.''
\item Lin-Schweitzer, ``Integrated Effort Needed to Mitigate Fracking While Protecting Both Humans and the Environment.''
\item Burton, ``Akhenaten''; Gessen, ``The Photo Book That Captured How the Soviet Regime Made the Truth Disappear.''
\item Simmons, ``How Narcissistic Leaders Destroy from Within''; Blanton et al., ``Moral Collapse and State Failure,'' 9--10.
\item Kemp, ``Are We on the Road to Civilisation Collapse?''
\item Ehrenpreis and Ehrenpreis, ``A Historical Perspective of Healthcare Disparity and Infectious Disease in the Native American Population,'' 289; Rincon, ``Prehistoric North Sea `Atlantis' Hit by 5m Tsunami''; Pearson et al., ``Annual Radiocarbon Record Indicates 16th Century BCE Date for the Thera Eruption,'' 1.
\item Nevle et al., ``Neotropical Human--Landscape Interactions, Fire, and Atmospheric CO$_2$ during European Conquest,'' 9; Juurakko, diCenzo, and Walker, ``Cold Acclimation and Prospects for Cold-Resilient Crops.''
\item Loury, ``DROUGHT''; Wanamaker et al., ``Surface Changes in the North Atlantic Meridional Overturning Circulation during the Last Millennium,'' 1; Berger, Loutre, and Mélice, ``Equatorial Insolation,'' 6; Lockwood et al., ``Are Cold Winters in Europe Associated with Low Solar Activity?,'' 5--6.
\item ``Did Climate Influence Angkor’s Collapse?''; Gill et al., ``DROUGHT AND THE MAYA COLLAPSE,'' 292.
\item Walter et al., ``Mass Migration and the Polynesian Settlement of New Zealand,'' 358; Allentoft et al., ``Extinct New Zealand Megafauna Were Not in Decline before Human Colonization,'' 4922; Rawlence et al., ``Ancient DNA and Morphometric Analysis Reveal Extinction and Replacement of New Zealand’s Unique Black Swans,'' 1.
\item Wilmshurst, ``Pre-European Deforestation''; Emami-Khoyi et al., ``Mitogenomics Data Reveal Effective Population Size, Historical Bottlenecks, and the Effects of Hunting on New Zealand Fur Seals (\textit{Arctocephalus forsteri}),'' 1.
\item Fagan, \textit{The Great Warming}, 5; Diamond, \textit{Collapse}, 132; Rolett and Diamond, ``Environmental Predictors of Pre-European Deforestation on Pacific Islands,'' 443; Diamond, ``Ecological Collapses of Past Civilizations,'' 366.
\item Baker et al., ``The Hekla 3 Volcanic Eruption Recorded in a Scottish Speleothem?,'' 340; Yurko, ``End of the Late Bronze Age and Other Crisis Periods: A Volcanic Cause?,'' 456--58; Manning et al., ``Severe Multi-Year Drought Coincident with Hittite Collapse around 1198--1196 BC,'' 719.
\item Spyrou et al., ``Analysis of 3800-Year-Old \textit{Yersinia pestis} Genomes Suggests Bronze Age Origin for Bubonic Plague,'' 1.
\item Carpenter, \textit{Discontinuity in Greek Civilization}, 16--17.
\item Weisweiler, ``Inequality''; Basri and Lawrence, ``Wealth Inequality in the Ancient Near East,'' 20--22.
\item Procopius, \textit{History of the Wars}, Books III and IV.
\item Gibbon, \textit{The History of the Decline and Fall of the Roman Empire}, 643.
\item The Editors of Encyclopaedia Britannica, ``Majorian.''
\item McCormick et al., ``Climate Change during and after the Roman Empire,'' 185; McCormick et al., 190--91.
\item Harris, \textit{The Ancient Mediterranean Environment between Science and History}, 173; Erskine, \textit{A Companion to Ancient History}, 171.
\item Scheidel and Friesen, ``The Size of the Economy and the Distribution of Income in the Roman Empire,'' 76.
\item Ballhatchet, ``Akbar''; Stein, \textit{A History of India}, 159.
\item Pletcher, \textit{The History of India}, 183.
\item Blanton et al., ``Moral Collapse and State Failure,'' 8; ``What Was the East India Company?''
\item Omer, ``Global Hunger.''
\item Ranganathan et al., ``How to Sustainably Feed 10 Billion People by 2050, in 21 Charts.''
\item Chowdhury et al., ``Does Agricultural Ecology Cause Environmental Degradation?,'' 3.
\item Jordan, ``Pollution and Crops.''
\item McKibben and Speth, ``They Knew: How the U.S. Government Helped Cause the Climate Crisis.''
\item Bearak and Popovich, ``The World Is Falling Short of Its Climate Goals. Four Big Emitters Show Why.''
\item Cheung, ``What Does Trump Actually Believe on Climate Change?''
\item Keyes et al., ``The Affordable Clean Energy Rule and the Impact of Emissions Rebound on Carbon Dioxide and Criteria Air Pollutant Emissions,'' 9.
\item Shaftel, ``Climate Change.''
\item Halpert, ``United States El Ni\~no Impacts | NOAA Climate.Gov.''
\item NASA Science Editorial Team, ``What Is the Sun’s Role in Climate Change?''; NASA Science Editorial Team, ``Why Milankovitch (Orbital) Cycles Can’t Explain Earth’s Current Warming - NASA Science.''
\item United Nations, ``Climate Change `Biggest Threat Modern Humans Have Ever Faced', World-Renowned Naturalist Tells Security Council, Calls for Greater Global Cooperation.''
\item Intergovernmental Panel on Climate Change, \textit{Climate Change 2013}, 896.
\item Lindsey and Dahlman, ``Climate Change.''
\item Diffenbaugh, Konings, and Field, ``Atmospheric Variability Contributes to Increasing Wildfire Weather but Not as Much as Global Warming,'' 1; Aumann, Ruzmaikin, and Teixeira, ``Frequency of Severe Storms and Global Warming,'' 4; Dai, ``Drought under Global Warming,'' 14--16; Cook et al., ``Global Warming and 21st Century Drying,'' 1; Meehl et al., ``How Much More Global Warming and Sea Level Rise?,'' 1; Vermeer, Rahmstorf, and Clark, ``Global Sea Level Linked to Global Temperature,'' 5; US EPA, ``Climate Change Impacts on Air Quality''; United Nations, ``Causes and Effects of Climate Change.''
\item Kanter, ``Scientist''; Kemp et al., ``Climate Endgame,'' 7.
\item ``Global Warming.''
\item Center for Science Education, ``Predictions of Future Global Climate.''
\item National Geographic Society, \textit{National Geographic Essential Visual History of the World}, 190--91.
\item Gates, ``Shattuck Lecture Innovation for Pandemics.''
\item Juling, ``Future Bioterror and Biowarfare Threats for NATO’s Armed Forces until 2030.''
\end{enumerate}

\clearpage
\section*{Bibliography}
\begin{enumerate}[leftmargin=*]
\item Albrecht, Konstanze, Kirsten G. Volz, Matthias Sutter, and D. Yves von Cramon. ``What Do I Want and When Do I Want It: Brain Correlates of Decisions Made for Self and Other.'' \textit{PLoS ONE} 8, no.\ 8 (August 22, 2013): e73531. \url{https://doi.org/10.1371/journal.pone.0073531}.
\item Allentoft, Morten Erik, Rasmus Heller, Charlotte L. Oskam, Eline D. Lorenzen, Marie L. Hale, M. Thomas P. Gilbert, Christopher Jacomb, Richard N. Holdaway, and Michael Bunce. ``Extinct New Zealand Megafauna Were Not in Decline before Human Colonization.'' \textit{Proceedings of the National Academy of Sciences of the United States of America} 111, no.\ 13 (April 1, 2014): 4922--27. \url{https://doi.org/10.1073/pnas.1314972111}.
\item Aumann, Hartmut H., Alexander Ruzmaikin, and Joao Teixeira. ``Frequency of Severe Storms and Global Warming.'' \textit{Geophysical Research Letters} 35, no.\ 19 (2008). \url{https://doi.org/10.1029/2008GL034562}.
\item Baker, Andy, Peter L. Smart, W. L. Barnes, R. Lawrence Edwards, and Andy Farrant. ``The Hekla 3 Volcanic Eruption Recorded in a Scottish Speleothem?'' \textit{The Holocene} 5, no.\ 3 (1995): 336.
\item Ballhatchet, Kenneth. ``Akbar,'' May 27, 2024. \url{https://www.britannica.com/biography/Akbar}.
\item Basri, Pertev, and Dan Lawrence. ``Wealth Inequality in the Ancient Near East: A Preliminary Assessment Using Gini Coefficients and Household Size.'' \textit{Cambridge Archaeological Journal} 30, no.\ 4 (November 2020): 689--704. \url{https://doi.org/10.1017/S0959774320000177}.
\item Bearak, Max, and Nadja Popovich. ``The World Is Falling Short of Its Climate Goals. Four Big Emitters Show Why.'' \textit{The New York Times}, November 8, 2022, sec.\ Climate. \url{https://www.nytimes.com/interactive/2022/11/08/climate/cop27-emissions-country-compare.html}.
\item Berger, A., M. F. Loutre, and J. L. Mélice. ``Equatorial Insolation: From Precession Harmonics to Eccentricity Frequencies.'' \textit{Climate of the Past} 2, no.\ 2 (October 12, 2006): 131--36. \url{https://doi.org/10.5194/cp-2-131-2006}.
\item Blanton, Richard E., Gary M. Feinman, Stephen A. Kowalewski, and Lane F. Fargher. ``Moral Collapse and State Failure: A View From the Past.'' \textit{Frontiers in Political Science} 2 (October 16, 2020). \url{https://doi.org/10.3389/fpos.2020.568704}.
\item Burton, Elizabeth. ``Akhenaten: The Forgotten Pioneer of Atenism and Monotheism.'' \textit{TheCollector}, May 9, 2020. \url{https://www.thecollector.com/akhenaten-monotheism/}.
\item Carpenter, Rhys. \textit{Discontinuity in Greek Civilization}. Cambridge\,; London\,: Cambridge U.P., 1966. \url{http://archive.org/details/discontinuitying0000carp}.
\item Center for Science Education. ``Predictions of Future Global Climate.'' UCAR Center for Science Education, 2021. \url{https://scied.ucar.edu/learning-zone/climate-change-impacts/predictions-future-global-climate}.
\item Chakraborty, Anujit. ``Present Bias.'' \textit{Econometrica} 89, no.\ 4 (2021): 1921--61. \url{https://doi.org/10.3982/ECTA16467}.
\item Cheung, Helier. ``What Does Trump Actually Believe on Climate Change?,'' January 23, 2020. \url{https://www.bbc.com/news/world-us-canada-51213003}.
\item Chowdhury, Shanjida, Sunjida Khan, Md Fouad Hossain Sarker, Md Kabirul Islam, Maruf Ahmed Tamal, and Niaz Ahmed Khan. ``Does Agricultural Ecology Cause Environmental Degradation? Empirical Evidence from Bangladesh.'' \textit{Heliyon} 8, no.\ 6 (June 18, 2022): e09750. \url{https://doi.org/10.1016/j.heliyon.2022.e09750}.
\item Columbia Climate School The Earth Institute. ``Did Climate Influence Angkor’s Collapse?,'' March 29, 2010. \url{https://www.earth.columbia.edu/articles/view/2661}.
\item Cook, Benjamin I., Jason E. Smerdon, Richard Seager, and Sloan Coats. ``Global Warming and 21st Century Drying.'' \textit{Climate Dynamics} 43, no.\ 9 (November 1, 2014): 2607--27. \url{https://doi.org/10.1007/s00382-014-2075-y}.
\item Dai, Aiguo. ``Drought under Global Warming: A Review.'' \textit{WIREs Climate Change} 2, no.\ 1 (2011): 45--65. \url{https://doi.org/10.1002/wcc.81}.
\item Diamond, Jared. \textit{Collapse: How Societies Choose to Fail Or Succeed}. Penguin, 2005.
\item ---{}---. ``Ecological Collapses of Past Civilizations.'' \textit{Proceedings of the American Philosophical Society} 138, no.\ 3 (1994): 363--70.
\item Diffenbaugh, Noah S., Alexandra G. Konings, and Christopher B. Field. ``Atmospheric Variability Contributes to Increasing Wildfire Weather but Not as Much as Global Warming.'' \textit{Proceedings of the National Academy of Sciences} 118, no.\ 46 (November 16, 2021): e2117876118. \url{https://doi.org/10.1073/pnas.2117876118}.
\item Dourado, Eli. ``A Beginner’s Guide to Sociopolitical Collapse,'' January 30, 2023. \url{https://www.elidourado.com/p/collapse}.
\item Ehrenpreis, Jamie E., and Eli D. Ehrenpreis. ``A Historical Perspective of Healthcare Disparity and Infectious Disease in the Native American Population.'' \textit{The American Journal of the Medical Sciences} 363, no.\ 4 (April 2022): 288--94. \url{https://doi.org/10.1016/j.amjms.2022.01.005}.
\item Emami-Khoyi, Arsalan, Adrian Paterson, David Hartley, Laura Boren, Robert Cruickshank, James Ross, and Terry-Ann Else. ``Mitogenomics Data Reveal Effective Population Size, Historical Bottlenecks, and the Effects of Hunting on New Zealand Fur Seals (\textit{Arctocephalus Forsteri}).'' \textit{Mitochondrial DNA Part A: DNA Mapping, Sequencing, and Analysis} 29, no.\ 4 (January 1, 2018): 567--80. \url{https://doi.org/10.1080/24701394.2017.1325478}.
\item Erskine, Andrew. \textit{A Companion to Ancient History}. John Wiley \& Sons, 2012.
\item Fagan, Brian. \textit{The Great Warming: Climate Change and the Rise and Fall of Civilizations}. Bloomsbury Publishing USA, 2010.
\item Gates, Bill. ``Shattuck Lecture Innovation for Pandemics.'' Bill \& Melinda Gates Foundation, 2018. \url{https://www.gatesfoundation.org/ideas/speeches/2018/04/shattuck-lecture-innovation-for-pandemics}.
\item Gessen, Masha. ``The Photo Book That Captured How the Soviet Regime Made the Truth Disappear.'' \textit{The New Yorker}, July 15, 2018. \url{https://www.newyorker.com/culture/photo-booth/the-photo-book-that-captured-how-the-soviet-regime-made-the-truth-disappear}.
\item Gibbon, Edward. \textit{The History of the Decline and Fall of the Roman Empire}. Philadelphia\,: Porter \& Coates, 1845.
\item Gill, Richardson B., Paul A. Mayewski, Johan Nyberg, Gerald H. Haug, and Larry C. Peterson. ``DROUGHT AND THE MAYA COLLAPSE.'' \textit{Ancient Mesoamerica} 18, no.\ 2 (October 2007): 283--302. \url{https://doi.org/10.1017/S0956536107000193}.
\item ``Global Warming.'' Text.Article. NASA Earth Observatory, June 3, 2010. \url{https://earthobservatory.nasa.gov/features/GlobalWarming/page3.php}.
\item Goriounova, Natalia A., and Huibert D. Mansvelder. ``Genes, Cells and Brain Areas of Intelligence.'' \textit{Frontiers in Human Neuroscience} 13 (February 15, 2019): 44. \url{https://doi.org/10.3389/fnhum.2019.00044}.
\item Halpert, Mike. ``United States El Niño Impacts | NOAA Climate.Gov,'' June 12, 2014. \url{http://www.climate.gov/news-features/blogs/enso/united-states-el-ni%C3%B1o-impacts-0}.
\item Harris, William V., ed. \textit{The Ancient Mediterranean Environment between Science and History}. Brill, 2013. \url{https://brill.com/edcollbook/title/24269}.
\item Intergovernmental panel on climate change, ed. \textit{Climate Change 2013: The Physical Science Basis}. New York: Cambridge University Press, 2014.
\item Jared Diamond- \textit{Collapse: How Societies Choose to Fail or Succeed}, 2017. \url{https://www.youtube.com/watch?v=KYegWOTFqGI}.
\item Jordan, Rob. ``Pollution and Crops.'' \textit{StanfordReport}, June 1, 2022. \url{https://news.stanford.edu/stories/2022/06/pollution-and-crops}.
\item Juling, Dominik. ``Future Bioterror and Biowarfare Threats for NATO’s Armed Forces until 2030.'' \textit{Journal of Advanced Military Studies} 14, no.\ 1 (June 30, 2023): 118--43. \url{https://doi.org/10.21140/mcuj.20231401005}.
\item Juurakko, Collin L., George C.\ diCenzo, and Virginia K.\ Walker. ``Cold Acclimation and Prospects for Cold-Resilient Crops.'' \textit{Plant Stress} 2 (December 1, 2021): 100028. \url{https://doi.org/10.1016/j.stress.2021.100028}.
\item Kanter, James. ``Scientist: Warming Could Cut Population to 1 Billion.'' \textit{The New York Times}, 2009, sec.\ Opinion. \url{https://archive.nytimes.com/dotearth.blogs.nytimes.com/2009/03/13/scientist-warming-could-cut-population-to-1-billion/}.
\item Kemp, Luke. ``Are We on the Road to Civilisation Collapse?'' BBC, 2019. \url{https://www.bbc.com/future/article/20190218-are-we-on-the-road-to-civilisation-collapse}.
\item Kemp, Luke, Chi Xu, Joanna Depledge, Kristie L. Ebi, Goodwin Gibbins, Timothy A. Kohler, Johan Rockström, et al. ``Climate Endgame: Exploring Catastrophic Climate Change Scenarios.'' \textit{Proceedings of the National Academy of Sciences of the United States of America} 119, no.\ 34 (August 23, 2022): e2108146119. \url{https://doi.org/10.1073/pnas.2108146119}.
\item Keyes, Amelia T., Kathleen F. Lambert, Dallas Burtraw, Jonathan J. Buonocore, Jonathan I. Levy, and Charles T. Driscoll. ``The Affordable Clean Energy Rule and the Impact of Emissions Rebound on Carbon Dioxide and Criteria Air Pollutant Emissions.'' \textit{Environmental Research Letters} 14, no.\ 4 (April 2019): 044018. \url{https://doi.org/10.1088/1748-9326/aafe25}.
\item Lindsey, Rebecca, and Luann Dahlman. ``Climate Change: Global Temperature | NOAA Climate.Gov,'' January 18, 2024. \url{http://www.climate.gov/news-features/understanding-climate/climate-change-global-temperature}.
\item Lin-Schweitzer, Anna. ``Integrated Effort Needed to Mitigate Fracking While Protecting Both Humans and the Environment.'' Yale School of Public Health, March 30, 2022. \url{https://ysph.yale.edu/news-article/integrated-effort-needed-to-mitigate-fracking-while-protecting-both-humans-and-the-environment/}.
\item Lockwood, M., R. G. Harrison, T. Woollings, and S. K. Solanki. ``Are Cold Winters in Europe Associated with Low Solar Activity?'' \textit{Environmental Research Letters} 5, no.\ 2 (April 2010): 024001. \url{https://doi.org/10.1088/1748-9326/5/2/024001}.
\item Loury, Romain. ``DROUGHT: What next after El Niño? La Niña?'' \textit{Spore}, no.\ 183 (2016): 8--8.
\item Manning, Sturt W., Cindy Kocik, Brita Lorentzen, and Jed P. Sparks. ``Severe Multi-Year Drought Coincident with Hittite Collapse around 1198--1196 BC.'' \textit{Nature} 614, no.\ 7949 (February 2023): 719--24. \url{https://doi.org/10.1038/s41586-022-05693-y}.
\item McCormick, Michael, Ulf Büntgen, Mark A. Cane, Edward R. Cook, Kyle Harper, Peter John Huybers, Thomas Litt, et al. ``Climate Change during and after the Roman Empire: Reconstructing the Past from Scientific and Historical Evidence.'' \textit{Journal of Interdisciplinary History}, 2012. \url{https://doi.org/10.1162/JINH_a_00379}.
\item McKibben, Bill, and James Speth. ``They Knew: How the U.S. Government Helped Cause the Climate Crisis.'' Yale E360, September 15, 2021. \url{https://e360.yale.edu/features/they-knew-how-the-u-s-government-helped-cause-the-climate-crisis}.
\item Meehl, Gerald A., Warren M. Washington, William D. Collins, Julie M. Arblaster, Aixue Hu, Lawrence E. Buja, Warren G. Strand, and Haiyan Teng. ``How Much More Global Warming and Sea Level Rise?'' \textit{Science} 307, no.\ 5716 (March 18, 2005): 1769--72. \url{https://doi.org/10.1126/science.1106663}.
\item NASA Science Editorial Team. ``What Is the Sun’s Role in Climate Change? - NASA Science,'' 2019. \url{https://science.nasa.gov/earth/climate-change/what-is-the-suns-role-in-climate-change/}.
\item ---{}---. ``Why Milankovitch (Orbital) Cycles Can’t Explain Earth’s Current Warming - NASA Science,'' 2020. \url{https://science.nasa.gov/science-research/earth-science/why-milankovitch-orbital-cycles-cant-explain-earths-current-warming/}.
\item National Geographic Society. \textit{National Geographic Essential Visual History of the World}. National Geographic Books, 2007.
\item National Trust. ``What Was the East India Company?'' Accessed June 21, 2024. \url{https://www.nationaltrust.org.uk/discover/history/what-was-the-east-india-company}.
\item Nevle, Richard, D. Bird, W. Ruddiman, and Robert Dull. ``Neotropical Human--Landscape Interactions, Fire, and Atmospheric CO$_2$ during European Conquest.'' \textit{Holocene} 21 (July 29, 2011): 853--64. \url{https://doi.org/10.1177/0959683611404578}.
\item Omer, Sevil. ``Global Hunger: 7 Facts You Need to Know.'' \textit{World Vision} (blog), March 18, 2024. \url{https://www.worldvision.org/hunger-news-stories/world-hunger-facts}.
\item Paulson Jr., Henry. ``Short-Termism and the Threat from Climate Change | McKinsey,'' April 1, 2015. \url{https://www.mckinsey.com/capabilities/strategy-and-corporate-finance/our-insights/short-termism-and-the-threat-from-climate-change}.
\item Pearson, Charlotte L., Peter W. Brewer, David Brown, Timothy J. Heaton, Gregory W. L. Hodgins, A. J. Timothy Jull, Todd Lange, and Matthew W. Salzer. ``Annual Radiocarbon Record Indicates 16th Century BCE Date for the Thera Eruption.'' \textit{Science Advances} 4, no.\ 8 (August 15, 2018): eaar8241. \url{https://doi.org/10.1126/sciadv.aar8241}.
\item Pletcher, Kenneth, ed. \textit{The History of India}. Britannica Educational Publishing, 2010.
\item Procopius. \textit{History of the Wars, Books III and IV: The Vandalic War}, 1916. \url{https://www.gutenberg.org/ebooks/16765/pg16765-images.html}.
\item Ranganathan, Janet, Richard Waite, Tim Searchinger, and Craig Hanson. ``How to Sustainably Feed 10 Billion People by 2050, in 21 Charts,'' December 5, 2018. \url{https://www.wri.org/insights/how-sustainably-feed-10-billion-people-2050-21-charts}.
\item Rawlence, Nicolas J., Afroditi Kardamaki, Luke J. Easton, Alan J. D. Tennyson, R. Paul Scofield, and Jonathan M. Waters. ``Ancient DNA and Morphometric Analysis Reveal Extinction and Replacement of New Zealand’s Unique Black Swans.'' \textit{Proceedings of the Royal Society B: Biological Sciences} 284, no.\ 1859 (July 26, 2017): 20170876. \url{https://doi.org/10.1098/rspb.2017.0876}.
\item Rincon, Paul. ``Prehistoric North Sea `Atlantis' Hit by 5m Tsunami.'' \textit{BBC News}, May 1, 2014, sec.\ Science \& Environment. \url{https://www.bbc.com/news/science-environment-27224243}.
\item Rolett, Barry, and Jared Diamond. ``Environmental Predictors of Pre-European Deforestation on Pacific Islands.'' \textit{Nature} 431, no.\ 7007 (September 23, 2004): 443--46. \url{https://doi.org/10.1038/nature02801}.
\item Santos, Laurie R., and Alexandra G. Rosati. ``The Evolutionary Roots of Human Decision Making.'' \textit{Annual Review of Psychology} 66 (January 3, 2015): 321--47. \url{https://doi.org/10.1146/annurev-psych-010814-015310}.
\item Scheidel, Walter, and Steven J. Friesen. ``The Size of the Economy and the Distribution of Income in the Roman Empire.'' \textit{The Journal of Roman Studies} 99 (November 2009): 61--91. \url{https://doi.org/10.3815/007543509789745223}.
\item Shaftel, Holly. ``Climate Change: Vital Signs of the Planet.'' \textit{Climate Change: Vital Signs of the Planet}, 2023. \url{https://climate.nasa.gov/what-is-climate-change.amp}.
\item Simmons, Lee. ``How Narcissistic Leaders Destroy from Within.'' Stanford Graduate School of Business, May 15, 2024. \url{https://www.gsb.stanford.edu/insights/how-narcissistic-leaders-destroy-within}.
\item Spyrou, Maria A., Rezeda I. Tukhbatova, Chuan-Chao Wang, Aida Andrades Valtueña, Aditya K. Lankapalli, Vitaly V. Kondrashin, Victor A. Tsybin, et al. ``Analysis of 3800-Year-Old \textit{Yersinia pestis} Genomes Suggests Bronze Age Origin for Bubonic Plague.'' \textit{Nature Communications} 9, no.\ 1 (June 8, 2018): 2234. \url{https://doi.org/10.1038/s41467-018-04550-9}.
\item Stein, Burton. \textit{A History of India}. John Wiley \& Sons, 2010.
\item Tainter, Joseph. \textit{The Collapse of Complex Societies}. Cambridge University Press, 1988.
\item \textit{The Collapse of Complex Societies} - Professor Joseph Tainter, 2023. \url{https://www.youtube.com/watch?v=X3LP5IMWe84}.
\item The Editors of Encyclopaedia Britannica. ``Majorian,'' March 25, 2024. \url{https://www.britannica.com/biography/Majorian}.
\item United Nations. ``Causes and Effects of Climate Change.'' United Nations. United Nations, 2022. \url{https://www.un.org/en/climatechange/science/causes-effects-climate-change}.
\item ---{}---. ``Climate Change `Biggest Threat Modern Humans Have Ever Faced', World-Renowned Naturalist Tells Security Council, Calls for Greater Global Cooperation,'' 2021. \url{https://press.un.org/en/2021/sc14445.doc.htm}.
\item US EPA, OAR. ``Climate Change Impacts on Air Quality.'' Overviews and Factsheets, October 19, 2022. \url{https://www.epa.gov/climateimpacts/climate-change-impacts-air-quality}.
\item Vermeer, Martin, Stefan Rahmstorf, and William C. Clark. ``Global Sea Level Linked to Global Temperature.'' \textit{Proceedings of the National Academy of Sciences of the United States of America} 106, no.\ 51 (2009): 21527--32.
\item Walter, Richard, Hallie Buckley, Chris Jacomb, and Elizabeth Matisoo-Smith. ``Mass Migration and the Polynesian Settlement of New Zealand.'' \textit{Journal of World Prehistory} 30, no.\ 4 (December 1, 2017): 351--76. \url{https://doi.org/10.1007/s10963-017-9110-y}.
\item Wanamaker, Alan D., Paul G. Butler, James D. Scourse, Jan Heinemeier, Jón Eiríksson, Karen Luise Knudsen, and Christopher A. Richardson. ``Surface Changes in the North Atlantic Meridional Overturning Circulation during the Last Millennium.'' \textit{Nature Communications} 3 (June 12, 2012): 899. \url{https://doi.org/10.1038/ncomms1901}.
\item Weisweiler, John. ``Inequality.'' In \textit{Oxford Classical Dictionary}, 2022. \url{https://doi.org/10.1093/acrefore/9780199381135.013.8257}.
\item Wilmshurst, Janet. ``Pre-European Deforestation.'' Web page. Ministry for Culture and Heritage Te Manatu Taonga, 2007. \url{https://teara.govt.nz/en/human-effects-on-the-environment/page-2}.
\item Wu, Min, Jean-Claude Walser, Lei Sun, and Mathias Kölliker. ``The Genetic Mechanism of Selfishness and Altruism in Parent-Offspring Coadaptation.'' \textit{Science Advances} 6, no.\ 1 (January 3, 2020): eaaw0070. \url{https://doi.org/10.1126/sciadv.aaw0070}.
\item Yurko, F. J. ``End of the Late Bronze Age and Other Crisis Periods: A Volcanic Cause?'' In \textit{Gold of Praise: Studies on Ancient Egypt in Honor of Edward F. Wente}, edited by Edward Frank Wente, Emily Teeter, and John A. Larson, 455--63. Studies in Ancient Oriental Civilization, no.\ 58. Chicago, Ill: Oriental Institute of the University of Chicago, 1999.
\end{enumerate}

\end{document}
