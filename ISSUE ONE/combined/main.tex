\documentclass[12pt, letter, openany]{book}
\usepackage{format}

\newcommand{\jissue}{1}
\newcommand{\jyear}{2025}


\begin{document}

\frontmatter

\vspace*{\fill}
\begin{center}
Copyright © 2025 by the authors.

All articles published in Academus are open access under the Creative Commons Attribution-NonCommercial 4.0 International License (CC BY-NC 4.0).

Authors retain copyright and grant the journal a non-exclusive license to publish and distribute their work.
\end{center}

\vspace*{\fill}

\chapter{Editor's Note}

Welcome to the inaugural issue of Academus, a student-run journal for the humanities and social sciences. We started Academus because we want to engage today’s youth with big questions— questions about knowledge and power, justice and community, technology and tradition. We noticed a gap: these brilliant minds were without a clear place to publish, debate, and build on one another’s ideas. This issue is our first attempt to fill that gap. We want to create a home for serious, accessible scholarship by high-school authors who aren’t waiting to be asked to join the conversation when they get to college. 

What follows is intentionally broad. The papers range across philosophy, political theory, history, economics, and public policy, and they share two traits we prize: argumentative clarity and concern for real-world stakes. We’re grateful to the authors and peer reviewers who trusted a brand-new journal with their work. We hope the pieces spark more inquiry, more disagreement, and most importantly, more writing. 

We open with Jacob Gong’s “A Binary Theory of Collapse,” which models civilizational failure as the interaction of anthropological and environmental forces, then presses the unsettling claim that contemporary climate dynamics shift that model from diagnostic to predictive. The essay’s synthesis, from Tainter to modern climate literature, invites readers to test theory against history and present policy. 

Joseph Huang’s “Evaluating Governmental Control on Social Media Speech” enters today’s most fraught public square. Weighing feasibility, abuse risks, and economic effects, the paper argues that state regulation is a poor instrument against online falsehoods compared with third-party fact-checking and scalable media literacy. In essence, he asks: what works without eroding liberal commitments? 

In “A Critique of the College Board Company,” Reed Chan traces the growth of a nonprofit into a de facto gatekeeper for admissions and curriculum, pairing historical analogy with contemporary antitrust concerns. Whether you agree with every charge, the piece raises a timely question about public goods administered by private actors. 

Justin Lee’s “A Fertility Crisis” surveys demographic decline across cultural, economic, and biological drivers, then rejects narrow pronatalism in favor of strengthening the underlying economic conditions that make family formation feasible. It’s an essay that refuses easy answers and keeps the policy horizon wide. 

Michael Gkatzimas’s “A History of Racism and a Case for Affirmative Action” distinguishes interpersonal from systemic racism and evaluates affirmative action under utilitarian and Kantian lenses. The argument lands on a modest but firm conclusion: imperfect tools can still be the fairest response to unfair histories. 

Kevin Nguyen’s “China’s Lost Children, Its Dying Economy, and the Economic Implications” reads the long shadow of the One-Child Policy across labor markets, savings behavior, housing, and global spillovers. The piece exemplifies our aim to connect institutional choices to worldwide consequences. 

In “On Humility,” Yichen Wang stages the Feldman–Kelly peer-disagreement debate and defends doxastic humility: when equally competent thinkers conflict, suspension of judgment is often the rational (and cooperative) move. It’s meta-epistemology with everyday implications for learning together. 

We close with Alexander Lian’s “Balancing Free Speech and Fair Elections: A Historical Analysis of U.S. Campaign Finance,” which moves from Tillman to Buckley to Citizens United and beyond, mapping how rules about money, speech, and parties keep rearranging the democratic game. The history doubles as a civics prompt: what kind of guardrails best protect both voice and vote?

Finally, I want to mention that we’re indebted to the teachers, mentors, and reviewers who made this first issue possible, and to the authors who trusted us with their work. If something here makes you want to argue, write it down, and send it in. May these pages invite careful thinking, charitable disagreement, and scholarship that carries beyond these covers.

\medskip

\emph{Vertias $\cdot$ Ratio $\cdot$ Sapientia.}

\medskip

\noindent Sincerly,

Sam Cao, Editor-in-Chief

\tableofcontents

\mainmatter

\begin{refsection}[refs/binary]
\input{source/binary}
\nocite{*}
\printbibliography[title={References}, heading=subbibliography]
\end{refsection}

\begin{refsection}[refs/evaluating]
\makechapter{Evaluating Governmental Control on Social Media Speech}{Evaluating Governmental Control on Social Media Speech}{Joseph Huang}{Acalanes High School}

In recent decades, fake news—whether spread intentionally or unintentionally—has emerged as a significant challenge for countries around the world. Widespread claims of voter fraud on social media following the 2020 U.S. election undermined trust in the electoral process, resulting in reduced voter turnout in subsequent elections (Sanchez and Middlemass). Similarly, social media disinformation has fueled dangerous and violent behavior during events such as the Rohingya genocide, Southport murders, and COVID-19 (Booth; Ferreira Caceres et al.; Ortutay). Accordingly, many have argued for government regulation aimed at suppressing disinformation. However, given the practical, ethical, and economic risks of such an approach, fake news is better addressed through independent mechanisms such as third-party fact-checkers and media literacy initiatives. 

The foremost consideration when evaluating the efficacy of government regulations is their feasibility, namely, the choice of enforcement mechanisms. In Germany, the 2017 Network Enforcement Act, which relied on content removal via user complaints, prompted platforms to over‐block content, often erring on the side of deletion to avoid potential fines, while producing little reduction in disinformation (Griffin). Utilizing AI to evaluate content is likewise difficult, as AI-based takedowns yield an “enormous number of false positives”; further, algorithmic fact-checking is largely limited to English, lacking sufficient multilingual training data (Marsden et al.). Another concern is that regulation may push platforms to prioritize legal compliance over responding to evolving disinformation threats, especially as misinformation outpaces slow regulatory frameworks (Bateman and Jackson). Ultimately, the technical and structural limitations of government enforcement render it an inflexible and inefficient means of combating the spread of disinformation. 

Another concern relates to classification of misinformation: who gets to decide what “fake news” constitutes? As historical examples illustrate, especially in countries with weaker political institutions, governments have used misinformation laws to silence dissent under the guise of public safety or national unity. In Bangladesh, India, Indonesia, Malaysia, the Philippines, and Thailand, “fake news” regulation extends to government criticism; “malicious intent” clauses within such regulation justify arbitrary arrests, extended pretrial detentions, and excessive fines (Anansaringkarn and Neo; Mahapatra et al.). Singapore’s regulations allow government officials to simply declare the falsity of online information, or even the removal of said information if doing so is “deemed to be in the public interest” (“Singapore”). Often, regulation in democratic countries is used as pretext by other, illiberal states to curb press freedoms (Henley; Mahapatra et al.). Even if well-intentioned, laws enabling government regulation of social media risk centralizing undue power under state control. Thus, empowering governments to define and police fake news carries inherent risks of overreach. 

An often-neglected consideration is the effect of government regulation on the overall economy. Given that regulation increases compliance costs for platforms, the bar of entry into the social media industry rises, reducing competition and consolidating incumbent power (Sperry). The effects of government regulation on the overall economy can be illustrated by the EU’s General Data Protection Regulation (GDPR), a law designed to protect user privacy online by requiring companies obtain explicit consent for data collection and use. The GDPR serves as an effective proxy for government regulation of social media since both impose high compliance costs. As expected, following the launch of GDPR, market concentration in web tracking increased by 17\%, and websites dropped smaller third-party tracking and advertising companies who could not ensure compliance (Prasad). In addition, economic modelling indicates that repealing Section 230—the U.S. law shielding platforms from liability for user content—would eliminate roughly 425,000 American jobs, illustrating how regulation aimed at policing fake news can stifle growth and employment across an entire economy (6 Myths About Section 230). In this broader context, regulation not only undermines innovation and market competition but also imposes widespread economic costs. 

Evidently, government regulation is an inadequate solution for the issue of fake news. Combating such an issue requires an alternative approach: third-party fact-checkers. This is supported by empirical evidence favoring third-party involvement—contrasted with the lack of empirical evidence favoring government regulation. While third party fact checkers consistently produce a 10-14\% decrease in false belief and a roughly 60\% reduction in the resharing of debunked posts, rigorous studies of government regulation are scarce, most likely due to the logistical and ethical challenges of testing national laws through randomized controlled trials (Chuai et al.; Porter and Wood). Although less rigorous than randomized controlled trials and given the aforementioned challenges, public-opinion surveys offer an alternative to gauge the perceived effectiveness of regulation. One such survey conducted in Singapore indicated that only approximately 34\% of respondents reported that the 2019 Protection from Online Falsehoods and Manipulation Act stopped the creation of fake news (Ang and Zhang). This data underscores a major weakness of government regulation: not only is there a lack of rigorous evidence supporting its effectiveness, but even the sparse survey-based evidence that exists suggests it may have little real-world impact. Fact-checkers are only a temporary solution, however; a far more sustainable and preventive approach is media literacy. Although media literacy programs yield a smaller drop in belief in fake news (~0.27 SD), they produce a significant drop in the sharing of fake news (~1.1 SD) (Huang et al.). Furthermore, fact-checking requires ongoing human effort and is less effective when dealing with fake news in non-English languages, leading to higher long-term costs. In contrast, media literacy can be applied more broadly, requires fewer recurring resources, and equips individuals to assess the reliability of new information on their own (Nygren and Efimova). Together, fact-checking and media literacy provide a more effective response to fake news than government regulation. 

In essence, while the harms of fake news on social media are pressing, the challenges of enforcement, the threat of political abuse, and the suppression of competition make government regulation an ill-advised solution. Instead, independent fact-checking and scalable media literacy programs offer empirically supported, rights-preserving alternatives. The true solution lies not in restricting speech, but in equipping individuals with the tools to critically evaluate information, ensuring a more informed and resilient society. 


\nocite{*}
\printbibliography[title=References,
heading=subbibliography]
\end{refsection}

\begin{refsection}[refs/critique]
\makechapter{A Critique of the College Board Company}{A Critique of the College Board Company}{Reed Chan}{Miramonte High School}

At the beginning of the twentieth century, American society saw massively expanding business and corporate consolidation in many economic sectors. Although the corporatization of industries like steel, mining, and agriculture is most popularized in American history, a very similar movement took place within the educational sector. In 1900, a group of northeastern colleges met to create the College Entrance Examination Board—the modern-day College Board—to set standardized requirements for college admissions. Although the original intent for the nonprofit was pure, to reduce the role of social inequalities in college admissions, the Board quickly took a turn towards monopolization. In the same manner that steel and oil tycoons combined resources to standardize production and dispose of competition through vertical and horizontal integration, the newly founded company wanted a standard measure to assess applicants' ability. What started as a collective attempt to corral growing applicant pools quickly turned into an educational monopoly. Throughout its lifespan, the College Board has mirrored Gilded Age corporations in privatizing and monopolizing education to maximize profits, hugely overstepping the boundaries of fair business practices today. 

Just as the Gilded Age trusts legalized and consolidated control in industry, university managers pursued standardization as a shield against unmanageable admissions. As enrollments grew and candidates differed in background and preparation, one objective examination appeared to impose order, meritocracy, and efficiency–the SAT. Beginning as early as 1926, the proposed standardized test promised to eliminate bias from college admissions by testing each student on equal grounds. Such efficiency, however, had unintended effects: secondary schools, always responsive to college entrance demands, adapted curricula to College Board designs. This undermining of the corporation's initial goals not only created a "teach to the test" style of high school teaching but also encouraged further dependency on the College Board by schools across the nation, as universities began to accept the SAT as a standardized test. The rise of the SAT catapulted the College Board corporation into becoming an industry leader, with the corporation’s reported 2019 revenue being over \$1.1 billion (\cite{totalregistration}). But how has this seemingly “nonprofit” organization gained this much revenue? Well, standard SAT and AP test fees can easily add up to over \$100 per test. Since these tests are essentially mandatory for millions of students nationwide, the College Board profits massively. Not only this, but the College Board also profits from illegally selling student data. In 2024, the College Board was caught sharing and selling student data in violation of New York State’s student privacy law (\cite{desantis2024}). Eventually receiving a \$750,000 penalty, the College Board continues to illegally profit off of unaware consumers. In true robber-baron fashion, the CEO of College Board, David Coleman, earned over \$1.6 million–all while exploiting and mercilessly upcharging his customers. By tracing the College Board's ascent, we can discern a broader trend: the privatization and corporatization of public spheres once thought inviolate to market influences. The Board's transformation from modest exam administrator to corporate overlord of educational access reflects the twentieth century's broader Gilded Age movement towards cold, privatized, profit-driven monstrosities that control our futures.

The abominable nature of the corporation has sparked extreme debates over the company’s legitimacy. Authors like Richard P. Phelps contend that the organization's control over standardized testing is a mirror image of the monopolistic behavior of its Gilded Age forebears, employing its market dominance to suppress competition and increase clout (\cite{phelps2018}). With this in mind, the Sherman Antitrust Act–originally crafted to dismantle Gilded Age monopolies–offers a potential legal avenue to challenge the College Board’s dominance. The corporation’s complete control over college admissions testing and curriculum could be viewed as unlawful monopolization, warranting full federal intervention. In a recent editorial in the California Law Review titled “The College Board: A Case for Antitrust Enforcement Under Section 2 of the Sherman Act”, it is noted that the Board's monopoly on admission examinations provokes valid antitrust issues due to the absence of acceptable alternatives, increasing its monopoly power (\cite{kronsburg2025}). Ultimately, the College Board is a perfect representation of Gilded Age corporate abuse of power, only in our own society today. Similar to famed policy-makers like President Taft and heroic muckrakers like Ida Tarbell, we must fight these tyrannical corporations for the betterment of students across the nation.

The best way to combat such abuses, at least for a student like me, is to raise awareness against these injustices. The Fabric of a Nation Textbook effectively covers the relationship between industry, labor, and political reform of the Gilded Age, but does not include how such corporations have continued into the current age, especially in less mainstream industries like education. A necessary change to the APUSH curriculum to better inform students of the current day could be to add specific examples of late 19th-century corporations that still exist today (like College Board, itself), and further explain the limitations and failures of policies made to hinder such corporations. Overall, the College Board's transformation—from a club of elite colleges to a centralized testing behemoth—mirrors the consolidation of economic power during the Gilded Age. Its influence on curricula and college admissions illustrates the trade-off between public good and privatization, and equity and efficiency. As students learn about the legacy of the Gilded Age, we must garner a more complex understanding of both the period that gave rise to it, but also how its themes persist to the current day.


\nocite{*}
\printbibliography[title=References, heading=subbibliography]
\end{refsection}

\begin{refsection}[refs/fertility]
\input{source/fertility}
\nocite{*}
\printbibliography[title=References,heading=subbibliography]
\end{refsection}

\begin{refsection}[refs/racism]
\input{source/racism}
\nocite{*}
\printbibliography[title=References,heading=subbibliography]
\end{refsection}

\begin{refsection}[refs/china]
\makechapter{China's Lost Children}{China's Lost Children, its Dying Economy, and the Economic Implications}{Kevin Nguyen}{California High School}{10.17613/8rcyg-n5q57}

First implemented beginning 1980, the One Child Policy (OCP) was a policy formulated by the Chinese Communist Party (CCP) with the goal of reducing the birth rates in a quickly industrializing China. By reducing population growth, the government hoped to avoid a Malthusian catastrophe and maintain an “optimal” population for China (L.\ Jiang) Throughout the existence of the policy, it saw a general trend in the gradual loosening of enforcement and growing number of exceptions to the policy. Despite this, the policy still managed to cause an enormous impact upon population demographics, preventing the birth of 100 to 400 million children (Q.\ Jiang), and radically skewing the gender ratio (Brittanica), causing an estimate of 3-4\% more males than females in China. Part of the consideration during the inception of the policy was that the OCP would allow for an increase in the ratio of working age adults to children, which would incentivize saving and increase productivity. While this was the case in the short term, in the long term (Feng), the OCP would cause a shrinkage and expansion (Silver \& Huang) of the working age population and the elderly population respectively, augmenting the strain upon welfare programs and taxpayers. These varieties of factors would not only have a profound effect on the Chinese economy, but consequentially, on the international economy. 

There were some benefits that came with the OCP. Mainly, lowering the fertility rate can result in demographic dividends (International Monetary Fund) that spur economic growth, a result of the diminished dependency rate (i.e., the ratio of the working populace to the non-working populace). Savings rates also increased by 7.5\% (Q. Liang), contributing 25\% to the growth of the Chinese economy, though this disputed, with some sources stating that the contribution of the OCP to saving rates is negligible (Song). The OCP was advantageous in other ways as well, increasing output per capita, narrowing the wealth and gap, and decreasing skill premium (Liao). However, there were also downsides presented by the OCP.  Families with more than one child spend more (Keyu), thus contributing to the economy. Often, this spending goes into education for the child, which serves to further the economic development in the future. The rise in dependency rates as a result of a rise in the proportion of the elderly population and legal obligation of China’s youth to support its elder will cause a contraction of the economy (Mansharamani). Unlike other Asian countries facing the same problems of a declining population of the working age, China has one of the highest female labor-participation rates in the world, being 60.5\% in 2023 (World Bank). This means that the potential effects for increasing female participation in labor is limited. According to some estimates, by 2035, 32.7\% of the population is expected to be 60+ (Economist Intelligence), 60 being the retirement age for males in China, females being even lower. As a result of the skewed gender ratios, men in China used the ownership of property as a sign of affluence, causing the formation of a housing crisis as men rush to purchase property and caused an oversaturation of the housing market. Construction has fallen by 60\% relative to pre-COVID levels (Hoyle), seriously impacting the Chinese economy. Further compounding the problem is the lingering cultural impact of the OCP, as the current generation does not want more children (Stanway), exasperating labor problems in the Chinese economy for the future. 

First, the shrinkage of the working age population in China is beginning to cause, its imports and exports will also begin to weaken, impacting the global economy. Just last month, China’s exports and imports declined by 7.5\% and 1.9\% respectively (Soo). Last year, China’s imports dropped by a little less than 9\%. This recent change reflects a growing trend of falling exports and imports, which can grow to have a profound effect on the global economy (Marsh). Companies such as Apple, Volkswagen, and Burberry rely heavily upon the consumer market in China (Marsh), had begun and will continually be impacted by the atrophying Chinese economy. There is evidence that in addition to the OCP, the apparent economic downturn wMarshas a result of the trade war between the United States and China, which has led to dramatically reductions in trade between the two countries (Bown).  

Second, the massive increase of saving rate, rising 20\% from 1983 to 2011 (Coeurdacier), was a result of many factors including low amounts of urbanization, lack of a social safety net, the strong growth rate of the economy, and the OCP (Hung). The result of this is that the money, instead of going back into the economy as money spent by consumers, instead wound up in peoples savings accounts. A high saving rate causes a litany of problems, including increased pressure on interest rates, causing excess investments, and leading to decreases in exchange rates (Song). A more balanced gender ratio may resolve this issue (Wei), and furthermore, reduce tensions between China and its trading partners.  

Third, the aforementioned gender imbalance brought about by the OCP in China has also caused a housing crisis, effecting the global housing market in the process. China is experiencing a decline in the demand for steel, cement, glass, etc. (Nunley), which is detrimental to the export economy of other nations. In addition to this, as a result of property owners’ unwillingness to sell their property at low rates, commercial property deals have sunk to record low levels globally (Real Estate Markets to Be Invigorated as Growing Pool of Buyers and Sellers Look to Make
a Splash.). Chinese investors who had invested aggressively in decades past have mostly resorted to off-loading projects in the face of the housing crisis (Callanan). In China, the fact that real-estate contributes to 22\% of the total GDP, combined with the fact that a drop of ~4.25\% in the Chinese GDP would cause a ~1.5\% drop in the GDP of advanced economies and a ~2.75\% drop in the GDP of emerging economies certainly makes the situation seem grim (Rogoff). 

The effects of the OCP, despite being highly controversial and contentious, will nonetheless remain highly relevant to not only the Chinese, but the global economy. It seems yet uncertain whether the recent policies implemented by the CCP encouraging births will reverse the current trend of the economy, which seems to be beginning to overturn its past record of phenomenal growth. It is also uncertain whether the rest of the world will also be dragged along with China into an economic slump. 
\nocite{*}
\printbibliography[title=References,heading=subbibliography]
\end{refsection}

\begin{refsection}[refs/humility]
\makechapter{On Humilty}{On Humility}{Yichen Wang}{Hong Kong International School}

When there is a case of peer disagreement, where two equally informed and rational individuals arrive at opposing beliefs about \emph{p}, a question arises: what should each person do once learning about the difference in opinion? Suppose A believes that \emph{p} is true, while B believes not-\emph{p}. After sharing evidence, A wonders if they should hold their judgment or reconsider. This puzzle is important in philosophy because it challenges the debate about humility or tenacity. Humility, as explained by Feldman, argues that A should suspend judgment in light of peer disagreement if fulfilling the following conditions: defeat, if an epistemic peer disagrees with \emph{p}; equal weight, if your peer’s opinion is as strong as your own; and independence, if there are no reasons to discount your peer outside of the disagreement itself. In contrast, Kelly argues in defense of tenacity, which explains that A can reasonably maintain her belief by right. Richard Feldman proposes humility based on the premise that rationality requires consideration of epistemic peers wholeheartedly, acknowledging that we may be wrong. Thomas Kelly argues against this and states that if there is disagreement between epistemic peers, it does not mean that the opposing view or evidence has not been considered. Based on these two arguments, one is inclined to support Feldman. A rational person should suspend judgment in the face of peer epistemic disagreement because of symmetry in reasoning capacity. This paper will defend Feldman’s position by evaluating theoretical scenarios under either lens and discussing two of Kelly’s main objections to show how problematic assumptions should not interfere with the nature of evidence, belief, and rational disagreement. 

The advantages of Feldman’s humility over Kelly’s tenacity are illustrated when the two philosophies are used to analyze theoretical cases. An example of such is the \emph{Dean on the Quad Case}, where you and I, epistemic peers, disagree on seeing a dean on the quad, given the same evidence. In such a scenario, where you believe the dean is on the quad and I don’t, Feldman’s philosophy would have you suspend your judgment on the situation. Since you disagree with me and have been established to be as observant, honest, and capable as I am, the most rational course of action would be to withdraw any conclusions and wait for more evidence. On the other hand, under Kelly’s philosophy, you would maintain your beliefs. Your evidence suggests that the dean is on the quad, but I disagree. Since my opinion does not constitute evidence for the dean’s absence, the rational conclusion is to continue believing this until further proof is provided. By suspending judgment based on Feldman’s philosophy, further evidence must be viewed to conclude. Maintaining a difference of opinion in this case, under Kelly’s philosophy, is not helpful because it does not lead to a conclusion, only a difference of opinion. 

Another theoretical example is the \emph{Restaurant Check Case}, where, in the process of splitting a bill’s cost, you conclude that each person’s share is \$45, while your friend believes that it is \$43. Similar to the last case, both parties are epistemic peers with equal math abilities and evidence. In this situation, Feldman would once again have you suspend judgment, reasoning that your friend’s math abilities are identical to yours, which warrants less confidence in your conclusion. Therefore, Feldman would once again withhold conclusions before confirming either answer with a calculator. Conversely, Kelly would maintain that since you have no evidence to believe that your total is wrong, the correct action would be to pay the \$45. Suspending judgment in this case would be best pending further review of evidence, which supports Feldman’s theory of humility. 

Thomas Kelly presents a challenge to the humility view, contending that one should not suspend judgment when faced with peer disagreement. Kelly argues that “when people give reasons for their beliefs, they do not typically say things like 3. Normally, the fact that someone believes \emph{p} is a result of the evidence for \emph{p}, not the evidence itself for p.” Since we are epistemic peers, with equal intelligence, well informed, and capable of rational judgment, it is possible to form beliefs based on a body of evidence. If we were to counter someone’s belief as evidence, it is considered double-counting. For Kelly, treating B’s belief as evidence would essentially be the same as counting the underlying evidence. Both A and B have evaluated the evidence and come to opposing conclusions. If A were to decide that B’s evidence is true, then the same evidence is counted twice, since B already concluded the fact in the evidence on their own.

In addition to double-counting evidence, Thomas Kelly objects to the humility approach in an epistemic disagreement because of the asymmetry objection. Kelly contends, “suppose that, as it turns out, you and I disagree. From my perspective, of course, this means that you have misjudged the probative force of the evidence. The question then is this: why shouldn’t I take this difference between us as a relevant difference, one which effectively breaks the otherwise perfect symmetry? … From my vantage point–as one of the parties within the dispute, as opposed to some on-looking third party– it is just this undeniably relevant difference that divides us on this particular occasion” (15). Based on this premise, it seems as if A and B need to have their evaluation of the evidence. 

Given their evaluation of evidence, suspending judgment is not the appropriate course of action here. Given that A and B are true epistemic peers, neither should have reason to doubt the assessment that led to their conclusion, even if there is disagreement. Humility does not work in this context because it suggests that A and B should be doubtful of their evidence when they should instead trust their judgment. However, appealing to one’s own perspective does not break the symmetry in any meaningful way. If both are true epistemic peers, simply believing the other is mistaken adds nothing beyond restating the disagreement. From a neutral standpoint, the disagreement remains balanced, and the most rational response is to suspend judgment until further evidence is available. Such an approach implies circular reasoning, which is derived from A and B both believing that they are right because of their evidence. This does not work in this context because, contrary to what Kelly argues, Feldman supports the idea that it is okay to suspend judgment pending evaluation of evidence so that either A or B can be right.

Overall, while the two philosophies have their strengths and weaknesses, Feldman’s humility ultimately proves better than Kelly’s tenacity. Firstly, the practice of humility is typically the safer option, as seen in the scenarios. Feldman preaches not jumping to conclusions, opting to suspend judgment until further evidence is given. This is especially true for the Restaurant Check Case, where waiting for a confirmed total makes the most sense to avoid overpaying. Furthermore, the idea of humility is formed around inherent respect for epistemic peers. This not only protects the opinions of others but also fosters practicality in collaboration. Instead of picking a side, humility instead respects the judgment of others and opts for a shared search for the truth, disregarding personal opinions. 

When faced with the dilemma of disagreement between epistemic peers, the most rational course of action is to suspend judgment following Feldman’s view of humility. Although Thomas Kelly raises objections against Feldman, such as the risk of double-counting and asymmetry in evaluation, his argument falls short due to circular reasoning and shortsightedness. On the contrary, Feldman’s approach fosters open-mindedness and a genuine search for the truth over simply prevailing in an argument. His philosophy leads to more accurate conclusions in situations portrayed in the \emph{Dean in the Quad} and the \emph{Restaurant Check Cases}, and offers a perspective of mutual respect and collaboration. 



\nocite{*}
\printbibliography[title=References,heading=subbibliography]
\end{refsection}

\begin{refsection}[refs/campaign]
\input{source/campaign}
\nocite{*}
\printbibliography[title=References,heading=subbibliography]
\end{refsection}

\end{document}
