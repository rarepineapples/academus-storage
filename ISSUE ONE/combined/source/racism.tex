\makechapter{A History of Racism and a Case for Affirmative Action}{A History of Racism and a Case for Affirmative Action}{Michael Gkatzimas}{Miramonte High School}

Racism is an intrinsic flaw in our society that echoes across our institutions, economic and class hierarchy, and current global political formation. Racism continues to be an issue today due to its deep-rooted importance in the creation of our society and the status quo. In order to begin fixing such a fundamental piece of our society, we must first define racism itself, examine its origins in early European colonialism, its eventual systematic oppression of minority groups, and finally, discuss the benefits and flaws of affirmative action - a proposed solution to the complex topic problem of racism. Although not perfect, affirmative action, grounded in the ethical theories of utilitarianism and Kantianism, is an effective and morally correct approach to addressing the historical and ongoing impacts of interpersonal and systemic racism.

To understand the topic of racism, we must first determine what race is. Contrary to popular belief, race was originally used as a method of organization based on class rather than physical features. The modern definition of race, however, is a societal construct originally designed in the 17th century in Western Europe, created to classify people into “natural groupings” based on physical features and observed behavior. This concept was created by Swedish botanist Carolus Linnaeus and German physiologist Johann Friedrich Blumenbach, who applied the concept of the “natural” hierarchy of various flora and fauna to groupings of people (\cite{smedley2024}). Although not initially malevolent, this divisionary concept of race quickly gained attention, as many Enlightenment thinkers and writers in Western Europe began to support this idea. Many assertions about the inferiority of specific races (typically targeting Africans) were published. Notable philosophers and thinkers such as Voltaire, David Hume, Thomas Jefferson, and Immanuel Kant all expressed negative opinions about the “primitive” nature of Africans. This trend of superiority saw its peak with the rise of Social Darwinism, a misnomer that twisted Charles Darwin’s biological theory of evolution (survival of the fittest) and applied it to race. This theory was created by Spencer and Walter Bagehot, who used it as a “scientific” way to further categorize races. This form of racism, now defined as discrimination or prejudice based on racial or ethnic groups played a major role in shaping the formation of our society today. Although many motivations existed for the rise of colonial movements in Europe, including the desire for economic gain, political advantage, and the spread of religion, one of the most intrinsic motivations that allowed colonists to commit such acts of extreme violence and exploitation was the self-righteous belief that the white man was key to civilizing a society. Spurred on by ideas of European superiority and perfection, colonial settlements across Africa, the Americas, and East Asia saw extreme exploitation of the native population. Social disruption, proselytism, mass killings, and forced labor were only a few of the appalling ramifications of early imperialism that were justified by the idea that European colonists were civilizing the savage and barbaric non-European population and bringing them the light of religion and “proper society” (\cite{kipling1899}).

Now that we have generally established the definition of racism and its origin, we must break down the specific subcategories of racism, their prominence in history, and their permanence today. The complex issue of racism can be broken down into two separate subcategories: interpersonal racism and systemic racism. Interpersonal racism is the most general form of racism, where conscious or subconscious bias plays a role in influencing interactions or perceptions of other people based on race. This form of racism is most often expressed in stereotypes and racial slurs, as well as in more violent forms such as hate crimes. Typically influenced by social norms, a historical example of interpersonal racism is the brutal conditions under which Africans were kidnapped and transported to the Americas in the system known as the Trans-Atlantic Slave Trade. In this system, goods and cash crops obtained through slavery in America were traded for African slaves captured across coastal regions of Western Africa. Infamously, the Middle Passage (transportation of slaves from Africa to the Americas) was known for its extremely overcrowded and destitute living conditions. Facing disease, malnutrition, and physical and mental abuse, around 12.5 million Africans, primarily males, were enslaved and taken away from their homes to work on plantations until their deaths. Interpersonal racism played a large role in the slave trade due to evolutionary beliefs in Social Darwinism, Africans were viewed as closer to creatures than humans. This dehumanization of a targeted race has been a common theme throughout history, with such forms of racism still relevant to us today. 

Interpersonal racism in the form of hate crimes has seen a massive increase in recent years, with almost 12,000 hate crimes being committed in 2022, compared to 7,200 in 2018. More specifically, violence against Asian Americans has seen a marked rise following the COVID-19 pandemic. The virus’ Chinese origin caused many people to associate Asian Americans with the virus, leading to the increased use of racist remarks, stereotyping, and violence. From the start of the pandemic in 2019 to 2020, the FBI reported a 77\% increase in hate crimes against first- and second-generation Asian Americans (\cite{findling2022}). Hate crimes make up only a small percentage of interpersonal racism. The most common forms are verbal abuse and discrimination. A study done by the American Journal of Public Health found that calling COVID-19 the “China Virus,” a label given by former President Donald Trump, was heavily associated with more frequent instances of bias and hate speech against Asian Americans (\cite{hswen2021}). This further illustrates how interpersonal racism follows societal trends created or endorsed by influential figures and continues to target and attack specific racial minorities. 

Now, let's discuss the topic of systemic racism, the form of racism that has been most influential throughout history and today. Systemic racism is defined as policies and practices that exist throughout society that result in discrimination or biased, harmful treatment of others based on race. Examples of systemic racism throughout US history include the Chinese Exclusion Act of 1882, which was the first legislation to ban immigration targeting a specific racial group. This act was preceded by extremely high racial tensions, where the increasing Chinese immigrant population was blamed for unemployment and low wages. This blaming of Chinese immigrants, known as scapegoating, is a common motivation for systemically racist policies. Similarly, during World War II, the American public feared that Japanese American citizens were spies for the Japanese government. They were forced into government facilities known as internment camps. Signed off by President Franklin D. Roosevelt, armed military personnel forcefully removed Japanese Americans from their homes and transported them into internment camps across the country. Lasting for over three years, these camps had terrible living conditions, with almost 2,000 Japanese dying in these camps due to a lack of medical care. After the war ended, thousands of Japanese were displaced, without homes and jobs. Japanese internment remains one of the worst instances of institutionalized racism in America’s history; this illustrates the deep-rooted racism that is built into this country. One of the biggest institutions of systemic racism in American history by far, however, was the use of Jim Crow laws in the South. Enacted in the late 19th century, Jim Crow laws racially segregated almost all aspects of life (schools, restaurants, public places, transportation, etc.). These policies were created to continually oppress African Americans and to perpetuate the status quo of white supremacy following the abolition of slavery. During this time, blacks faced policies such as “redlining” - policies that isolated black communities into poverty and instituted a cycle of crime, over-policing, and poverty that we still see today. Racial profiling and police brutality reinforce this system of oppression. Because of the stereotype of blacks being criminals (perpetuated by low wages leading to increasing crime and policing in black neighborhoods), around 41\% of African Americans report being stopped by a police officer solely because of their race (\cite{stewart2024}). Not only this, but blacks face a disproportionately high rate of police brutality, with 25\% of all people killed by law enforcement being black males (black males comprise only 6\% of the population). Institutionally racist policies of redlining, the biased criminal justice system, racial profiling, and police brutality all contribute to a cycle of racism, poverty, and a lack of effective change that only perpetuates the corrupt status quo of today.

While no solution will perfectly end all forms of racism, affirmative action remains the best option for combating and ending the deep-rooted cycle of racism that our country was built on. Affirmative action is defined as a set of policies that seek to benefit marginalized groups, working to bridge inequalities in pay, education, and employment. Forms of affirmative action can be categorized into three categories: outreach (offering marginalized groups opportunities), remedial (offering compensation to historically disadvantaged groups), and diversifying legislation. Examples of such policies include introducing quota systems, meaning organizations are required to give a certain number of positions to disadvantaged groups. Initially created in an attempt to end discrimination in the 1960s, quota systems attempt to end inequities in employment, education admissions, roles in politics, and others by diversifying the workforce and ensuring that all people have an equal chance at success regardless of the opportunity given from birth. Many quota systems have seen success in employment - a study conducted by the United Nations University in Europe found that 63\% of the 194 studies reviewed concluded that affirmative action improved “outcomes for ethnic, religious, or racial minorities“ (Gisselquist). Affirmative action has not seen universal approval; many universities in the US have recently pushed back on diversity quotas in admissions, arguing that affirmative action removes the core meritocratic nature of admissions. On June 29, 2023, the Supreme Court ruled that using race to determine college admissions is ultimately unconstitutional, sentiments supported by the banning of affirmative action in public colleges across nine different states. The amount of discourse around affirmative action brings the policy’s morality into question. In order to analyze such a contentious topic, let’s analyze affirmative action through the lens of utilitarianism and Kantism. 

The principle of utilitarianism deems that an action is considered morally correct if it promotes the greatest happiness for the greatest number. When analyzing affirmative action under Act Utilitarianism (a branch of utilitarianism prioritizing short-term happiness for a specific act), policies such as quota systems may seem immoral, as such policies may have the opposite effect of increasing racial tensions between minority groups, possibly leading to “reverse discrimination” (often called anti-white racism) and increasing hate crimes and acts of racism, rather than increasing success and equality. Under the lens of Rule Utilitarianism (a branch of utilitarianism that conforms to determined laws or rules, typically favoring long-term gain), however, affirmative action is much more appealing, as quota policies ultimately increase diversity in the workplace, leading to more overall success and happiness across racial groups, as well as mending the damage of oppressive systematic racism.

In Kant’s deontology, an action is defined as morally correct if it can be universalized without contradiction, respects other humans as an end (all humans autonomously reason) rather than a means, and is purely driven by duty rather than consequence. Affirmative action can be applied universally without contradiction, as the primary goal of such policies is to introduce equity and balance the inequalities of birth-given opportunities (something that will always be true regardless of universality). Those who argue against affirmative action often cite the rise of tokenism and ineffective policy as negative consequences of affirmative action. Tokenism, or the practice of meeting diversity quota requirements solely to give the appearance of diversity, is treating other humans as means rather than an end (treating someone as a diversity check box rather than a human being). Nonetheless, affirmative action is morally correct as the basis of such policies appeals to the duty to uphold the basic principle of equality (that differential treatment is justified in morally different situations) rather than achieving peak efficiency in the workforce or other consequentialist concerns.

Overall, affirmative action is a flawed yet effective way of ending the cycles of racism and poverty that have been perpetuated throughout our country's history of systematic and interpersonal racism. While issues such as tokenism and ineffective policies exist, the goal of affirmative action is to lead society toward a healthier and more equitable future. Meritocratic systems must be rewritten to account for differing opportunities, something that is only solved through affirmative action policies such as reimbursement of disenfranchised communities and diversity quotas. “Reverse discrimination” and rising tension between racial groups may be a consequence of diversity, but it is a necessary part of transforming our broken society into one of true equality and opportunity. While there will always be those disadvantaged by affirmative action policies, those experiencing the benefits (marginalized communities) deserve their own equal opportunity after being subjected to decades of systematic oppression. 


