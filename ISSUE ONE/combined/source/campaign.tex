\makechapter{Balancing Free Speech and Fair Elections}{Balancing Free Speech and Fair Elections:\\ A Historical Analysis of U.S. Campaign Finance}{Alexander Lian}{Miramonte High School}


\section{Introduction}

Along with the expansion of political rights comes the escalating number in campaign spending. Large corporations and small individual donors seem to have a growing interest in voicing their political opinions through donating to candidates of their choice. Although campaign finance is protected by the First Amendment as free speech, its increasing presence in American politics has posed challenges to a true democracy by exposing government agendas to corruption and exacerbating polarization. 

Campaign finance regulations have long been the center of debates in the American government because they require a careful balance between protecting freedom of speech and limiting corporate power. To understand the impact of campaign finance, one must examine the U.S. government’s position and evaluate it throughout history. 

\section{History}

Seeing the dusk of the Gilded Age characterized by rampaging corporate capitalism, the Tillman Act of 1907 set tight boundaries and high standards: corporations were banned entirely from contributing money to influence any federal elections (Bitzer). Proponents of the Act saw it apt to suppress corporate corruption in U.S. politics. Corporate giants at the time often donated heavily to elect officials in exchange for loose regulation or help against the Unions: “Wealthy industrialists funded political campaigns, ensuring that candidates who supported pro-business policies were elected. These contributions were often accompanied by promises of future favors or business opportunities, creating a quid pro quo arrangement that undermined democratic principles” (Youvan). Rutherford B. Hayes, the 19th president of the United States, confirmed the plight of American democracy in his diary: “It is a government by the corporations, of the corporations and for the corporations” (Klein). However, along with emerging progressive leaders such as Theodore Roosevelt and Woodrow Wilson reinvigorating fair elections and expanding government regulation over the private sector, corporate corruption through campaign contributions waned away into the stage curtains of history. With the memories of the age of entrenching corporate corruption diminishing in the public’s eyes, the debate of whether or not to restrict campaign finance was brought up again in the 1976 definitive Supreme Court case \emph{Buckley v. Valeo.} One year after establishing the Federal Election Commission (FEC), \emph{Buckley v. Valeo} lifted the cap for candidate expenditure in an election. The court justified its ruling by declaring that political expenditure is an act of free speech protected under the First Amendment, while donations to candidates remained restricted. Although candidates can legally make unlimited expenditures such as hanging up flyers or investing in commercials post \emph{Buckley v. Valeo}, their sources of campaign spending remain highly regulated. Nevertheless, allowing candidates to spend infinite amounts of money on their campaigns left the candidates with an unquenchable desire to attract corporate interests. Corporations found ways to circumvent campaign finance restrictions to satisfy those eager for money by donating “soft money” to support candidates. “Soft money” spending is independent expenditure promoting a particular party without directly addressing the candidate. Independent expenditure is any spending that does not go through the hands of the candidates directly and thus isn’t subjected to campaign finance regulations. While the concept of “soft money” seemed to be immune from “quid pro quo” corruption, in practice, there was little oversight of precisely what this “soft money” spending promoted. Furthermore, the initiative of corporate “soft money” was inadvertently recognized by the Supreme Court in the 1978 Case First Bank of Boston v. Bellotti: “The inherent worth of the speech in terms of its capacity for informing the public does not depend upon the identity of its source, whether corporation, association, union, or individual” (“CITIZENS UNITED”). The trend of “soft money” spending was put to an abrupt stop 12 years later by the Bipartisan Campaign Reform Act (BCRA), more commonly called the \emph{McCain-Feingold Act. McCain-Feingold} addressed two issues: first, closing the legal loophole of “soft money” by empowering the FEC to oversee corporation spending before elections more strictly; second, banning political ads that name a specific candidate 1 or 2 months before federal elections, called “electioneering communications.” The Supreme Court case McConnell v. FEC one year later upheld the two decisions made in \emph{McCain-Feingold}. Since \emph{Buckley v. Valeo}, the debate over campaign finance regulation remained fierce for the entirety of the last quarter of 20th century: on one hand, the rulings of \emph{Buckley v. Valeo} and \emph{First Bank of Boston v. Bellotti} upheld corporate contributions as a form of speech; on the other hand, \emph{McCain-Feingold} and \emph{McConnell v. FEC} ruled on the basis that the “disproportionate power” wielded by corporations exposes the government to corruption (Gerken). 

In 2008, conservative non-profit organization \emph{Citizens United} released a film criticizing then-Senator Hillary Clinton prior to the primary election. Understanding that the film would have been punished by \emph{McCain-Feingold} as it stood as a form of electioneering communication, Citizens United sued the Federal Election Commission to bring the case to court. The Supreme Court eventually came to a 5-4 ruling siding with Citizens United (Lau). \emph{Citizens United v. FEC} started a new chapter in American politics because it revoked some of the most vital clauses under \emph{McConnell v. FEC.} Specifically, soft money and electioneering communications were both reintroduced into federal elections: “The Court noted that [McConnell v. FEC]’s prohibition on corporate independent expenditures and electioneering communications is a ban on speech and ‘political speech must prevail against laws that would suppress it, whether by design or inadvertence’” (“Citizens United”). 

\section{Campaign Finance Post \emph{Citizens United} Ruling}

The ruling in Citizens United gave birth to Super Political Action Committees (Super PACs), which are allowed to spend unlimited amounts of money to support a particular candidate as long as they do not directly interact with them. With unlimited campaign spending allowed under \emph{Buckley v. Valeo} and unlimited independent spending allowed by \emph{Citizens United v. FEC}, the Pandora's box for corruption has already been opened. As Michael Kinsley, an American political journalist, posits: “The scandal [of politics] isn’t what’s illegal; the scandal is what’s legal” (Hitchens). Although perennially debated, the balance between the regulation of corporation action and the freedom of speech was concluded by Citizens United. 

Within the contemporary borderline that defines campaign finance, corporations corrupt government agendas by critically influencing election results. In the 2020 presidential election, the voter turnout rate was 67\% of the eligible voter population, the highest since 1900 when the voting population was restricted by gender and race (Menand). However, to what extent do votes truly embody people’s control over the government and policies? The matter of fact is that although people can choose who emerges from the candidate list in the primary or general election, the public is blind to the facilitation of parties and interest groups that pushed the candidates onto the list behind the scenes. Even during elections, the public’s opinion is implicitly and explicitly guided by propaganda, attack ads, and other political advertisement content. The real power player behind every election is money. 

In December of 2024, Elon Musk, arguably one of the biggest proponents of Trump’s electoral success and the candidate to head the Department of Governmental Efficiency, was angered by a Bipartisan government spending bill. He subsequently used the social media platform X, purchased in 2022, to threaten political retaliation against any Republican legislator supporting the bill by claiming to support their opponents in future elections (Shear et al.). Musk’s threats were not unwarranted. As the Federal Election Commission finds, in the 2024 election cycle, Musk’s donation to the GOP totaled a staggering \$277 million through his own America PAC and RBG PAC (Ingram; Pino et al.). Musk’s campaign finance action gives him leverage over the Republican party and, effectively, the president himself. Within one week, president-elect Trump sided with Musk’s contention by calling the bill backed by his own Republican House Speaker “a betrayal of our country.” Speculations on the left wing arise, calling Musk the “fourth branch of government” (Leingang). Indeed, corporate influence over the American government through shaping the tide of elections has never been a rare scene in this country. 

In 1999, Congress passed the Gramm-Leach-Bliley Act, allowing virtually any Wall Street firm to merge and consolidate. By empowering the largest Wall Street firms to gain monopolizing power, GLBA indirectly contributed to the growing housing bubble that eventually burst in the 2008 financial crisis (Peoples). Wall Street’s “not-so-secret weapon” for the success of GLBA was its enormous soft money donation to federal legislators throughout the 1990s. The non-profit organization “Open Secrets” reveals that soft money contributions from Wall Street’s insurance and securities firms soared from \$18.02 million in 1994 to \$111.4 million in 2000. In recent years, GLBA hasn’t been the only case where Wall Street firms strategically invested in campaigns in exchange for legislative support for loosening government regulations. Michael Corcoran found in 2018 that among the 17 Democrat legislators who voted with Republicans to repeal parts of the 2010 Dodd-Frank Act, which sought to protect consumers from Wall Street’s predatory lending practices, the majority of them were among the top recipients of Wall Street donations in the same election cycle. Corcoran further found that: “Nine Democrats also crossed party lines to appoint Goldman Sach’s bailout attorney Jay Clayton to lead the Securities and Exchange Commission,” who was denounced by the Democratic party for his exploitation of the 2008 financial crisis. 

Although the Supreme Court ruled in 2010 that donation is a form of political speech, corporate donations to candidates are principally valued as a form of investment instead: “The Supreme Court’s McConnell v. FEC decision… noted that more than half of the 50 largest corporate soft-money donors to the political parties in 2000 gave to both parties, a clear sign that access, not ideology, motivated those donations” (Pildes). The gap between the law’s assumption of corporate intention and the de facto motive behind corporate contributions leaves room for opportunists. 

The intention behind candidates accepting corporate donations can be summarized by one word: reelection. Corporations can be seen as fishermen holding baits of enormous campaign donations. At the same time, candidates must decide between biting the bait for better chances of winning or leaving their opponents to take the gift. Since the time of political machines in the 1800s, governmental officials have proven time and time again that they will bite the bait for the sake of reelection. In the 2022 federal elections, Senate members succeeded at a 100\% incumbent rate, while House members saw a 94\% incumbent rate. The reelection rate in the House has never dropped below 85\%, while the reelection rate in the Senate has never dropped below 88\% since the Reagan era (Hall-Jones). These dramatic numbers rely on the significant financial gap between incumbents’ campaigns and their challengers’ campaigns: incumbent Senators running in the 2022 elections averaged \$29.7 million in campaign contributions, compared to a trivial \$2.1 million average raised by their challengers; incumbent Representatives averaged \$2.8 million in money raised, compared to \$308,000 by their challengers (Hall-Jones). When elected officials want to consolidate their seats, continual and reliable support from corporations becomes indispensable. However, when incumbents are encouraged to bite the bait, corporations also ask for benefits in return: “Even after controlling for past contracts and other factors, companies that contributed more money to federal candidates subsequently received more contracts” (Witko).

As Professor Patrick Flavin of Baylor University posits: “There is ample evidence suggesting contributions exert sway behind the scenes by influencing who legislators agree to meet with, what issues they focus on, and how they allocate their scarce time while in office” (Flavin).  Professor Clayton Peoples of the University of Nevada furthers that companies consistently influence policies through their established ties with the legislators by donating to their campaigns. In the 1990s, one lawmaker argued that “[t]here’s no way in hell that [legislation granting China ‘most favored nation’ trading status] would have passed … if all these companies hadn’t come flooding in and making campaign contributions and asking for peoples’ support … There’s absolutely no doubt in my mind that money changed votes” (Makinson). Although there is no doubt that the ties between corporations and politicians open a legal loophole for the people’s voice to be overlooked, the question remains on how these corporate-pushed agendas could tangibly hurt the people. 

Campaign finance fuels extreme ideologies, therefore exacerbating political polarization. Although corporate contributions certainly gained momentum and influence after Citizens United, one cannot neglect the weight of individual contributions made by the public. In the 2020 presidential election, of the \$1 billion received by Joe Biden, roughly 80\% of the donations came from individual contributions, defined as donations to candidates or PACs made by regular citizens ("Campaign finance data."). The rise of individual contributions intuitively balances corporations’ involvement in politics. Ian Vandewalker of the Brennan Center for Justice argues that the emergence of individual donors, particularly small donors, benefits American politics by fostering “more ideological diversity” (Vandewalker). However, he also recognized that “All donors, regardless of how much they give, tend to be more partisan and ideological than the average voter” (Vandewalker). Professor Pilder of the New York University further explored the intricacies between individual donors’ financial action and the magnitude of their ideology: Republican donors are more conservative than non-donors on economic issues, while Democratic donors are more progressive than non-donors on social issues (Pildes). Among individual donors, small donors – donors who donate less than \$200 – are particularly inclined to donate based on impulses of extreme ideals. Professor Pildes furthers that small donors often contribute to political campaigns after being triggered by “viral moments of outrage.” This phenomenon explains why Donald Trump received enormous sums from small donors with his frequent utilization of social media platforms such as X to attract viral attention. Meanwhile, the growing presence of small donor campaign contributions indicates that incumbents are more concerned with appealing to this group of constituents: “In 2016, small donors contributed \$1 billion to federal elections; in 2020, that rose to more than \$4 billion” (Barber et al.; Pildes). As Culberson found through examining small donor giving in the 2006 through 2010 U.S. House elections: “ideologically extreme incumbents tend to raise more money from small donors.” Culberson’s finding suggests that if legislators wish to gain small donor contributions for their next election, they will position themselves at ideological extremes through both legislative and vocal actions. Meanwhile, Professor Keena of the Virginia Commonwealth University and his colleagues suggest otherwise in their 2019 study: “The amount of money raised from small donors during the previous election cycle has a negligible impact on a senator’s positioning” (Keena et al.). While these contradictory findings put the investigation of politicians’ legislative advocacies in a gridlock, what remains certain is that politicians turn to more extreme vocal actions. 

From inflammatory commentaries to attack ads, campaign finance propagates radical ideologies in American society. Citizens United’s victory in 2010 gave birth to a key player in nurturing polarization: dark money. As the Dean of Yale Law School, Heather Gerken, explains, “dark money” is independent spending by organizations such as Super PACs that cannot be traced to its origin. The issue of dark money, as Gerken explains, is rooted in the unscrupulous “SuperPACs and 501(c) organizations” that wield them to rise as “shadow parties” running the country. Marina Pino of the Brennan Center for Justice further quantifies: “While final numbers are not yet available, in 2024 anonymous sources directed more than \$1 billion, at a minimum, to independent political committees supporting candidates on both sides of the aisle” (Pino et al.). One of the primary goals of dark money is to fund anonymous attack ads. In the 2008 Montana state election, 63-year-old incumbent John Ward experienced such attacks first-hand (Taddonio). Just before election day, a 501(c) 4 issue group named Western Tradition Partnership flooded his district with vicious attack ads denouncing him for his stance on an electricity rate bill. The advertisements played a crucial role in defeating Ward in that election. The issue with Western Tradition Partnership’s action wasn’t that it raised genuine public concern about social problems but because it published and amplified such concerns to push their political agendas. Karolin Loendorf, a former county commissioner who worked for and investigated Western Tradition Partnership, explains that the “social welfare” group had a “candidate hit list” of legislators who don't support their favorable policy. By wielding tremendous financial power gained from dark money, the group could use attack ads to replace those candidates with others who comply with their needs. As Professor Ansolabehere of Harvard University finds, negative advertisement leads to lower voter turnout. He estimates that over 6 million voters quit voting in the 1992 presidential election due to various attack ads. But when relatively moderate constituents leave elections, the radicals’ voices in elections gain strength: “We are losing some of our best citizens, and pandering to the extremists who remain” (Ansolabehere et al.). Attack ads create a world where candidates and interest groups respond primarily to the radical voter base that they “hand-picked” through inciting extremism and fear to amass more campaign contributions that help them consolidate their control. Yet the consequences of political polarization are clear. In government, although laws passed during polarized periods have a more long-lasting impact, legislative efficiency is dramatically slowed due to congressional gridlock (Tutella); in society, the virtual political camps people are divided into based on their political ideologies push political violence and lead to democratic decline (McCoy et al.). 

19th-century American economist Henry George wrote: “Equality of political rights will not compensate for the denial of the equal right to the bounty of nature.” If the American people's political liberty cannot translate to policies that better their lives through the existing system of democracy, then such democracy has failed to live up to its purpose. Although the First Amendment protects campaign finance, its presence in U.S. politics poses corruption and polarization that bar political liberty from attaining its purpose. 

