\documentclass[12pt, a4paper, twoside]{article}
\usepackage{format}
\usepackage{tikz}
\usepackage{pgfplots}
\pgfplotsset{compat=1.18} 
% Do not alter above

% Metadata: put your article information here 
\newcommand{\jtitle}{A Critique of the College Board Company}
\newcommand{\jauthor}{Reed Chan}
\newcommand{\jaffiliation}{Miramonte High School}

% Editors will change these fields after acceptance 
\newcommand{\jvolume}{1}
\newcommand{\jyear}{2025}
% \newcommand{\jdoi}{N/A}  

% References should be placed in refs.bib and cited with \autocite{<source>}
% Quotations can be placed in quote environments: \begin{quote}<your quote>\end{quote}
% Footnotes can be added with \footnote{<your footnote>}

% Your Content

\begin{document}

\maketitle{}

At the beginning of the twentieth century, American society saw massively expanding business and corporate consolidation in many economic sectors. Although the corporatization of industries like steel, mining, and agriculture is most popularized in American history, a very similar movement took place within the educational sector. In 1900, a group of northeastern colleges met to create the College Entrance Examination Board—the modern-day College Board—to set standardized requirements for college admissions. Although the original intent for the nonprofit was pure, to reduce the role of social inequalities in college admissions, the Board quickly took a turn towards monopolization. In the same manner that steel and oil tycoons combined resources to standardize production and dispose of competition through vertical and horizontal integration, the newly founded company wanted a standard measure to assess applicants' ability. What started as a collective attempt to corral growing applicant pools quickly turned into an educational monopoly. Throughout its lifespan, the College Board has mirrored Gilded Age corporations in privatizing and monopolizing education to maximize profits, hugely overstepping the boundaries of fair business practices today. 

Just as the Gilded Age trusts legalized and consolidated control in industry, university managers pursued standardization as a shield against unmanageable admissions. As enrollments grew and candidates differed in background and preparation, one objective examination appeared to impose order, meritocracy, and efficiency–the SAT. Beginning as early as 1926, the proposed standardized test promised to eliminate bias from college admissions by testing each student on equal grounds. Such efficiency, however, had unintended effects: secondary schools, always responsive to college entrance demands, adapted curricula to College Board designs. This undermining of the corporation's initial goals not only created a "teach to the test" style of high school teaching but also encouraged further dependency on the College Board by schools across the nation, as universities began to accept the SAT as a standardized test. The rise of the SAT catapulted the College Board corporation into becoming an industry leader, with the corporation’s reported 2019 revenue being over \$1.1 billion (\cite{totalregistration}). But how has this seemingly “nonprofit” organization gained this much revenue? Well, standard SAT and AP test fees can easily add up to over \$100 per test. Since these tests are essentially mandatory for millions of students nationwide, the College Board profits massively. Not only this, but the College Board also profits from illegally selling student data. In 2024, the College Board was caught sharing and selling student data in violation of New York State’s student privacy law (\cite{desantis2024}). Eventually receiving a \$750,000 penalty, the College Board continues to illegally profit off of unaware consumers. In true robber-baron fashion, the CEO of College Board, David Coleman, earned over \$1.6 million–all while exploiting and mercilessly upcharging his customers. By tracing the College Board's ascent, we can discern a broader trend: the privatization and corporatization of public spheres once thought inviolate to market influences. The Board's transformation from modest exam administrator to corporate overlord of educational access reflects the twentieth century's broader Gilded Age movement towards cold, privatized, profit-driven monstrosities that control our futures.

The abominable nature of the corporation has sparked extreme debates over the company’s legitimacy. Authors like Richard P. Phelps contend that the organization's control over standardized testing is a mirror image of the monopolistic behavior of its Gilded Age forebears, employing its market dominance to suppress competition and increase clout (\cite{phelps2018}). With this in mind, the Sherman Antitrust Act–originally crafted to dismantle Gilded Age monopolies–offers a potential legal avenue to challenge the College Board’s dominance. The corporation’s complete control over college admissions testing and curriculum could be viewed as unlawful monopolization, warranting full federal intervention. In a recent editorial in the California Law Review titled “The College Board: A Case for Antitrust Enforcement Under Section 2 of the Sherman Act”, it is noted that the Board's monopoly on admission examinations provokes valid antitrust issues due to the absence of acceptable alternatives, increasing its monopoly power (\cite{kronsburg2025}). Ultimately, the College Board is a perfect representation of Gilded Age corporate abuse of power, only in our own society today. Similar to famed policy-makers like President Taft and heroic muckrakers like Ida Tarbell, we must fight these tyrannical corporations for the betterment of students across the nation.

The best way to combat such abuses, at least for a student like me, is to raise awareness against these injustices. The Fabric of a Nation Textbook effectively covers the relationship between industry, labor, and political reform of the Gilded Age, but does not include how such corporations have continued into the current age, especially in less mainstream industries like education. A necessary change to the APUSH curriculum to better inform students of the current day could be to add specific examples of late 19th-century corporations that still exist today (like College Board, itself), and further explain the limitations and failures of policies made to hinder such corporations. Overall, the College Board's transformation—from a club of elite colleges to a centralized testing behemoth—mirrors the consolidation of economic power during the Gilded Age. Its influence on curricula and college admissions illustrates the trade-off between public good and privatization, and equity and efficiency. As students learn about the legacy of the Gilded Age, we must garner a more complex understanding of both the period that gave rise to it, but also how its themes persist to the current day.

\printbibliography

\end{document}