%! TEX program = LuaLaTeX 
\documentclass[12pt, a4paper, twoside]{article}
\usepackage{CJKutf8}
\usepackage{format}
% Do not alter above

% Metadata: put your article information here 
\newcommand{\jtitle}{Why One Child?\\How Mao, Scientism, and the West Shaped\\The One-Child Policy}
\newcommand{\jauthor}{Person B}
\newcommand{\jaffiliation}{some school}

% Editors will change these fields after acceptance 
\newcommand{\jvolume}{3}
\newcommand{\jyear}{2025}
\newcommand{\jdoi}{10.17613/82c3s-nac44}  

% References should be placed in refs.bib and cited with \autocite{<source>}
% Quotations can be placed in quote environments: \begin{quote}<your quote>\end{quote}
% Footnotes can be added with \footnote{<your footnote>}

% Your Content

\begin{document}

\maketitle{}

\begin{quote}
The burgeoning population has outstripped economic expansion, overburdening the planet and becoming a crucial issue of social development, national and international political stability.

\hfill Song Jian \autocite[p.\ 11]{SongKongYu1988}
\end{quote}

\section{Introduction}

According to an ancient Chinese legend, the mother goddess, Nüwa, molded the first humans out of yellow clay. A symbol of creation, she nourished the birth of a new world, crafting nascent life with careful precision. Millennia later, the Chinese Communist Party (CCP) likewise assumed a God-like role, seeking to control the nature of humanity. However, instead of creating life, the CCP suppressed and extinguished life; and, in stark contrast to Nüwa's beneficent nature, the CCP was motivated by fear---fear of overpopulation, depleted resources, and the impact of a rapidly rising population on the country's modernization efforts. With the implementation of the one-child policy in 1980, the CCP began an undertaking that they hoped would produce an optimal demographic structure, safeguarding the country's economic destiny and elevating China’s position on the international stage. However, a slew of indelible aftereffects followed this attempt at securing stability and prosperity. The CCP had unknowingly planted the seeds for a monumental population crisis.

By the mid-2000s, China was grappling with significant social and economic challenges stemming from its population policies: a shrinking workforce \autocite[p.\ 356]{White2016PopulationPolicyContext}, skewed male-to-female sex ratio, and rapidly aging population. The decline in the working-age population had resulted in widespread labor shortages, significantly impacting businesses across the country \autocite{SCMP2022WorkerShortages}. Meanwhile, the gender imbalance, with far more men than women (\cites[p.\ 117]{WangEtAl2016AJS}{CameronMengZhang2018}[pp.\ 169--186]{WeiZhangLiu2017}{Hodges2024HousingMarket}), fueled social unrest, contributing to a rise in crime, inflated housing prices, and a marriage crisis where men struggled to find wives \autocite{Fong2016OneChild}.

Simultaneously, China’s rapidly aging population increasingly pressured younger generations, creating long-term financial and social strain \autocite[pp.\ 1429--1430]{JiangSanchezBarricarte2011}. Through decades of coercive extensive abortions, sterilizations, and contraception campaigns, the one-child policy fundamentally altered Chinese demographics, enduringly curtailing people’s aspirations of marriage and childbirth.

However, if the consequences of a one-child policy were so immense, why did the party uphold it for so long? More importantly, why did the one-child policy exist in the first place?

This paper seeks to answer these questions through an investigation of the historical context and multitude of influences that produced the policy. First, the immediate historical circumstances underlying Chinese birth planning in the 1970s and 80s will be described, with a focus on three aspects in particular: the People’s Republic of China (PRC) under Mao’s direction, the fear of a population crisis in the West, and the origins of scientism as an ideological force in China. Following this, a brief history of Chinese family planning is outlined. Family planning will first be traced to its genesis in the works of early philosophers, after which its presence in the 20th century during and beyond Mao’s time in power will be clarified. Furthermore, the radicality of the one-child policy is explored, highlighting how it differs from the “longer (\emph{wan}), later (\emph{xi}), fewer (\emph{shao})” rule (LLF). Finally, this paper concludes with a comprehensive analysis of the causes and justifications of the one-child policy, bringing together the previously discussed topics within a cohesive framework. 

Today, the one-child policy is commonly depicted as the calculated actions of an authoritarian government determined to retain power; in doing so, they stopped at nothing, not even people’s bedrooms, to achieve their malicious goals. However, this is not the full story; more accurately, the one-child policy resulted from a wholesale rejection of the “ideological” following Mao’s calamitous campaigns, an obstinate belief in Scientism, and the pervasive influence of the West both intellectually and financially.

\section{Historical Circumstance}
\subsection{Mao's China}

In the decades following its establishment, the PRC was transformed from a backwater, underdeveloped nation into one of the most prominent political and economic powers of the modern world. However, the PRC’s first chair[pp.\ 1429--1430]man, Mao Zedong, contributed little to such a development---in fact, China was in a constant state of societal tension, chaos, and bloodshed under Mao, largely a result of the mass mobilization campaigns he conducted in order to indefinitely perpetuate the peasant revolution. Two campaigns are particularly relevant to the development of birth-planning policies in China: the Great Leap Forward (\emph{a yuejin}) and the Cultural Revolution (\emph{wenhua da geming}).

By the late 1950s, the collectivization of agriculture, which began almost immediately after the nation’s founding, had been well under way with the Land Reform Movement (\emph{tudi gaige}). To conduct a final collectivization program, Mao launched a comprehensive five-year plan, later known as the Great Leap Forward, leading to increased collectivization and labor-intensive industrialization. Hoping to increase productivity, the state entirely abolished private property, collectivizing it into communes (\emph{gongshe}). Rather than transform China into a communist utopia, however, these reforms produced to the opposite. Factors such as inefficiency \autocite{Britannica2024GreatLeapForward} diminished manpower\footnote{The result of a diversion of labor toward the production of steel and iron without the proper, necessary industrial equipment. \cite{Smil1999GreatFamine}.}, the Four Pests Campaign\footnote{The Four Pests Campaign was a campaign in which Mao order that the “four pests” – that is, rats, flies, mosquitoes, and sparrows, to be exterminated as a form of “pest control”, which would supposedly aid with agricultural output. For the effects of such a policy in relation to the Great Leap Forward and the Great Famine, see Rebecca Kreston, “Paved With Good Intentions: Mao Tse-Tung’s ‘Four Pests’ Disaster,” Discover Magazine, October 15, 2019, \url{https://www.discovermagazine.com/health/paved-with-good-intentions-mao-tse-tungs-four-pests-disaster}.}, and natural disasters together triggered one of the most devastating famines in human history---the Great Chinese Famine (\cites[pp.\ 51--73]{KungLin2003GreatLeapFamine}[pp.\ 125--140]{LinYang1998AgriculturalCrisis}), or the “three-year great famine” (\emph{san nian da jihuang})---resulting in a death toll ranging from 15 million to 55 million (\cites{Remais2023BerkeleyPublicHealth}[pp.\ 13--25]{ChenYixin2015}). During this period, grain production fell precipitously, leading to numerous occurances of cannibalism \autocite{Bernstein1997HiddenChineseFamine}. As one might imagine, this had tremendous consequences for population growth. As Sinologist Susan Greenhalgh notes, the “ugly, lumpy” characteristic of an initial decline followed by a rebound in fertility exceeding pre-famine levels was later used as a justification for family planning policies as many “[appreciated] the importance of smoothing it out” \autocite[pp.\ 113--114]{Greenhalgh2008JustOneChild}.

Less than a decade after the Great Leap Forward, Mao instigated the Cultural Revolution, aiming to rectify the CCP and re-instill a revolutionary spirit in the populace, especially the youth. As described by Historian Youqin Wang, the revolution mobilized students against their own teachers, accusing the latter of being “feudal, capitalist, and revisionist” \autocite[pp.\ 6--32]{WangYouqin2001StudentAttacks}. As such, teachers were beaten, forced to injure each other, or forced to commit suicide \autocite[pp.\ 6--32]{WangYouqin2001StudentAttacks}. As a part of the effort to erase the counter-revolutionary past, traditional Chinese symbols, names, and artwork were destroyed, replaced with new “revolutionary” names and objects \autocite[pp.\ 61--62]{Lu2020RhetoricCulturalRevolution}; in Beijing alone, 475 roads were renamed to include the word “revolution” \autocite[p.\ 168]{Dutton1998}. During this time, the chaos reigning throughout the country forced the government to largely cease its functionalities, leaving the state paralyzed in the face of an anarchic movement that was quickly getting out of hand \autocite{Greenhalgh2008JustOneChild}. Seeing the unbridled turmoil and destruction, Mao finally called for an end to the violence in late 1968, although the Cultural Revolution persisted in a diminished state until his death \autocite{Phillips2016GuardianCulturalRevolution}. It was the unrestrained chaos brought about by such events that helped develop an aversion to the “ideological” in both the minds of the people and those in power, paving the way for the rise of scientism in China.

Outside Mao’s debilitating campaigns, China began to make progress on the international stage in the 1970s, gaining global acceptance and self-awareness after its period of self-imposed isolation, which had begun during the initial stages of the Great Leap Forward \autocite{Dreyer2007ChineseForeignPolicy}. On October 25, 1971, the PRC was admitted as a permanent member of the UN Security Council, replacing the seat’s previous holder, the Republic of China (Taiwan) \autocite{AIT2010UNRes2758}. This acceptance into the international community is especially pertinent to family planning, as gradually, the PRC began importing Western scientific methodology and literature \autocite[pp.\ 74--82]{Minami2024PeoplesDiplomacy}, including highly quantitative methods of demography \autocite[pp.\ 132--134]{Greenhalgh2008JustOneChild}. In addition, China’s integration into the global intellectual environment give China a novel lens to quantify its own scientific backwardness \autocite[p.\ 85]{Greenhalgh2008JustOneChild}; however, not only was Chinese science criticized, but China’s economy and population were also scrutinized and compared to the far more industrialized nations such as France and the United States \autocite[pp.\ 111-113, 118--120]{Greenhalgh2008JustOneChild}. It was these concerns, made possible by a connection to the West, which were the key to the creation of a harsh and coercive family planning policy.

\subsection{Malthusian Crisis}

While China’s economy was experiencing tremendous turbulence under Mao’s political movements and collectivization efforts, Western nations were experiencing what became known as the “Golden Age of Capitalism.” Skyrocketing productivity and an expansion of international trade created a surge of consumer demand, which, when combined with a seamless conversion of wartime industries to consumer industries, led to a so-called “baby boom” (\cites{UN2017PostWarReconstruction}{Moffatt2020ThoughtCo}{Easterlin1987}). Soon, however, the exploding population became a cause for concern, and the problem of overpopulation---a so-called “Malthusian Crisis”---began to loom large in the eyes of the people. This mania was only compounded further by highly dramatized, pseudo-scientific works in the late 1960s and 70s, filled with colorful rhetoric describing the human race’s inevitable doom at the hands of overpopulation \autocite[p.\ 73]{DesrochersHoffbauer2009}.

It is important to note that concerns regarding overpopulation did not originate in the 1960s and 70s; rather, during the early 19th century, British economist Thomas Malthus was the first to discuss the potentially disastrous consequences of unchecked population growth. In his \emph{Essay on the Principle of Population} (1798), Malthus proposed that while population growth followed a geometric trend, growth in food production followed a merely arithmetic trend \autocite{Malthus1999}. Consequently, as population growth outpaced food production, societal order breaks down and chaos ensues. This imbalance between population and available resources would be the fundamental axiom upon which future Malthusian authors would produce their studies.

One of the most prominent works influenced by Malthus’s principle was \emph{The Population Bomb}, published in 1968 by Paul and Anne Ehrlich. The book became hugely successful owning to its apocalyptic rhetoric regarding the imminence of an environmental catastrophe, its authors advancing the claim that there are “[t]oo many people, packed into too-tight spaces, taking too much from the earth” \autocite{Mann2018Smithsonian}. “The battle to feed all of humanity is over[,]” according to \emph{The Population Bomb}, “In the 1970s and 1980s, hundreds of millions of people will starve to death in spite of any crash programs embarked upon now …We are today involved in the events leading to famine and ecocatastrophe; tomorrow we may be destroyed by them” \autocite[p.\ xi]{Ehrlich1971PopulationBomb}. Their extensive use of fatalist language is most evident when quantified. For example, in a section of just four pages they utilized over 25 instances of, as climate scientist Roger Revelle describes, “apocalyptic adverbs and adjectives''.\footnote{They use the following words and phrases: ``staggering, sobering, disaster (three times), enormously, drastically, catastrophic, dramatically, tremendous, highly lethal, extremely dangerous (twice), especially violent, more severe, extremely fortunate, extremely vulnerable, almost total, high potential, renewed spectre, not gruesome enough, colossal hazard, biological doomsday, superlethal, [and] disastrously effective.'' \cite[pp.\ 66-70]{Revelle1971}.} Later in the early 1980s, it was exactly this type of oratory that was popularized in China and that convinced its leaders of the need for strict, coercive policies.

Four years after \emph{The Population Bomb}, \emph{Limits to Growth} (LTG) was published, co-authored by researchers at the Club of Rome (COR), an informal group of intellectuals and businessmen who discussed the world’s most pressing issues (\cites{Masini2001}{ClubOfRomeAbout}). \emph{LTG} was another work that has its origins in Malthus, though it adopts a more mathematical approach. Based on their simulation model World3, the COR scientists asserted that the global population would reach its maximum capacity in approximately a hundred years, and that continued population growth would surpass the planet’s “limit to growth,” leading to a catastrophic collapse of human society (\cites[p.\ 23]{MeadowsEtAl1976}[p.\ 132]{Greenhalgh2008JustOneChild}).

Concurrently, Western scientists were also authoring studies that saw the application of cybernetics in population control, more specifically, its potential as a means of preventing irreversible damage to the Earth’s ecosystems. Often, their conclusions were quite radical. For example, the authors of \emph{A Blueprint for Survival} (1972) recommended nearly halving the British population, reducing it from 56 million to 30 million (\cites{Goldsmith1972}[p.\ 133]{Greenhalgh2008JustOneChild}[p.\ 359]{Kwakernaak1977}). Similarly, a Dutch study proposed an even steeper reduction, suggesting a decrease from 13.5 million to 5 million – a decrease of approximately 63\% \autocite[p.\ 359]{Kwakernaak1977}. However, though the impressive results obtained by these calculations were intended only as exploratory policymaking tools, they would provide the groundwork for justifying extensive population control after being imported into the PRC.

\subsection{Scientism in China}

In the early 1910s, a long-awaited period of modernization and transformation finally occurred in the newly formed Republic of China. Immediately after its establishment, a new generation of youth began eagerly advocating for the replacement of old cultural mores and preconceptions with modern Western social values. The glorification and reverence of science played a crucial role in this movement, as many saw science as the answer to China’s obsolete Confucian ideals, which they believed were the root cause of the nation’s weakness on the international stage \autocite[pp.\ 76--78]{Lee1987IronHouse}. Many young intellectuals believed that if China did not overcome its national weakness, it would be overtaken and destroyed by predatory, imperialist states \autocite[p.\ 34]{Uberoi1968Scientism}. This Darwinian mindset also justified the displacement of traditional values with new, “scientifically” founded ones, a seemingly natural evolution to replace China’s outmoded and inadequate ideas \autocite{Uberoi1968Scientism}. These ideas were broadcast widely with the fiery language and confident assertions made by leading youth intellectuals of the time.

Central to the New Culture Movement’s philosophy was the figure of “Mr. Science (\emph{sai xiansheng}),” a personification of the Western scientific ideal that came to embody the movement’s push for progress and enlightenment. Chen Duxiu, one of the most prominent members of the movement and founder of the radical journal \emph{New Youth} (\emph{xin qingnian}), proclaimed that “[w]e now believe that only these two Messrs. [Mr. Democracy and Mr. Science] can eliminate the darkness in China’s politics, morality, learning, and thought” \autocite[p.\ 280]{Fan2022MrScience}. While a steadfast belief in the capabilities of science was apparent in China at large, given Chen Duxiu's key role as one of the CCP’s founders, it would continue to be especially prevalent in the Communist Party.

\section{A History of Family Planning in China}
\subsection{Family Planning in Ancient China}

China was, over its five-millennium long history, decidedly pro-natal, encouraging human reproduction and population growth. Traditional Confucianism emphasized the importance of a large family, declaring that it promoted happiness \autocite[p. 935]{WangEtAl2016AJS}; in fact, remaining celibate was unfilial. Mencius, a follower of Confucius, stated that “there are three things that are unfilial (\emph{bu xiao you san}), and to have no posterity is the greatest of them (\emph{hou wu wei da})” \autocite{MenciusLeggeWikisource} The \emph{I Ching} (\emph{yijing}), a core text of Confucianists doctrine, tells that “it is the great virtue of heaven and earth to bestow life” \autocite{LiShuhua2012IChing}. Moreover, since having more children increased the productivity of the family, the basic unit of production in ancient China, rulers commonly promoted pronatalist policies (\cites[p.\ 15]{EntwisleHenderson2000}[p.\ 933]{WangEtAl2016AJS}{SJSUWatkinsCulturalRevolutionSite}). While incentive structures such as state-subsidized midwifery and material benefits like foodstuffs, land, livestock, or labor exemptions regularly accompanied childbirth \autocite[pp.\ 931--934]{WangEtAl2016AJS}, authorities also employed disincentives. For example, if individuals did not marry at a certain age, the government would arranged marriages or imposed jail time \autocite[p.\ 933]{WangEtAl2016AJS}. Despite the primitive methods used by these states, their pronatalist policies were generally successful, as such ideas have persisted in rural areas into the modern day.

These examples also illustrate a crucial element that facilitated the one-child policy: robust state control. China’s government, a historied bureaucratic formation, is an overpowering one \autocite[pp.\ 168--223]{Hui2005}. Its bureaucracy is not only efficient, but also capable of mobilizing on a remarkably impressive scale; however, it is also persistently intrusive and dictatorial. It was precisely the statist nature of the Chinese government that allowed it to conduct ambitious, large-scale family planning programs.

\subsection{Family Planning in 20th Cenutry China}

Following the communist victory in the Chinese Civil War, Mao began implementation of his own population policies. While it is regularly claimed that Mao was an unwavering pronatalist (\cites{MaJian2013Guardian}{FongWang2022HRW}[pp.\ 35--52]{JisenMa1998PoliticsPopulationGrowth}), this generalization is inaccurate as his stance towards family planning can only be described as inconsistent and frequently contradictory. In the early 1950s, considering population growth a key driver of economic growth, birth control and contraceptives were condemned and banned by the state, respectively \autocite{Fitzpatrick2009TimeOneChild}. In 1945, however, the need for some form of family planning policy was realized after a census reported the population to be 530 million, much higher than what was previously assumed \autocite[pp.\ 7--8]{ZhangYangLiang2001}. Thus, birth control was encouraged \autocite[p.\ 53]{Greenhalgh2008JustOneChild}. Nevertheless, when the Great Leap Forward was launched in 1958, efforts were halted, only to briefly resume in the 1960s, when the State Council instated a Family Planning Commission, which initiated a supply of free contraceptives for couples of childbearing age \autocite[pp.\ 8--13]{Qian1983ChinaPopulationNewsletter}. In the latter parts of the decade, with the chaos of the Cultural Revolution breaking out around the nation, such efforts were again terminated \autocite[p.\ 53]{Greenhalgh2008JustOneChild}. Mao made his last statement regarding population control in December 1974, noting in a report by the State Planning Commission, “Population must be controlled (\emph{renkou fei kongzhi buxing})” \autocite[p.\ 54]{ZhangYangLiang2001}. This, in time, would set in motion the construction of a new field of Chinese demographic science---the science that would produce the one-child policy \autocite[pp.\ 68--60]{Greenhalgh2008JustOneChild}.

Before population control could begin, however, such a notion was still difficult to justify politically; as per Maoist doctrine, Malthusianism was equivalent to capitalism. It was argued that since capitalism needed a “reserve army of the unemployed,” a labor force “which [could] be employed at any time,” overpopulation had been engineered as a means of oppressing the impoverished \autocite[p.\ 938]{WangEtAl2016AJS}. The dysfunctions Malthus described were in fact the consequences of societal inequality that could be solved with the socialist mode of production. Further, they claimed that imperialist powers used overpopulation to justify expansionism, an especially pertinent example being the case of Imperial Japan.

Thus, before an adequate justification was produced, any study of demographics was forbidden, with those who attempted to create independent studies in the field suppressed and persecuted. Beginning in 1957, Mao launched his Hundred Flowers Movement, a seemingly benign, even welcomed campaign in which scholars and intellectuals were encouraged to voice opinions that challenged state policy \autocite[pp.\ 7--12]{WangEtAl2016AJS}. As Mao famously proclaimed, “[let] a hundred flowers bloom (\emph{baihua qifang}) and a hundred schools of thought contend (\emph{baijia zhengming})[!]” Quickly, however, criticisms of the regime rose to a level deemed unacceptable, and central authorities subsequently launched the “Anti-Rightist Campaign (\emph{fanyou yundong}),” superficially to remove rightist infiltrators of the CCP \autocite{ShaShangzhi2020}; in practice, it was implemented to silence political detractors \autocite[p.\ 3]{Vidal2016AntiRightist}. One such affected person was Ma Yinchu, an economist and president of Beijing University (beida), who, in March 1957, published “New Population Theory,” a paper that called for “strong measures” to prevent the rapid ballooning of China’s population \autocite[pp.\ 56--57]{Greenhalgh2008JustOneChild}. Despite his usage of Marxian instruments in justifying his proposal, he was swiftly denounced as being Malthusian in 1958, when Mao began supporting a more pro-natal approach towards family planning \autocite[p.\ 57]{Greenhalgh2008JustOneChild}. Thereafter, Ma was fired from his position at \emph{beida}, becoming a political “nonperson” \autocite[p.\ 687]{Tien1981DemographyInChina}. Ma’s fall from grace marked the beginning of a decade-long taboo surrounding population science that was only destigmatized by Premier Zhou Enlai in 1970. 

\subsection{``Longer, Later Fewer'' \& the One-Child Policy}

In 1973, with the introduction of the LLF (\emph{wan, xi, shao}) slogan at China’s first official national birth conference, various restrictions on marriage and childbirth were implemented \autocite[p.\ 88]{GreenhalghWinckler2005}. In urban areas, bridegrooms were required to be above 28, and brides above 25, to marry; in rural areas, it was 25 and 23 respectively (\cites[p.\ 586]{BongaartsGreenhalgh1985}[p.\ 149]{WhyteWangCai2015}). Furthermore, couples were required to delay each child by at least four years (\cites[p.\ 149]{BongaartsGreenhalgh1985}[p.\ 149]{WhyteWangCai2015}), and a limit of two children for urban couples and three for rural couples was introduced \autocite[p.\ 586]{BongaartsGreenhalgh1985}. This number would eventually be reduced to two children, irrespective of urbanity. It was also during this period that political justification for population control was at last produced. In the preface to Engels’ \emph{Origin of the Family, Private Property, and the State} (1884), he describes a “twofold character” to production. On the one hand, one must contend with the production of the “means of existence,” and on the other, the (re)production of humans \autocite[pp.\ 71--72]{Engels1972}. Thus, since societal organization is determined by both types of production, only a “unified socialist plan,” where both are regulated by the state, will be able to manage an optimal balance between the two \autocite[p.\ 71]{Greenhalgh2008JustOneChild}.

By 1975, official statements regarding population began to circulate between the Central Committee and provincial governments, with birth planning often given primary focus on provincial agendas \autocite[p.\ 89]{GreenhalghWinckler2005}. What marked a pivotal shift was the central government’s move from focusing on the spacing of births to reducing the overall number of births. This transition culminated in the 1978 Fifth National People’s Congress, where family planning became an official state matter and a “basic national policy” \autocite{LinZhibo2007};  in fact, an article was inserted into the Chinese constitution, declaring that “[t]he state advocates and encourages planned reproduction” \autocite[p.\ 65]{Tien1980WanXiShao}. Alongside this, the national birth planning group was expanded, along with adopting a new slogan that reflected a revamped, more aggressive approach to controlling population.\footnote{“[P]arty secretary in command (\emph{shuji guashuai}), whole party acts (\emph{quandang dongshou}), propaganda and education (\emph{xuanchuan jiaoyu}), models show the way (\emph{dianxing yinlu}), strengthen scientific research (\emph{jiaqiang keyan}), improve [medical] technique (\emph{tigao jishu}), implement measures (\emph{cuoshi luoshi}), mass movements (\emph{qunzong yundong}), persevere (\emph{zhizhi yiheng}).” \cite{GreenhalghWinckler2005}.}

Towards the turn of the decade, the one-child policy began to take shape tangibly. Susan Greenhalgh traces this decisive “gestational” period (approximately between December 1979 and September 1980) of the soon-to-be policy by focusing on three phases of its formulation \autocite[p.\ 198]{Greenhalgh2008JustOneChild}. First, a group of Chinese cyberneticists, experts in control theory, began pushing for the application of scientific methods to population management \autocite[pp.\ 211--223]{Greenhalgh2008JustOneChild}, gerrymandering the boundaries of population science to insert their own solutions to the problem (\cites[p.\ 195]{Greenhalgh2008JustOneChild}[pp.\ 1--35]{Gieryn1999}). Beginning in January 1980, the cyberneticists formed strategic alliances with high-ranking, influential politicians and scientists, investing their own proposals with substantial political force \autocite[chap.\ 7]{Greenhalgh2008JustOneChild}. By April, the one-child policy emerged from the political arena as the unchallenged victor; in September, it was officially unveiled to the public in an open letter, signaling the beginning of a new era of Chinese family planning \autocite[pp.\ 298--299]{Greenhalgh2008JustOneChild}.

\section{Radicality Defined}

Having briefly illustrated a synopsized history of population control in China, this paper will now explore how the one-child policy was distinctly “radical” as compared to earlier measures. To highlight these differences, the LLF rule is compared with the one-child policy, with their temporal adjacency minimizing extraneous differences across comparisons. Thus, three distinguishing factors can be found between the two periods, demonstrating the radicality of the one-child policy: its overbearing limit of “one child,”\footnote{It should be noted that while there was no uniform “one-child policy,” with differing strengths of enforcements and a multitude of exceptions across the nation, such was only the case due to circumstances beyond the control of the central government. This will be discussed in the following paragraph.} its “top-down” enforcement style, and its legitimized and extensive uses of coercive methodologies.

First, the one-child policy required reducing the fertility rate to one child per family, though there were regional variations, such as in rural regions or regions with ethnic minorities \autocite[pp.\ 373--387]{ShortZhai1998}. While during LLF, authorities placed a limit of two children in urban areas \autocite[p.\ 586]{BongaartsGreenhalgh1985}, the one-child policy saw the number of families with two children reduced to 5.8\% of the population \autocite{Xinhua2013BackgroundBirthPolicy}. In the same census, the percentage of one child and one-and-a-half child families (families allowed more than one child, especially if the first child was female or disabled) was 37.5\% and 52.8\% of the total population, respectively \autocite{Xinhua2013BackgroundBirthPolicy}. Despite the existence of families with two or more children, such cases were the exception, not the norm.

Second, the one-child policy employed a “top-down,” statist method of policy enforcement. While health officials and “moderate development-minded leaders” cooperated to construct LLF, ensuring that it reflected “popular interest” \autocite[p.\ 66]{Greenhalgh2008JustOneChild}, the one-child policy consisted largely of centrally devised and disseminated population quotas. These quotas were often highly unrealistic, forcing local cadres to employ coercive measures to avoid penalties \autocite[pp.\ 155, 164]{LiXiaorong2015LicenseToCoerce}. Taking advantage of the Chinese state’s immense power and acting in opposition to Mao’s “mass line” doctrine of enforcing political agendas\footnote{“Mass line” refers to a Maoist methodology for implementing policies, where an initial policy is formulated, revised based on testing, and retested. See Steiner, H. Arthur. “Current ‘Mass Line’ Tactics in Communist China.” American Political Science Review 45, no. 2 (June 1951): 422–436.}, the one-child policy blatantly overruled the people’s concerns. In fact, authorities regularly performed family planning actions despite violent resistance in the countryside, with the state “quietly accepting” the use of force as a means of policy enforcement \autocite[p.\ 166]{Greenhalgh2008JustOneChild}. Furthermore, the political momentum built up around the bloated family planning apparatus was a key factor in the continuation of the one-child policy, making the termination of such an institution difficult \autocite[p.\ 126]{FengCaiGu2013}. Undoubtedly, the obstinate nature of a “top-down” administrative method and the convoluted central birth-planning apparatus driving the policy contributed to its invasiveness and duration. 

Third, the one-child policy legitimized and even sanctioned discriminatory and coercive practices. Although the LLF rule saw the initially implementation of many violent enforcement mechanisms, these measures were only popularized during the one-child policy \autocite[pp.\ 150--151]{WhyteWangCai2015}. It was the one-child policy’s stricter criteria and overbearing governmental pressure that led to the widespread acceptance and application of coercion \autocite{LiXiaorong2015LicenseToCoerce}. For instance, besides depriving out-of-plan children of household registration (\emph{hukou})\footnote{An essential component in receiving government subsidized education, having access to public transportation, getting married, opening a bank account, among other things. See Wang, Zhihe, Ming Yang, Jiaming Zhang, and Jiang Chang. “Ending an Era of Population Control in China: Was the One-Child Policy Ever Needed?” The American Journal of Economics and Sociology 75, no. 4 (2016): 929–979. Accessed June 19, 2024. \url{https://www.jstor.org/stable/45129326}.}, during the one-child policy, even the subsidies provided to the family’s previous children were forcefully “returned” to the state \autocite[p.\ 157--158]{LiXiaorong2015LicenseToCoerce}. Furthermore, there were also exceptionally hefty fines, up to around 30 to 50\% of the median income in some areas, and usually over 50 times the subsidies offered for having only one child. We also find discrimination towards women and female children. Despite the recent invention of the no-scalp vasectomy, a far safer and more convenient measure compared to female sterilization, vasectomy rates were far below those of their female counterparts \autocite[p.\ 2]{GonzalesEtAl1992NoScalpel}. Moreover, no action was undertaken to address the perception that male children were more valuable than female children; in 1980, five percent of female babies “disappeared” from official records, presumably a victim of abandonment or infanticide, a phenomenon all too common \autocite[pp.\ 159--160]{LiXiaorong2015LicenseToCoerce}. In its most intense year, 1983, the one-child policy saw 14.4 million abortions, 20.7 million female sterilizations, and 17.8 million IUD insertions, the vast majority of patients being involuntary ones \autocite[p.\ 154]{WhyteWangCai2015}. As evidenced, the one-child policy not only enforced strict limitations on family size, but also perpetuated harmful gender biases and practices. 

Finally, it should be noted that despite LLF’s comparatively lax policy enforcement mechanisms and family planning obligations, it was far more effective at reducing fertility while avoiding adverse socioeconomic consequences. During LLF, China’s fertility rate saw a decrease of roughly 0.3 per year, while the sex ratio during this period remained stable \autocite{FangChen2018VoxEU}. On the other hand, the one-child policy from 1980 to 2016 saw China’s fertility decrease by merely 0.025 per year \autocite{HuangSilverPew2022}, while the sex ratio rose to around 120 males per 100 females born, only dropping to relatively mundane levels post-2010. Surprisingly, the one-child policy was less effective and more destructive than its predecessor.

\section{Radicality Explained}

To begin investigating the one-child policy, understanding Deng Xiaoping, especially his political restructuring, is essential. After Mao’s death, a brief power struggle unfolded between Hua Guofeng, the de facto leader and supporter of Mao’s policies, and Deng Xiaoping. Eventually, Deng successfully outmaneuvered Hua, becoming China’s paramount leader by 1980. In a bid to reinforce party legitimacy, Deng employed several political tactics that helped promote the new regime as being “pragmatic-not-dogmatic” and based on “facts-not-ideology” \autocite[p.\ 97]{Greenhalgh2008JustOneChild}. This approach is best exemplified by two of Deng’s slogans: “crossing the river by feeling the stones (\emph{mozhe shitou guohe}),” and “seeking truths from facts (\emph{shishi qiushi}).” Such slogans underscored the difference between Mao, whose political actions were “ideological” and “utopian,” and Deng, whose approach was methodical and pragmatic \autocite[p.\ 97]{Greenhalgh2008JustOneChild}. Indeed, the ideological difference between Dengist and Maoist went far beyond the merely rhetorical, as many of Deng’s actual policies reflected a scientistic perspective on governance.

\subsection{Deng's New China}

One particularly relevant example of Deng’s political stance was his “Four Modernizations (emph{sige xiandaihua}),” introduced in 1978 with the goal of strengthening China’s agriculture, industry, defense, and science and technology (S\&T). Deng wanted to create a \emph{xiaokang shehui}, or a “modern socialist country that is prosperous, strong, democratic, culturally advanced, harmonious, and beautiful (\cites{LiNan2018ThirdPlenumCommunique}{Xinhua2020Xiaokang}).” Though the Four Modernizations were unofficially announced by Zhou Enlai in 1963 \begin{CJK}{UTF8}{gbsn}\autocite{RenminArchive2016ScienceTechMeeting}\end{CJK}, the political turbulence caused by Mao’s various political campaigns prevented it from being realized. Despite this, Zhou, one of the most popular leaders in the CCP, supported the Four Modernizations throughout his life, imbuing the proposal with significant political gravitas. Deng would be the one to finally put the plan into action, advocating for the four modernizations as one of the core tenants of his agenda to modernize China, especially focusing on S\&T \autocite[p.\ 94]{Greenhalgh2008JustOneChild}. Later, birth planning was proposed as to negate the effects of China’s enormous population in their bid to realize the four modernizations.

Another example of Deng’s revisionist thinking is found in his theory of “socialism with Chinese characteristics (\emph{zhongguo tese shehuizhuyi}),” which saw the partial adoption of free-market economics in China to promote economic growth. It was argued that only by first adopting such measures, would the emergence of a Marxist communist society become conceivable \autocite{Fengqi2021DengRichFirst}. To quantify these proposals for economic growth, Deng stated that his economic reforms would allow China to achieve a per capita output of \$1,000 by the turn of the century – more than quadruple per capita output in 1980. Compared to Mao’s vague, abstract expressions, Deng’s scientistic and utilitarian approach towards realizing a socialist state imbued his regime with inherent authority and legitimacy; scientism would come to infiltrate all aspects of governmental decision-making, ultimately allowing for the creation of the one-child policy.

\subsection{Scientism in Politics}

Why was the influence of scientism so prominent in Chinese politics? Two causes can be identified: first, under Mao, the natural sciences were granted significant political influence. Second, the natural sciences were able to sway top CCP officials with complex, often incomprehensible, mathematical models and procedures.

During Mao’s time in power, he invested significantly in defense science, allocating substantial amounts of developmental resources for their research. As such, defense science enjoyed access to “foreign literature, … data, and … [electronic] computers \autocite[p.\ 139]{Greenhalgh2008JustOneChild};” accordingly, many top officials began to view their work as being more credible and “scientific” than the work of social scientists, who were barred from enjoying similar advantages. Even after Mao’s death, a preference for the natural sciences remained. Whereas the natural scientists were no longer constrained by party politics, oriented by “the reasoning of modern science and mathematics” alone, social scientists were used as a tool by politicians to “empirically illustrate” problems and “articulate” solutions in the debilitating framework of Maoist-Leninist thought \autocite[p.\ 84]{Greenhalgh2008JustOneChild}.

A germane example of the CCP’s favorability towards the natural sciences was, to borrow terminology from Susan Greenhalgh, the “Song group” \autocite{Greenhalgh2008JustOneChild}, a group of cyberneticists led by Song Jian, a leading rocket scientist. Through Mao’s goodwill, Song acquired extensive political connections and access to population data unavailable to social scientists; they received endorsements from top officials and scientists who brandished enormous political influence \autocite[p.\ 140]{Greenhalgh2008JustOneChild}. Among these could be found Qian Xuesen and Xu Dixin, father of the Chinese space program and head of the Population Association of China respectively \autocite[p.\ 243]{Greenhalgh2008JustOneChild}. As previously mentioned, scientism had existed in China since the beginning of the 18th century; their early influence on the nascent Communist Party of the 1920s and 30s almost certainly contributed to this reverence of the natural sciences, which influenced the CCP during and beyond Mao’s time. Song’s political dominance will play a crucial role in constituting the one-child policy.

Similarly, in their formulation of the population problem, Song relied heavily on complex mathematics, which many could not understand. Officials accepted Song’s findings entirely because they “used control theory [(cybernetics)],” or that “the mathematics and equations were impressive.” One individual even rather egotistically acknowledged that the calculations seemed credible as “he himself could not do them” \autocite[p.\ 246]{Greenhalgh2008JustOneChild}. As only Song had the means to interpret their own models, they received recognition and credibility as the “guardian[s]” of their work \autocite[p.\ 294]{CaidenWildavsky1980}. Such blind acquiescence towards the natural sciences not only permitted Song to have substantial sway over policy decisions, but also cemented the future prevalence of scientism in policymaking. Song repeatedly underlined the scientific rigor and objective nature of his work, in one case, even adding arbitrary decimals to demonstrate the exactitude with which his calculations were apparently conducted \autocite[p.\ 218]{Greenhalgh2008JustOneChild}. “[T]here are still quite many intellectuals who,” as he commented in a paper, “starting off with [biased] sentiment, go so far as to challenge the irrefutable logic of [the] natural sciences” [emphasis added] \autocite{Song1985SystemsScienceReforms}. Such use of rhetoric and display of mathematical prowess would be key in establishing ethos for claims made by the cyberneticists.

Concurrently, while many touted the cyberneticists’ works as irrefutable, the work of social scientists was perceived as being uninformed and unsophisticated. In lieu of the cyberneticists’ computer-generated graphs and figures, social scientists had only “crude,” hand drawn ones that were hardly comparable \autocite[p.\ 209]{Greenhalgh2008JustOneChild}. Furthermore, social scientists generally pushed for demographic policies that were far more moderate, making them seem “unoriginal,” “too ideological,” and not scientific enough compared to the cyberneticists \autocite[p.\ 260]{Greenhalgh2008JustOneChild}. 

\subsection{Scientistic Methodology}

We can now articulate four key aspects of Song's approach to population science: first, they reframed the population crisis as an environmental issue; second, they drew heavily on methodological principles from rocket science; third, they often relied on incomplete and unreliable data; and fourth, they overlooked the potential social repercussions of their policy.

First, Song re-conceptualized the population crisis as not merely a national concern, but a \emph{global, environmental} one. In essence, they argued that China’s population growth was not only a threat to national security and survival \autocite[p.\ 153]{Greenhalgh2008JustOneChild}, but that it was an existential and all-encompassing threat to the entirety of “human survival.” Not only did they argue China could not realize its four modernizations or achieve a \emph{xiaokang} society if population growth was to continue, but also that China’s role in engendering this looming demographic and ecological crisis would severely damage the nation’s international reputation \autocite[p.\ 154]{Greenhalgh2008JustOneChild}. In a manner analogous to their Western counterparts, Song painted images of environmental devastation, portraying unmitigated population growth as the Earth's foremost killer. To demonstrate the imperativeness of the population crisis, Song conspicuously borrows the clichéd Western metaphor of Earth as a spaceship, how even the vast cosmos offers no recourse (\cites[p.\ 149]{Greenhalgh2008JustOneChild}[p.\ 21]{Ehrlich1971PopulationBomb}). Thus, they framed the one-child policy as the “only solution” to keep China’s population below 1.2 billion, which they calculated to be the limit beyond which Deng’s previously mentioned economic objectives would become unattainable, and beyond which the environmental damage would be enormous and irreversible \autocite[pp.\ 158, 240]{Greenhalgh2008JustOneChild}.

Second, Song advocated for the central government to draft a strong, unidirectional policy that the various provincial administrations would further disseminate down the hierarchy. Upon such a structure, each administration would develop and implement a unique birth control plan modified with regional constraints in mind \autocite[pp.\ 29--32]{SongTuanYu1985}. Thus, regional authorities would remain adhered to the central formulation despite regional variations in enforcement strength and punishment. Influenced by his work in the defense industry, Song’s proposal was entirely “top down,” rejecting the responsive nature of LLF, even quietly accepting the use of coercion “in the interest of achieving greater goals” \autocite[p.\ 166]{Greenhalgh2008JustOneChild}. The complex and mathematical nature of Song’s models necessitate that policies originate from a central government to its citizens, resulting in a “programmatic” approach to policy implementation and enforcement \autocite[p.\ 262]{ArthurMcNicoll1975}. Evidently, the forceful and unresponsive nature of the one-child policy made it highly coercive.

Third, Song relied heavily on spotty statistics, resulting in inaccurate conclusions. Although China had a robust system of census collection during the 50s and 60s, the Great Leap Forward had caused a collapse in census infrastructure, preventing accurate population data from being gathered \autocite[chap.\ 2]{Banister1987}. As such, no demographic data were collected in the 1970s \autocite[p.\ 62]{Greenhalgh2008JustOneChild}. One member of Song described there as being “no [good] input data (\emph{meiyou shuju}),” and that though the data was “difficult (\emph{kunnan}),” they were “workable (\emph{kao de zhu})” \autocite[p.\ 161]{Greenhalgh2008JustOneChild}. In one case, they conducted calculations under the assumption that Chinese protein intake matched that of the West, despite China’s agricultural sector being far too underdeveloped to facilitate such consumption \autocite[p.\ 159]{Greenhalgh2008JustOneChild}. Despite the lack of reliable data as the basis for drawing conclusions (much less informing policy), Song nonetheless conducted ostensibly precise calculations while being forced to make “countless heroic assumptions” regarding their data \autocite{Greenhalgh2008JustOneChild}. Perhaps most alarmingly, numerous concerns regarding potential defects in the data were all but dropped by the time conclusions were formulated. Song merely presented their findings as infallible \autocite[pp.\ 161, 217]{Greenhalgh2008JustOneChild}, while government authorities censored any complaints regarding the accuracy of the data \autocite[pp.\ 257--259]{Greenhalgh2008JustOneChild}. It was this combination of erroneous calculations and political invulnerability that eventually birthed the one-child policy.

Lastly, Song ignored the social consequences of a one-child policy. In an article published in the People's Daily, Song and his colleagues claimed that “we [China] cannot run into these problems in this century, and in the first twenty years of the twenty-first century they will not be serious” \autocite[p.\ 248]{Greenhalgh2008JustOneChild}. Not only did they argue that social consequences were of no major concern, they further claimed that were one to take such concerns into account, any attempt at birth control would be impossible. As Song stated in a paper, “If social customs and psychological conditions are considered, $\beta(t)$ [the Total Fertility Rate] can hardly go below a level acceptable to the public” \autocite[p.\ 251]{SongTuanYu1985}. Much like his recommendation of a “top down” enforcement solution, Song’s ignorance of social consequences can again be attributed to his work in the defense industry. Under such circumstances, as one might imagine, social consequences were seldom considered \autocite[p.\ 153]{Greenhalgh2008JustOneChild}. Further, despite Song’s ability to mathematically calculate the population trajectory, they did not have the analytical tools and empirical data to grasp the gravity of resulting social consequences, allowing the cyberneticists to merely write off such abstract concerns raised by social scientists as being mere speculation \autocite[p.\ 286]{Greenhalgh2008JustOneChild}. Moreover, due to the manner in which Song had presented the population crisis, regarding it as an impending, global, and existential environmental threat, the one-child policy was portrayed as the price that had to be paid in order to save not only China, but the world, from “impoverishment and extinction” \autocite[p.\ 287]{Greenhalgh2008JustOneChild}. As Song emphasized: “At least before the end of this century, we must stick to the policy of one child per couple, so that the people of the 21st century will be able to enjoy family happiness” \autocite[p.\ 4]{Song1985SystemsScienceReforms}.

\subsection{The West in China}

In 1978, Song travelled to Helsinki to participate in the Seventh Triennial World Congress of the International Federation of Automatic Control. During his travels, he encountered Western cyberneticists, some of whom were experimenting with applying control theory to demography \autocite[p.\ 142]{Bacaer2011}. Finding this idea appealing \autocite[p.\ 144]{Greenhalgh2008JustOneChild}, Song came to borrow many originally Western conceptualizations of the population crisis: an emphasis on the environmental impacts of overpopulation, the aforementioned analogy of Earth as a spaceship, the formulation of overpopulation as a transnational issue, and many others. Additionally, they imitated Western tactics of manipulating empirical data to underscore the scale of the crisis. For one such case, they illustrated rapid population growth by showing population increases over progressively shorter time intervals \autocite[pp.\ 6--7]{Goldsmith1972}. They also defined an “optimal population” based on the environment's carrying capacity, arguing that unregulated growth would exceed this limit \autocite[pp.\ 46--47]{Goldsmith1972}. However, Song not only borrowed the Western construct of overpopulation; he combined fearmongering language from Western works such as \emph{The Population Bomb} with the obdurate belief in “scientism” present in the CCP to produce new rhetorical apparatuses.

Susan Greenhalgh identifies three main rhetorical tools Song used: \emph{quantification}, \emph{categorization}, and \emph{comparison}. First, Song reformulated textual reports into numerical reports, stripping them of their original context. Through a process Greenhalgh terms “faerification,” empirical statements were reborn as “facts” and categorical truths, truths independent of circumstance (\cites[p.\ 109]{Greenhalgh2008JustOneChild}[p.\ 23]{Latour1987}). This fell in line with Deng’s doctrine of “seek truth from facts,” allowing their calculations to be easily politicized, and to be seen as having originated from pure mathematical logic, untainted by political machinations. Second, they created visual representations of data, harnessing “ocular power” to present information in an easily digestible manner. In doing so, they invented a “new demographic and political reality” through their graphs, compacting their intricate computations into a universally accessible form \autocite[p.\ 110]{Greenhalgh2008JustOneChild}; this, as previously mentioned, also allowed them to manipulate the data in ways that supported their conclusions. Third, they highlighted China’s apparent deficiencies through comparisons with the West, portraying China as a backwater nation that \emph{should} have been able to compete with the West but was hopelessly outmatched due to its unwieldy population. Using these three devices, Song “revealed” overpopulation as the root cause of China’s ills.

It should be noted that despite Song’s diligent borrowing from the West, there remain aspects of Western methodology that were not adopted. One such difference was that when Western scientists argued for a reduction in fertility, they typically imagined such a process to occur over an extended period of at least several decades. For instance, while Dutch scientists proposed that a 40\% reduction in fertility transpire over the course of 40 years \autocite[p.\ 365]{Kwakernaak1977}, Song proposed that Chinese fertility be reduced by 50\% in merely \emph{five} \autocite[p.\ 163]{Greenhalgh2008JustOneChild}. Unlike the Song group, towards whom criticisms were either ignored, disregarded, or outright forbidden due to their lack of political capital, Western literature regularly received public criticism. For example, The Population Bomb was cast as fearmongering and unscientific \autocite[pp.\ 66--70]{Revelle1971}, and the LTG, methodologically lacking and scientifically inaccurate \autocite[pp.\ 1182--1183]{Nordhaus1973}. All things considered, the most prominent and unanimous concern of various scholars was that any conclusions reached through long-term demographic projections are highly contingent on uncontrolled factors; that such deductions are at best half-truths, if not outright misleading and fallacious statements (\cites[p.\ 1183]{Nordhaus1973}[p.\ 52]{Sanderson1994SimulationModels}[p.\ 133]{ColeEtAl1973}). As evidence, while Western natural scientists had checks and balances all the while controlling little practical political influence, Chinese natural scientists practically dictated state policy.

Aside from influencing China in a purely intellectual sense, Western international organizations also offered financial incentives in support of Chinese birth control programs. According to the Chinese Embassy in Norway, by 1999, the “UNFPA [United Nations Fund for Population Activities] had provided US\$177 million of assistance and carried out 123 projects in China” \autocite[pp.\ 953--954]{WangEtAl2016AJS} These funds contributed financially to the continuation of the one-child policy, representing an international endorsement of Chinese family planning efforts \autocite[p.\ 102]{YiFuxian2013}. In 1983, the UN Secretary-General even awarded Qian Xinzhong, the minister of family planning, with the United Nations Population Award, apparently impressed by how the Chinese government had “marshaled the resources necessary to implement population policies on a massive scale” \autocite{PDR1983UNPopAward}. Private foundations also contributed significantly to birth control efforts. For example, the Rockefeller Foundation provided financial support for birth control programs, supplied free condoms, and initiated efforts for China to begin domestic production of birth control pills (\cites[p.\ 119]{Chung2002}[p.\ 954]{WangEtAl2016AJS}). Even the US government supported the policy, though only indirectly. From 1965 to 2004, \$17.3 billion was invested into the UNFPA under a directive known as NSSM-200, commonly known as the “Kissinger Report” \autocite[p.\ 27]{Clowes2004KissingerReport}, through which a substantial portion of the funds were funneled to China \autocite[p.\ 956]{WangEtAl2016AJS}. 

The motivations of these Western sponsors varied widely. While in the case of the UNFPA, their support of the one-child policy seems to have been a by-product of their effort to increase access to birth control worldwide, the Rockefeller Foundation had the explicit agenda of solving overpopulation, with Rockefeller himself describing population growth as “an outstanding problem” \autocite[p.\ 954]{WangEtAl2016AJS}. Even more malicious was NSSM-200, which sought to continue the extraction of mineral resources from third-world nations; in fact, the document itself suggests that US aid should only be conducted through international non-governmental organizations as to avoid charges of “economic or racial imperialism” \autocite[p.\ 382--383]{Grimes1998ReproductiveRights}. As illustrated here, Western involvement in the formulation and preservation of the one-child policy went far beyond the merely intellectual, as both direct financial assistance and indirect political legitimization played a part.

\section{Conclusion}

What price is worth paying for survival? In 1980, China’s answer to this question was the one-child policy. On September 25th, an open letter was addressed to the members of the Communist Party and the Youth League. Although the letter itself was less than two thousand characters long, and had been formulated in a matter of months, it would come to inform Chinese thinking for the next 40 years. Inevitably, the confluence of Mao’s disastrous campaigns, Western fears regarding overpopulation, and Scientism’s entrenched nature in Chinese intellectual culture produced the one-child policy. Like a vast dam attempting to hold back a powerful river, China’s one-child policy was an attempt to control the uncontrollable – the very flow of life itself. However, the policy was by no means adopted out of any semblance of convenience; instead, it was framed as China’s final recourse, “deemed a ‘solution when there was no solution’ (\emph{meiyou banfa de banfa}),” an ultimate measure implemented when all other option had failed \autocite[p.\ 287]{Greenhalgh2008JustOneChild}. 

Ultimately, the one-child policy is best seen as the result of a tyrannical government asserting its control over its citizens, exacerbated by a series of scientific missteps and pervasive Western influence. The one-child policy illustrates the power of “science” in shaping policymaking, even when unjustified, and underscores the significant errors that unchecked government authority can produce \autocite[p.\ 343]{Greenhalgh2008JustOneChild}. Science, blindly accepted, is akin to gospel in today’s world; globalization has advanced tenfold since the inception of the policy; authoritarianism still reigns in a myriad of regions throughout the world. The one-child policy will be remembered as one of the most radical social experiments in recent history, a uniquely modern intersection of science and politics. It is a policy that formed a pillar of China’s national identity for more than three decades, shaping the trajectory of its society, economy, and culture. As historians look back upon this era, they perceive a nation wrestling with immense challenges – both those of its history and those it unknowingly created for its future. Only time will tell if China chooses to rewrite its story or remain bound by the past. The one-child policy may have been uniquely Chinese, but the lesson learned is universally applicable; we cannot reduce demographics to mere mathematics, as the cost of ignoring the human element is far too high.



\begin{CJK}{UTF8}{gbsn}
\nocite{*}
\printbibliography
\end{CJK}

\end{document}

