%! TEX program = LuaLaTeX 
\documentclass[12pt, a4paper, twoside]{article}
\usepackage{format}
% Do not alter above

% Metadata: put your article information here 
\newcommand{\jtitle}{The Zhejiang-Jiangxi Campaign}
\newcommand{\jauthor}{Person E}
\newcommand{\jaffiliation}{some school}

% Editors will change these fields after acceptance 
\newcommand{\jvolume}{3}
\newcommand{\jyear}{2026}
\newcommand{\jdoi}{10.17613/82c3s-nac44}  

% References should be placed in refs.bib and cited with \autocite{<source>}
% Quotations can be placed in quote environments: \begin{quote}<your quote>\end{quote}
% Footnotes can be added with \footnote{<your footnote>}

% Your Content

\begin{document}

\maketitle{}

On the 18 of April 1942, 16 B-25 bombers were launched from the USS Hornet towards the Japanese archipelago. Led by Colonel James Doolittle, the Doolittle raid's purpose was twofold: retaliation for the Pearl Harbor attack four months prior, and instilling uncertainty into the Japanese populace \autocite[p.\ 10]{DoolittleGlines1995ICouldNeverBeSoLuckyAgain}. Although the physical damage caused by the raid was insignificant, its impact on American morale cannot be understated \autocite[chap.\ 18]{Scott2015TargetTokyo}.

Nevertheless, what is often left out of the narrative is the retaliatory campaign the Japanese launched following the Doolittle raid. In the Zhejiang-Jiangxi region of China, around 250,000 civilians were massacred, property was destroyed wholesale, and deadly biological weapons were indiscriminately used \autocite[p.\ 385]{PacificWarMuseum2024ChinaIncidentWarMedal}. I argue that the consequences of the Doolittle raid should be reevaluated, in light of the immense devastation in the aftermath of the Doolittle raid.

The history of Sino-American collaboration preceding direct US involvement in the Pacific War can be traced back to a combination of events in the late 1930s. Heretofore, despite friendly relations with China and a distaste for Japanese militarism, the US withheld support from the Chinese. The reasoning was threefold: First, fueled by the trauma of the First World War and the economic downturn brought about by the Great Depression, the US had adopted an isolationist foreign policy \autocite{StateHistorian2013AmericanIsolationism}. Second, the squabbles between the Communist and Nationalist parties added a factor of uncertainty \emph{vis-à-vis} the idea of transmitting aid \autocite{Doolittle1942GeneralReportJapaneseRaid}. Third, the US had no strategic interests in China and feared provoking a Japanese response \autocite[p.\ 69--70]{CannonEtAl2009IB20thCenturyWorldHistory}. In 1937, however, two events caused public opinion to turn decisively pro-Chinese: the Marco Polo Bridge Incident, which reignited Sino-Japanese hostilities; and the sinking of the gunboat USS Panay by Japanese bombing attacks \autocite[pp.\ 47--54]{WrightNelson1939AmericanAttitudes}. 

From that point on, the US became increasingly involved in the war. On 27 September 1940, Japan signed the Tripartite Pact, becoming an official member of the Axis \autocite{Britannica2023TripartitePact}. By mid-1941, the American Volunteer Group, otherwise known as the “Flying Tigers”, arrived in China and began offering informal assistance to the Nationalists \autocite{Doubek2021FlyingTigersNPR}. Later in August, the US enacted a ban on oil exports to Japan, citing a reluctance in shipping oil to “aggressor countries” (\cites[p.\ 74]{Feis1950RoadToPearlHarbor}{Roosevelt1941FreezingJapaneseAssets}). By December, the shortage of American imports finally prompted an aggressive Japanese response in the form of the Pearl Harbor attack on December 7, 1941, officially bringing the US into the Second World War \autocite{TAMUCC2024PearlHarborMotives}. Nonetheless, having caught the allies off-guard, the Japanese came to amass a sweeping empire stretching across the entire pacific by spring of 1942 \autocite{WWIIMuseum2019PacificStrategy}.

In response to the growing Japanese threat, President Franklin D. Roosevelt, collaborating with his military advisors, devised plan intended to both crush Japanese morale and reinvigorate the American war effort \autocite[p.\ 10]{Glines1988DoolittleRaid}. They planned to secretly send bombers to attack the Japanese mainland, without the protection of fighter aircraft. US fighter aircraft at the time lacked the ability to serve as long distance escorts, thus, it would have been extremely risky to deploy fighters for such a long-range operation \autocite{RogersNDInnovationDoolittleBlackMonday}. James Doolittle, an experienced flight instructor and test pilot, was chosen to lead the operation \autocite{Britannica2024JamesHDoolittle}. Initially, he had suggested that the bombers land in Vladivostok \autocite[p.\ 27]{Glines1988DoolittleRaid}. However, this was deemed to be impossible, as the Japanese had signed a neutrality pact with the Soviets back in 1941 \autocite{MolotovMatsuokaTatekawa1941NeutralityPactText}. It was then decided that the bombers would land in China, despite the fact that the airplanes would have to fly an additional 1,100 kilometers, and fierce Chinese resistance to this idea (\cites{Britannica2024DoolittleRaid}[p.\ 32]{Chun2006DoolittleRaid}[p.\ 27]{Glines1988DoolittleRaid}). Chiang Kai-shek, the leader of Nationalist China protested against raid, stating that such action would incite Japanese aggression, giving them a \emph{casus belli} to overrun the Chinese airfields and their respective provinces \autocite[p.\ 209]{Taylor2009Generalissimo}. However, by that point, the plan could not be altered. It had been decided that the airplanes would land first in the Zhejiang and Jiangxi provinces, refuel, then make their way to Chongqing (Chungking) \autocite[p.\ 32]{Chun2006DoolittleRaid}. Having constructed a plan, the carrier taskforce set sail April 1, 1942 \autocite[p.\ 440]{CravenCate1948AAFWW2Vol1}.

On the 18th of April, 1942, one day before the planned date of the attack, however, the naval task force was spotted by a Japanese picket boat \autocite[p.\ 442]{CravenCate1948AAFWW2Vol1}. The picket boat radioed Tokyo, notifying the military of an impending attack \autocite[p.\ 45]{Chun2006DoolittleRaid}. Despite having lost the element of surprise and being around 500 kilometers farther away than the predetermined launching spot, the bombers had to be immediately released \autocite{Correll2009DoolittlesRaid}. The Japanese high command failed to consider the possibility of an attack, observing the immense distance of the taskforce from the Japanese mainland; thus, the bombers faced no opposition \autocite[p.\ 441]{CravenCate1948AAFWW2Vol1}. Upon reaching Tokyo, Nagoya, Ōsaka, plus other minor targets, they were able to complete the mission while meeting minimal resistance \autocite{ArmyAirForces1942DoolittleReport}.

Following the successful attacks, most of the raiders proceeded South towards the Chinese mainland \autocite{Britannica2024DoolittleRaid}. Though, due to their early launch, none of the remaining planes had enough fuel to reach their intended refueling stops. As such, the aircraft crash-landed at various locations in the Zhejiang and Jiangxi provinces [\cites{Britannica2024DoolittleRaid}{Doolittle1942GeneralReportJapaneseRaid}].

Now, stranded in an active warzone, the crashed airmen turned to Chinese resistance fighters, who played an important role in ensuring the safe transportation of American airmen away from the frontlines. Of the 75 airmen who landed in Chinese territory, 64 were rescued by civilians and guerilla fighters who offered medical assistance, meals, and provided escorts to Chongqing (\cites{Leng2023DoolittleFriendshipGlobalTimes}{Walsh1992WPMeetAgaina}). As Lee Chennault, the leader of the Flying Tigers recorded in his memoir, “There are hundreds of American pilots and crewmen alive today who owe their lives to the aid of Chinese, farmers, guerrillas, and soldiers who guided them back to safety knowing full well that the price of detection was death for themselves, their families, and their community” \autocite[p.\ 169]{Hotz1949WayOfAFighter}. After the war, gunner and engineer David Thatcher recalled, “They did not have anything, but they gave us all they had” \autocite{Zhuang2018AfterTheAirRaidSCMP}.

Despite this, the fate of those who were captured by the Japanese military did not fear so well. Provisions were instituted by the Japanese government, applicable specifically towards the captured Doolittle raiders \autocite[p.\ 70]{Liverpool2008KnightsOfBushido}. It was stated that any air attack, “upon ordinary people, upon private property of a non-military nature, against other than military objectives, and [as a] ‘violation of war-time international law’” would be punishable by death, or a prison sentence exceeding ten years \autocite[pp.\ 70--71]{Liverpool2008KnightsOfBushido}. Four of the captured raiders perished in the hands of the Japanese: three being executed under the aforementioned provisions, and the other dying of dysentery \autocite{Britannica2024DoolittleRaid}. 

The Japanese military, infuriated by the boldness of the raid, launched the Zhejiang-Jiangxi campaign on May 15th, less than a month after the events of the attacks on the Japanese mainland \autocite[p.\ 386]{Schoppa2011SeaOfBitterness}. While the official rationale for the offensive was to destroy Chinese air bases threatening he Japanese mainland, the viciousness of this campaign speaks to an ulterior motive – retribution for the Doolittle raid \autocite[pp.\ 375--376]{Scott2015TargetTokyo}. The Japanese, with some 180,000 men in total, advanced upon the provinces of Zhejiang and Jiangxi. They moved along the Zhejiang-Jiangxi railway and occupying areas such as Quzhou (Chuchow), Yihuang (Ihwang), Nancheng, and Yushan (see fig.\ 2) (\cites[p.\ 28]{Schoppa2011SeaOfBitterness}{Carter2023ComplexLegacyDoolittleRaid}{Zhao2016LifetimeOfSufferingChinaDaily}).

Aside from destroying the intended airbases, the Japanese also looted and razed towns and cities they happened across \autocite[chap.\ 22]{Scott2015TargetTokyo}. “They shot any man, woman, child, cow, hog, or just about anything that moved,” As Father Wendelin Dunker, a priest residing in Ihwang described of the town, “They raped any woman from the ages of 10-65” \autocite[p.\ 381]{Scott2015TargetTokyo}. The city of Nancheng was subjected to similar travesties, as over 800 women were rounded up and regularly assaulted in a storehouse outside of the city (\cites[p.\ 382]{Scott2018MilitaryTimesRipples}). In Quzhou, the Japanese began a campaign of what can only be described as deliberate genocide, killing around 10,000 and displacing a further 30,000 \autocite[p.\ 385]{Scott2015TargetTokyo}. As Father Bill Stein, another priest, described of the town of Yushan, “Now you can walk thru street after street seeing nothing but ruins, … In some places you can go several miles without seeing a house that was not burnt…” \autocite[p.\ 385]{Scott2015TargetTokyo}. The town, which had a population of 70,000, saw over 80\% of its homes destroyed \autocite[p.\ 166]{Yamamoto2000Nanking}. Furthermore, not only was the brutalization being carried out on communal levels, but atrocities of similar degrees were also likewise committed on individual scales.

Again, examples of such terrorization abound, including instances of elderly individuals being pushed off bridges in Yujiang (Yukiang) and being used as target practice; those who were not shot, drowned \autocite[p.\ 382]{Scott2015TargetTokyo}. In Linchuan (Linchwan), corpses of families were tossed into wells, contaminating the local water supply \autocite[p.\ 383]{Scott2015TargetTokyo}. There were even reports of soldiers unearthing, looting, then destroying graves in Yintang \autocite[p.\ 383]{Scott2015TargetTokyo}. The Japanese troops made the utter ruination of all that they came upon a primary objective, demonstrating that their true intent during this campaign far exceeded the supposed goal of national security.

Punishments were especially severe for those discovered to have aided Americans. There were accounts of individuals coerced into a “bullet contest”, where a group of ten individuals was forced into a line, then penetrated by a single bullet \autocite[p.\ 384]{Scott2015TargetTokyo}. Some were burned alive, and others beheaded (\cites[p.\ 384]{Scott2015TargetTokyo}{ChinaDaily2015VillagersHelpedSavePilots}). In a cruel twist of fate, it was often the little trinkets and items left as a gift of gratitude that led to the identification of those who assisted the Americans \autocite{ChinaDaily2015VillagersHelpedSavePilots}. Aside from utilizing the traditional methods of destruction, Japanese forces also deployed biological weapons to such an extent that it has been cited as “the most significant BW [Biological Warfare] campaign ever conducted” \autocite[p.\ 19]{Carus2017ShortHistoryBioWarfare}.

Unit 731 utilized vaporized versions of infections, spreading them in wells and rice fields, even leaving contaminated foodstuffs for the inhabitants to find (\cites[p.\ 138]{Tanaka1996HiddenHorrors}[p.\ 19]{ChevrierEtAl2006BTWCProceedings}[p.\ 18]{Carus2017ShortHistoryBioWarfare}). During the outset of the offensive, special aircraft released \emph{Y. pestis}, which causes plague, around the area of the Zhejiang-Jiangxi railroad \autocite[p.\ 18]{Carus2017ShortHistoryBioWarfare}. It is reported that the Chinese army suffered casualties of up to 30,000, and among the Japanese themselves, over 10,000 were infected and around 1,700 died of their infections (\cites[p.\ 427]{Tohmatsu2010StrategicCorrelation}). As General Shiro Ishii, microbiologist and overseer of Unit 731, stated bluntly, “The bacteria weapons used in Zhejiang-Jiangxi War Zone were very effective, causing several fierce epidemics” (\cites{PacificAtrocities2024BacteriologicalWarfareChina}{PBS2024ShiroIshii}). Needless to say, the devastation brought about by such vicious weapons was ineffable.

Having successfully captured and destroyed the enemy airfields, the Japanese began their withdrawal on August 15th, while being counterattacked by the Chinese army \autocite[p.\ 12]{Sherry1996ChinaDefensive}. In a last-ditch attempt to cause as much havoc as possible, the Japanese released 6,000 prisoners infected with typhoid and paratyphoid back into Chinese territory (\cites[p.\ 18]{Carus2017ShortHistoryBioWarfare}[pp.\ 160, 169]{Yin2002RapeOfBiologicalWarfare}). By late September, they had ceded back nearly the entire zone-of-occupation and left behind a trail of destruction \autocite[p.\ 2250]{XiongZhou2000GreatDictionaryMilitary}. As Chiang Kai-shek described in a telegram to Washington, “[t]he Japanese slaughtered every man, woman, and child in these areas—let me repeat, every man, woman, and child” \autocite{Scott2015SmithsonianUntoldStory}.

In retrospect, while the morale-lifting effects of the Doolittle raid in the US should not be disregarded, we should certainly place such benefits alongside its costs. We should realize that as a direct result of the raid, Japanese retaliation in cities like Quzhou and Nancheng was marked by countless atrocities against civilians. This serves as a stark reminder of the devastating human costs incurred. 

The response of contemporaneous American news outlets, however, was relatively muted and vague. The New York Times reported, “At the moment the Japanese are concentrating the bulk of their offensive power against China… But it may have been General Doolittle's air raid that precipitated it” \autocite[p.\ 92]{NYT1942ChineseCampaign}. The Los Angeles Times provided a comparatively more forceful, though ambiguous response, “To say that these slayings were motivated by cowardice as well as savagery is to say the obvious. The Nippon war lords have thus proved themselves to be made of the basest metal…” \autocite[p.\ 390]{Scott2015TargetTokyo}. However, these news reports failed to leave a tangible impact in Western historiography and was relegated to the dustbin of history \autocite{Scott2015SmithsonianUntoldStory}. Even today, sources remain scant regarding the ramifications of the Doolittle raid, mostly focusing on the positive outcomes of the attack (\cites{Mitter2013ForgottenAlly}{MeltzerMensch2023NaziConspiracy}).

The Doolittle raid forces us to confront the duality of history and the moral ambiguity surrounding acts of war. As Walter Cronkite, the famous broadcast journalist, once said, “In seeking truth, you have to get both sides of a story” \autocite{Felling2007CronkiteHoffa}.

It’s time we began to look at the other side of the Doolittle raid – its cruel and tragic reprisal.

\newpage

\section*{Appendix}



\begin{figure}[h!]
	\centering
	\includegraphics[width=0.7\textwidth]{images/fighters.jpg}
	\label{figure:figure1}	
	\caption{\emph{Chinese Rescue Crashed Shangri-La Bombers.} April 18, 1942. Photo, 17.9 x 22.9 cm. \url{https://www.gilderlehrman.org/collection/glc0955311}}
\end{figure}

\begin{figure}[h!]
	\centering
	\includegraphics[width=0.7\textwidth]{images/map.jpg}
	\label{figure:figure2}	
	\caption{Map of areas occupied by the Imperial Japanese Army during the Zhejiang-Jiangxi campaign. (Map by author)}
\end{figure}

\begin{figure}[h!]
	\centering
	\includegraphics[width=0.7\textwidth]{images/madame.jpg}
	\label{figure:figure2}	
	\caption{\emph{Madame Chiang Kai-Shek (Song Meiling) with Members of the Doolittle Raiders (Left to Right) James Doolittle, John Hilger, and Richard Cole.} April 30, 1942. Photo. \url{https://www.tshaonline.org/handbook/entries/hilger-john-allen-jack}}
\end{figure}





\printbibliography

\end{document}

