%! TEX program = LuaLaTeX 
\documentclass[12pt, a4paper, twoside]{article}
\usepackage{format}
% Do not alter above

% Metadata: put your article information here 
\newcommand{\jtitle}{The Incompatibility of Luck and Knowledge}
\newcommand{\jauthor}{Person D}
\newcommand{\jaffiliation}{some school}

% Editors will change these fields after acceptance 
\newcommand{\jvolume}{3}
\newcommand{\jyear}{2026}
\newcommand{\jdoi}{10.17613/82c3s-nac44}  

% References should be placed in refs.bib and cited with \autocite{<source>}
% Quotations can be placed in quote environments: \begin{quote}<your quote>\end{quote}
% Footnotes can be added with \footnote{<your footnote>}

% Your Content

\begin{document}

\maketitle{}

The definition of knowledge has long been a subject of dispute, with numerous philosophers offering their own interpretations throughout history. One of the most famous theories of knowledge, the JTB or justified true belief theory proposed by Plato and Theaetetus, describes S knowing p if and only if: p is true, S believes p, and S is justified in believing p. The JTB analysis also included the assumptions that justified beliefs could be false and that deductions from p are justified in their belief. This broad theory of knowledge was widely accepted until philosopher Edmund Gettier illustrated how the JTB was not sufficient for knowledge, countering the theory and creating Gettier cases.

Gettier cases became counterexamples used to disprove proposed theories of knowledge. An example of a Gettier case that could be used to disprove the JTB is in the case of a broken clock: person P walks into a room at exactly 3:00 PM, looking at a typically reliable clock reading 3:00. While your reasoning to believe it’s 3:00 PM is perfectly justified, in fact, the clock stopped working exactly 12 hours ago. So by pure luck, all three requirements for knowledge are met, even though the truth was fully accidental. Gettier cases like these typically follow a bad luck plus good luck pattern, or in this case, bad luck being the clock breaking 12 hours ago, followed by good luck of the person checking the clock at exactly 3:00 PM.

The “Safety Theory” informs us that a belief, even a true belief, may not necessarily count as knowledge. This is due to the fact that a certain belief could have been false. Some true beliefs may be accidental. For example, Elizabeth may have been driving through Barn County and seen what she thinks is a barn. She identifies that the structure she saw is a barn. However, the Barn County she’s driving through is filled with false barn facades. In addition,  the one that Elisebeth saw was, in fact, a real barn. Elizabeth’s belief that this is a barn is justified. However, her seeing a real barn in an area with many fake barns is only by chance. She could have easily seen a fake barn and believed it was a real barn. Her belief that she’s looking at a barn is not free from error. For a belief to align with the Safety Theory, it must be a belief that doesn’t have error.

A productive way to evaluate theories of knowledge is to test them against deliberately constructed Gettier counterexamples. Zagzebski provides a general “recipe” for generating such cases, showing that any account lacking an explicit luck component will be vulnerable. The process begins with a belief that clearly satisfies the theory’s conditions for knowledge, then introduces bad luck to disrupt the normal connection between justification and truth. Finally, good luck is added to restore the truth. This combination of bad luck and good luck in Gettier cases blocks the theory of knowledge, resulting in a belief that is true, but accidental.

Zagzebski’s method illustrates why any theory that fails to exclude this “bad luck + good luck” pattern will remain susceptible to counterexamples. The safety theory responds by adding a condition: a belief counts as knowledge only if it remains true across most nearby possible worlds. This condition blocks the final “good luck” step by making the truth stable.




\printbibliography

\end{document}

