%! TEX program = LuaLaTeX 
\documentclass[12pt, a4paper, twoside]{article}
\usepackage{format}
% Do not alter above

% Metadata: put your article information here 
\newcommand{\jtitle}{Artificial Intelligence as GPT: Effect on Economics and Societes}
\newcommand{\jauthor}{Person F}
\newcommand{\jaffiliation}{some school}

% Editors will change these fields after acceptance 
\newcommand{\jvolume}{3}
\newcommand{\jyear}{2026}
\newcommand{\jdoi}{10.17613/82c3s-nac44}  

% References should be placed in refs.bib and cited with \autocite{<source>}
% Quotations can be placed in quote environments: \begin{quote}<your quote>\end{quote}
% Footnotes can be added with \footnote{<your footnote>}

% Your Content

\begin{document}

\maketitle{}

\section{Introduction}

Throughout human history, the most transformative drivers of innovation and development have been general-purpose technologies (GPTs): GPTs such as steam, electricity, and computers have generated rapid economic growth and caused widespread social restructuring \autocite[p.\ 0]{BresnahanTrajtenberg1995GPT}. Over the past decade, many have coined Artificial Intelligence (AI) as the latest GPT. However, I contend that there are key differences between AI and past GPTs. While AI has the same broad applicability and potential for economic impact as previous GPTs, its adoption will be significantly faster and more volatile because of its digital nature, near-zero marginal costs, and wide scope of application. Accordingly, its economic impact will be more rapid, disruptive, and polarizing and will require proactive institutional adaptation to avoid inequality and market concentration.

\section{Context \& Analytical Framework}

First, a brief history of the GPTs considered in this study. One of the first GPTs of the modern era was steam power. Developed in England in the late 18th century, steam initially had a negligible impact on overall productivity and economic growth and was limited to a few industries prior to the 1830s. It was only in the mid-19th century, when high-power steam engines and complementary technologies such as railways and steamships were further improved, that productivity surged. From 1870 to 1910, steam accounted for approximately 29 percent of productivity growth in the UK \autocite[p.\ 524]{Crafts2021AIasGPT}. Electricity followed a similar trend. Despite becoming commercially viable in the late 19th century, the technology did not have a prominent effect on productivity until the 1920s and 1930s, when adoption rates surged to approximately 23 percent of firms in the US, contributing to around 10 percent overall productivity growth per annum \autocite[pp.\ 16--17]{FilippucciGalLaengleSchief2025OECDG7ProductivityGainsAI}. In contrast, the adoption of semiconductor-based computing occurred much faster than that of previous technologies. Within a little more than a decade from its initial commercial viability in the late 1970s and the 1980s, US labor productivity growth rose from approximately 1.6 percent per year (1974–1995) to over 3 percent (1995–2004) \autocite[p.\ 525]{Crafts2021AIasGPT}. Merely ten years after the personal computer’s introduction, approximately 40 percent of US firms were using computers \autocite[p.\ 16]{FilippucciGalLaengleSchief2025OECDG7ProductivityGainsAI}.

We now define some characteristics that all GPTs share. A general-purpose technology is a single foundational innovation that is pervasive, improvable, profitability-enhancing, and broadly spillover-generating \autocite[p.\ 521]{Crafts2021AIasGPT}. Commonly, GPTs follow an S-curve style of adoption: initially, the technology struggles with diffusion because firms are reluctant to upend their existing operations, but as it improves and prior investments mature, adoption accelerates before eventually plateauing as markets reach saturation and most potential adopters have already transitioned (see fig. 1). A key driver of this process is the continual improvement of GPTs over time, which increases the incentive for firms to adopt them \autocite[p.\ 5]{BresnahanTrajtenberg1995GPT}. Along with technological development, GPTs often create complementary innovations. The invention of electricity, for example, led to the creation of light bulbs, batteries, and electric vehicles. In this way, GPTs produce a wealth of entrepreneurial opportunities, in both directly related and adjacent industries.

\begin{figure}[h!]
	\centering
	\includegraphics[width=0.9\textwidth]{images/graph1.jpg}	
	\caption{The S-curve of GPT Adoption. Source: Author's illustration.}
\end{figure}

Similar drawbacks also appear between AI and earlier GPTs. For example, GPTs tend to increase the share of output allocated to capital owners, thereby increasing inequality. This occurred during the steam-driven British Industrial Revolution, when the share of profits in GDP allocated to capital owners rose from 17.2 percent in 1770 to 29.5 percent in 1860 \autocite[pp.\ 526--528]{Crafts2021AIasGPT}. Additionally, GPT development can occur unevenly across the world, producing long-lasting disparities: in 2023, 95 percent of global AI research was carried out by the US, excluding China \autocite[pp.\ 18--19]{David1989ComputerAndDynamo}.

Having surveyed the similarities, let us elucidate the differences between GPTs, which we can measure using the following metrics: adoption speed and diffusion channels, complementary assets and infrastructure, impact on productivity, impact on labor, impact on market structure, and impact on regulation and society.

\section{Comparative Analysis}

\subsection{Adoption Speed}

One of the most apparent differences between AI and previous GPTs is their adoption speed. Steam and electricity require capital-heavy investments. Steam, in its most common forms—the steamboat and steam train—entails high fixed costs and significant physical infrastructure. Likewise, electricity requires the construction of costly power plants and transmission grids. The adoption of computers necessitated lower initial costs, although their adoption still necessitated major changes to organizational processes, software systems, and data architecture. In contrast, the adoption of AI is much faster. From 2024 to 2025, AI adoption in the US rose from 4 percent to 6 percent; if this trend continues, AI is predicted to reach approximately 50–60 percent adoption within the decade \autocite[p.\ 17]{FilippucciGalLaengleSchief2025OECDG7ProductivityGainsAI}.

Several factors explain this rapid adoption rate. First, distribution is digital and incurs low marginal cost: firms can call APIs or deploy managed services without owning costly infrastructure (\cites[p.\ 9]{ErdilBesiroglu2023ExplosiveGrowthAIAutomation}[p.\ 21]{BrynjolfssonRockSyverson2017AIModernProductivityParadox}). In the past two years, the cost of quality-adjusted AI models has declined by over 80 percent, enabling firms to access cutting-edge technologies at only a fraction of their previous expense \autocite[p.\ 8]{AndreBetinGalPeltier2025OECDIndicatorsAIMarkets´}. Second, open-source frameworks further lower entry barriers, widening participation beyond that of large incumbents. This counteracts the expected concentration of power in the hands of the first innovators, which was typical of earlier GPTs. Third, AI’s learning curve is comparatively shallow: natural-language interfaces allow non-experts execute complex tasks with little to no training, which generates especially large gains for less-experienced \autocite[p.\ 17]{FilippucciGalLaengleSchief2025OECDG7ProductivityGainsAI} users. This is not to say that AI has no structural requirements; rather, AI’s constraints differ from those of earlier GPTs. Instead of extensive hardware, for AI, diffusion depends on intangible complements such as data governance, data quality, model deployment, and systems monitoring \autocite[pp.\ 4--10]{BrynjolfssonRockSyverson2017AIModernProductivityParadox}. Consequently, as with prior GPTs, the technology is still likely to exhibit a productivity J-curve: early investments in complements and learning depress measured gains before benefits arrive, but the trough should be shallower and shorter because less investment in physical infrastructure is required (see fig. 2) \autocite{BrynjolfssonRockSyverson2018ProductivityJCurve}.

\begin{figure}[h!]
	\centering
	\includegraphics[width=0.9\textwidth]{images/graph2.jpg}	
	\caption{Graph comparing the J-curve of GPTs, 1761–2025. Source: Author’s calculations using Bank of England, TFPGUKA (via FRED, Federal Reserve Bank of St. Louis); U.S. Bureau of Labor Statistics, OPHNFB (via FRED); and Gene Smiley, “The U.S. Economy in the 1920s,” EH.Net Encyclopedia.}
\end{figure}

\subsection{Economic Impacts}

A point of similarity between AI and past GPTs is their impact on labor. Both steam and electricity and their complementary tools displaced manual labor while expanding industrial employment (\cites[p.\ 155]{AgrawalGansGoldfarb2023LessAutomation}{Soroushian2024PastWavesAutomationTeachAI}) Thus, labor productivity rose with adoption, leading to rising wages over time \autocite[pp.\ 14--15]{David1989ComputerAndDynamo}. Computers automated routine cognitive tasks, such as bookkeeping, data entry, and basic automate an even larger share of tasks as it can perform non-routine cognitive functions \autocite[p.\ 156]{AgrawalGansGoldfarb2023LessAutomation}. However, although it is likely to cause short-term job displacement, permanent job loss is improbable. Historically, GPTs have primarily caused reallocation across sectors rather than outright unemployment. Furthermore, like past GPTs, AI will likely create new tasks, most likely related to prompting, oversight, evaluation, and data stewardship \autocite[p.\ 526]{Crafts2021AIasGPT}. What is certain is that AI will raise individual productivity, with generative models producing a 30–60 percent productivity gain depending on the task \autocite[p.\ 9]{FilippucciGalLaengleSchief2025OECDG7ProductivityGainsAI}. For an individual worker, there are two possible non-exclusive outcomes. First, owing to AI’s shallow learning curve, labor productivity may be boosted broadly, reducing inequality \autocite[p.\ 155]{AgrawalGansGoldfarb2023LessAutomation}. Second, because of its skill-amplifying nature, already productive workers and firms may become even more productive, thereby widening the skill gap \autocite[pp.\ 8--9]{BrynjolfssonRockSyverson2017AIModernProductivityParadox}. Which of the two effects takes hold will depend on institutional factors, such as the efficacy of education systems, labor market policies, and access to AI technologies. 

Regulation is a primary influence on the pace and impact of AI development. In previous GPTs, safety and competition regulations evolved alongside the technology. In the UK, the Factory Acts were passed to protect workers using steam engines in factories; likewise, in the US, antitrust statutes prevented electricity monopolies from stifling competition \autocite{UKParliamentFactoryAct1833}. AI is likely to be subject to similar regulations, perhaps to an even greater degree \autocite[pp.\ 9--10]{ErdilBesiroglu2023ExplosiveGrowthAIAutomation}. Regulation’s effect is nuanced: clear, risk-based rules can accelerate adoption by building public and organizational trust, while vague or heavy ex-ante requirements can slow adoption rates and entrench incumbents. The economic trajectory of AI thus hinges on regulatory design and the balance of productivity benefits against the risks of polarization and market concentration.

\section{Conclusion}

AI shares many similarities with previous GPTs. For instance, it can reorganize economics and catalyze waves of complementary innovation while maintaining the fundamental structure of the labor market. Simultaneously, its ease of distribution makes its adoption swifter and more unpredictable. Such speed magnifies both the potential gains in terms of broad productivity gains, new entrepreneurial opportunities, and greater accessibility, and the potential risks, including inequality, market concentration, and geopolitical asymmetries. History shows that GPTs rarely deliver equitable progress without sound policy intervention. Therefore, we must focus on designing effective and equitable regulations, both on a national and international scale, to ensure that AI’s benefits are maximized while its downsides are minimized. AI is a powerful tool that can reshape the course of human progress, but only if used correctly.





\printbibliography

\end{document}

