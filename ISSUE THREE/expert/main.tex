\documentclass[12pt, a4paper, twoside]{article}
\usepackage{format}
% Do not alter above

% Metadata: put your article information here 
\newcommand{\jtitle}{Skepticism about Expert Identification}
\newcommand{\jauthor}{Person A}
\newcommand{\jaffiliation}{some school}

% Editors will change these fields after acceptance 
\newcommand{\jvolume}{3}
\newcommand{\jyear}{2025}
\newcommand{\jdoi}{10.17613/82c3s-nac44}  

% References should be placed in refs.bib and cited with \autocite{<source>}
% Quotations can be placed in quote environments: \begin{quote}<your quote>\end{quote}
% Footnotes can be added with \footnote{<your footnote>}

% Your Content

\begin{document}

\maketitle{}

The dilemma of identifying expertise has long faced skepticism. In such situations, an epistemic novice–someone with limited knowledge and skills in a certain domain–struggles to verify the expertise of an expert who has access to superior knowledge and skills. This expert, an individual with formal or informal training who possesses competence beyond that of extensive experience, specializes in a certain domain, having epistemic authority over a wide range of propositions and skills.

The argument about skepticism regarding expert identification is strong at first glance. There is real epistemic difficulty since a novice lacks the tools needed to evaluate whether someone is actually an expert. This creates a problem because one cannot assess expertise unless one already possesses it. However, upon further review, the argument fails because it relies on a strict standard that is unreasonable since it undermines the manner in which a novice can responsibly identify an expert through both social and second-order means.

The premise that appears problematic is the second one, stating that “novices can justifiably identify someone as an expert only if they can assess whether the person is sufficiently well placed in the relevant domain” \autocite[p.\ 178]{watson2022fixing}, which is known as the easy recognition problem (ERP). It becomes a problem of circularity because to spot an expert, one must already be an expert. This approach is unappealing because there is no clear way to directly identify an expert.

In real-life situations, it is common to see a novice rely on second-order indicators of expertise. According to Watson, second-order indicators include credentials, institutional affiliation, peer endorsement, and track records \autocite[p.\ 179]{watson2022fixing}. Unfortunately, despite having some benefits, they are ultimately useless. This is because novices are typically not in a position to assess domain-specific material, even though they can evaluate simple social signals. Goldman supports Watson in Chapter 7 by identifying the concept of reliability, which explains that justified belief through a reliable process is not enough to identify an expert.

For instance, during the COVID-19 pandemic, many individuals claimed expertise on public health via social media, displaying titles like “Doctor” or “Epidemiologist” without proper credentials. Novices, overwhelmed by contradictory voices, often turned to institutional affiliations or publication records to make judgments. While these second-order indicators were sometimes misleading, they remained essential tools in distinguishing between pseudoscience and legitimate expertise. As Alvin Goldman notes, what matters is whether the process of belief-formation is generally reliable, not whether the believer possesses full understanding.

The skeptic’s second rejection of second-order markers collapses under its own premise. Watson argues about the wicked domain, which “are those in which it is difficult to obtain reliable feedback about performance, and where this makes it hard to tell who the genuine experts are” \autocite[p.\ 210]{watson2022fixing}. Without the immediate and reliable feedback of a “kind domain”, wicked domains are liable to manipulation, the novices being at the mercy of possibly biased epistemic authority. This is typically seen in blindly trusting novices being deceived by credentials or uninterpretable, misleading information.

A useful contrast lies between domains like chess and contemporary art. In chess, expertise is evident through repeated performance and clear win-loss outcomes, which is a “kind” domain. In contrast, contemporary art criticism is a “wicked” domain, where feedback is ambiguous, success is interpretive, and consensus among experts is rare. In such cases, novices struggle even more to evaluate who possesses genuine authority.

It is important to acknowledge that knowledge is socially distributed. There is a view of epistemic dependence where people must trust each other in order to know anything. Watson explains, “trust in experts is not epistemically optional” \autocite[p.\ 139]{watson2022fixing}. When expertise is present, novices can learn enough to become experts too. A novice can't identify an expert without sufficient knowledge, and that is why this argument collapses on itself.

One might object that relying heavily on second-order indicators opens the door to manipulation. Institutions may become gatekeepers of expertise, enabling those with power or prestige to be misidentified as experts. For example, in high-profile academic fraud cases, individuals with impressive credentials published fabricated results for years before detection. If novices cannot assess domain-specific content, how can they guard against such failures?

While this concern is valid, it does not follow that we should discard second-order mechanisms entirely. Watson emphasizes that these mechanisms function not in isolation but within social networks that allow for correction, such as peer review, public scrutiny, and whistleblowing. In fact, many cases of fraud were ultimately exposed through such systems, not through direct domain expertise. Therefore, instead of seeking infallibility, novices can adopt a policy of defeasible trust—trust that is open to revision as more evidence becomes available.

This problem instead leads to epistemic paralysis. If a novice cannot truly identify an expert until they themselves are an expert, then nobody can begin their process of learning. Watson states the circular problem presented blocks “justified belief in any expert testimony” \autocite[p.\ 138]{watson2022fixing}. For real-life applications, this argument does not work because there are complex domains where expertise is relied on.

The qualifications to identify an expert are too rigid to be defensible. It assumes epistemic autonomy that does not consider real-life situations. People form justified beliefs in different ways. A novice seeks an expert to learn from. They trust the information shared by the expert because of factors like trust in their credentials and overall credibility. Justified beliefs are rational because they stem from some background knowledge that leads to their desire to graduate from novice to expert. This rational trust, informed by proper second-order evidence, dismantles the skeptical argument of novices recognizing that experts are impossible.

While the skeptical argument highlights a disconnect in novice-expert relations, it ultimately fails. The second proposition illogically demands that novices must directly assess expert placement is unreasonably strict and leads to epistemic paralysis. The proposed solution, second-order markers, while imperfect, are indispensable tools for navigating expertise in real life. The fact that these markers are unreliable, particularly in wicked domains, only strengthens the need for trust and socially distributed knowledge. Since knowledge acquisition depends on these social mechanisms, the skeptical standard collapses under its own premise.

\nocite{*}
\printbibliography

\end{document}

