\documentclass[12pt, a4paper, twoside]{article}
\usepackage{format}
\usepackage{tikz}
\usepackage{pgfplots}
\pgfplotsset{compat=1.18} 
% Do not alter above

% Metadata: put your article information here 
\newcommand{\jtitle}{The Asymmetry of Moral Obligations}
\newcommand{\jauthor}{Sam Cao}
\newcommand{\jaffiliation}{Miramonte High School}

% Editors will change these fields after acceptance 
\newcommand{\jvolume}{2}
\newcommand{\jyear}{2025}
\newcommand{\jdoi}{10.17613/n9xdf-8dh92}  

% References should be placed in refs.bib and cited with \autocite{<source>}
% Quotations can be placed in quote environments: \begin{quote}<your quote>\end{quote}
% Footnotes can be added with \footnote{<your footnote>}

% Your Content

\begin{document}

\maketitle{}

\section{Introduction}

This paper examines our moral obligations to current and future generations, as well as the accompanying policy implications. Although many analyses attempt to resolve the problem by outlining specific moral obligations, I argue that such an approach is ineffective. Moral disagreement pervades today’s cosmopolitan societies, demanding that we address such disagreements pragmatically. Thus, a deeper question emerges: How can we manage division regarding moral obligations? Regarding this concern, I contend that while we might ground obligations to the living primarily in traditional morality, obligations to future people would depend exclusively on instrumental reasoning.

\section{Justifying a Meta-Ethical Approach}

Putatively, a moral obligation is defined as a demand that an agent ought to fulfill, such that failing to do so constitutes wrongdoing (\cite{pick2004} pp.\ 26–27).

Although this definition implies that moral obligations are broadly recognized, this is farfrom the truth; agents frequently contest even fundamental principles such as justice, fairness,and impartiality (\cite{atari2020}). Many national and international concerns, such as climate change, resource overconsumption, and national debt, expose deep moral disagreements (\cite{hasan2023}). This paper specifically
focuses on climate change. Faced with the issue of greenhouse gas emissions, developed and
developing nations offer contradictory judgments: developed economies advocate for universal emissions reduction requirements, asserting that climate change cannot be contained if any parties are exempt, whereas developing nations assign such burdens exclusively to developed nations, citing universal emissions cuts as unjust barriers to industrialization (\cite{belfiori2024}). Due to the absence of a common moral foundation, one’s view of morality may seem arbitrary or even repugnant to others. Additionally, we do not know how future values might differ from contemporary values,
undermining the relevance of fixed moral rules grounded in the former (\cite{macaskill2019}). Thus, a universally agreed-upon understanding of moral obligation cannot exist.

The priority then shifts from laying down fixed moral precepts to fostering cooperation both among current agents and between current and future agents.\footnote{For varied arguments against a prescriptivist view of morality, see Francesco Allegri, \emph{Conflicting Values and Moral Pluralism in Normative Ethics}; David B. Wong, \emph{Natural Moralities: A Defense of Pluralistic Relativism}; and Jonathan Dancy, \emph{Ethics Without Principles}.} Contractarianism, a subset of social contract theory, suits this approach. Beyond its grounding in mutual benefit and respect for individual autonomy present in all social contract theories, it also features a motivationally internalist structure and requires only that agents possess basic instrumental rationality.\footnote{For more general arguments vis-à-vis a moral social contract approach, see Jan Narveson, \emph{Social Contract: The Last Word in Moral Theories}; Kelly James Clark, \emph{Why Be Moral? Social Contract Theory Versus Kantian-Christian Morality}; Susan Dimock, \emph{Two Vritues of Contractarianism}; Rucha Kulkarni, \emph{A Critical Analysis of Moral Contractarianism: Towards a Revised Framework}.} Despite these advantages, traditional contractarianism confronts three key challenges: establishing trust among agents in the absence of preexisting norms (the assurance problem); maintaining compliance among self-interested agents (the compliance problem); and grounding moral agreements in non-moral reasons (the “wrong kinds of reasons” objection, which argues that actions motivated by non-moral considerations generate no genuine moral duty) (\cites[pp.\ 3–4]{kogelmann2019}[pp.\ 347]{sayremccord2013}[pp.\ 15–16]{moehler2020contractarianism}{prichard1912}). To address these concerns, this paper adopts a further evolution of contractarianism: Michael Moehler’s \emph{multilevel social contract theory.}

\section{Minimal Morality}
\label{sec:minimal_morality}

Multilevel social contract theory introduces a bifurcation of morality into \emph{first} and \emph{second}-\emph{level} morality. First-level morality, also known as traditional morality, arises from informal, often intuitive norms shaped by the existing conditions of a society. First-level morality rests on two assumptions: first, that agents possess commitment power, whereby established norms are upheld even when defection offers short-term gain; and second, that agents are not bound by prescriptive beliefs, whereby norms already shared are maintained (\cites[pp.\ 5–6]{moehler2020contractarianism}[p.\ 229]{moehler2024diversity}). One way to capture this level is through David Hume’s moral conventionalism, which holds that although agents are self-interested, they are also “morally sensible,” adhering to the two assumptions outlined above (\cites{hume1998}[pp.\ 232-233]{moehler2024diversity}). In effect, first-level morality offers a putative account of moral norms.

In contrast, second-level morality relies on self-interested bargaining conducted by rational agents, whom Moehler refers to as \emph{homo prudens}. These agents:
\begin{enumerate}

\item value long-term cooperative gains, and

\item base decisions on instrumental rationality and empirical assumptions regarding social cooperation (\cites[p.\ 8, 48]{moehler2020contractarianism}[p.\ 234]{moehler2024diversity}).

\end{enumerate}
From (i) and (ii), \emph{homo prudens} derive the weak principle of universalization (WPU), a key bargaining constraint that guides action in second-order morality (\cite[pp.\ 48–52]{moehler2020contractarianism}). According to the WPU, agents act upon the likely outcomes of hypothetical negotiations, treating others according to their basic needs and beyond that threshold, according to their relative hypothetical bargaining power (\cite[p.\ 50]{moehler2020contractarianism}). As such, the WPU both facilitates stable cooperation on the basis of non-moral values
and ensures the survival of its contracting agents (\cite[p. 50]{moehler2020contractarianism}).

A crucial structural element of multilevel social contract theory is that, as long as agents share the non-moral goal of long-term peaceful cooperation, moral contracts can be formulated to address increasingly localized disagreements (\cite[p.\ 54]{moehler2020contractarianism}). In other words, \emph{n}th levels of morality can emerge beyond second-level morality, all based on hypothetical bargaining premised on the WPU. These successive layers of moral contracts follow the principle of subsidiarity: problems should be addressed at the most local level (\cite{merriamwebster2025}). In practice, this means that agents defer to higher-level morality only when there is moral disagreement at a lower level (\cite[p.\ 235]{moehler2024diversity}). As an \emph{ultima ratio}, agents who break the contract are excluded entirely (\cite[p.\ 58]{moehler2020contractarianism}). The successive layers of morality effectively ensure the resolution of moral conflicts without compromising the overarching cooperative framework.

In addition, these moral layers supply substantive solutions to traditional contractarianism’s objections. First, the assurance problem is resolved through first-level morality, which presupposes existing moral norms, thereby bypassing the need to cultivate trust among agents who lack prior moral consensus. Second, the compliance problem is addressed by the principle of subsidiarity, as aligning moral rules closely with prudential interests deters agents from defecting. Third, the “wrong kinds of reasons” objection is mitigated by recognizing that second-level morality is not meant to replicate first-level norms; rather, we should in fact expect a divergence in the way morality is justified, as agents invoke second-level morality only when first-level morality proves insufficient.

Finally, a further concern often raised about a multilevel approach is the potential for entrenched power asymmetries: if power differentials expand to such a degree that those in power can be certain that they will remain powerful, such agents will not act in accordance with the WPU. To address this, Moehler proposes the \emph{veil of uncertainty}, arguing that even powerful agents act prudently as shifts in power differentials are unpredictable (\cite[p.\ 144]{moehler2018minimal}). However, as Brian Kogelmann notes, cases such as the Greek and American Constitutions challenge the idea that power asymmetries can be neutralized by uncertainty: these constitutions illustrate the writers’ favoritism toward their own, particularly economic, interests (\cite[p.\ 5–6]{kogelmann2019}) That being said, at its core, the objection regarding the framers of constitutions acting in self-interest misrepresents what bargaining under the WPU entails. Although this may conflict with first-level moral sensibilities, agents favoring their own interests are a predictable feature of second-level morality. Furthermore, although power disparities may favor the framers, these constitutions still safeguard every stakeholder’s basic needs; hence, they remain consistent with the WPU and second-level morality.\footnote{Both the Greek and American constitutions explicitly outline protections for basic civil and economic liberties, with subsequent amendments and evolving jurisprudence producing a robust system that safeguards citizens' fundamental needs.}

\section{Applying Multilevel Social Contract Theory}

Now, while applying the theory to obligations owed to contemporaries is straightforward, extending it to future generations is far more challenging. Without consensus regarding such moral obligations (see §3), we cannot ground duties to future people in traditional norms. Therefore, we must reach an agreement based on instrumental rationality alone.

However, in doing so, we encounter a major dilemma: how can we negotiate with individuals who do not yet exist? Ostensibly, future individuals cannot exercise any form of bargaining power, making bilateral negotiation impossible (\cite[p.\ 17]{kulkarni2024}). However, I contend that future people do possess bargaining leverage. To demonstrate this, I draw upon Fausto Corvino’s \emph{economic-synchronic model of direct intergenerational reciprocity} (\cite[p.\ 398]{corvino2023}). Corvino identifies four ways in which unborn people (UBs) determine the well-being of now-living people (NLs):
\begin{enumerate}

	\item Corporations invest primarily based on the expectation of future revenue from UBs; without UBs, incentives for long-term investments are nullified, destabilizing NLs’ economy.

	\item Value investors regularly fund startups, assuming future demand by UBs; without UBs, incentives to fund innovation, and thus, opportunities for NLs vanish.

	\item Institutions commonly issue debt relying on UBs to repay it; without UBs, many initiatives would face severe funding shortfalls and be unlikely to proceed to the detriment of NLs.

	\item Many infrastructure and welfare programs are justified by their utility to UBs; without UBs, many such initiatives, which often benefits NLs, would lack justification (\cite[pp.\ 400-402]{corvino2023}).

\end{enumerate}

This economic interdependence between UBs and NLs ultimately yields a defensible framework for establishing a moral contract with future people, as now-living \emph{homo prudens} must consider UBs’ interests. Corvino refers to the resulting arrangement as \emph{transgenerational sufficiency}, an arrangement between NLs and UBs that optimizes both parties’ gain: for NLs, this means attempting to transfer as little as possible to UBs while extracting optimal benefit; for UBs, this means maximizing the amount that they receive from NLs (\cite[p.\ 404]{corvino2023}).

Therefore, although traditional norms can’t justify such duties to future people, second- level morality, guided by the economic-synchronic model, allows us to formulate a bargaining situation in which \emph{homo prudens} hold moral obligations toward future people.

\section{Implications for Policy Making}

With a multilevel framework for intergenerational obligations in place, we now turn to climate change. By analyzing the problem through a multilevel framework, we find that since no shared moral baseline exists, second-level morality is better suited to addressing climate change than first-level morality.

To illustrate the contrast between these two approaches, consider two major climate agreements: the Kyoto Protocol and the Paris Agreement.

The Kyoto Protocol of 1997 constituted the first globally coordinated effort to reduce emissions. It introduced a legally binding framework for emissions reduction that institutionalized the distinction between nations based on their economic development (\cite[p.\ 24]{maslin-nodate}) Assuming that developed nations were exclusively responsible, the treaty bound only 37 nations to emissions targets, while major emitters, including China and India, were excluded because they were classified as developing economies (\cite{unfccc2025kyoto}). Naturally, this engendered conflict between developed and developing countries. In fact, the world’s largest emitter—the United States— never ratified the treaty, citing its unfairness toward developed nations as justification for doing so (\cite{mignone2007}). Ultimately, the Kyoto Protocol’s rigid bifurcation based on economic development compromised its legitimacy and long-term viability, resulting in weak global compliance. These outcomes suggest that frameworks premised on prescriptive morality fail to support effective cooperation.

In 2015, the Paris Agreement was adopted to remedy the shortcomings of the Kyoto Protocol. In contrast to the Kyoto model, the Paris Agreement’s flexible architecture accommodates the varying capacities of individual states and ensures compliance among signatories (\cite[p.\ 150]{moehler2020climate}). It also allows local leaders to participate, amplifying the voices of groups with nuanced self-interests irrespective of national political concerns (\cites{broto2022}[p.\ 67]{skjaersethetal2021}[p.\ 151]{moehler2020climate}). Additionally, the Agreement requires that all signatories contribute to reducing emissions, Additionally, the Agreement requires that all signatories contribute to reducing emissions, such that at every subsequent conference, signatories must commit to more ambitious emissions-reduction targets, although emissions goals are not legally binding. The Paris framework rather entrusts each country with the authority to set and enforce its own emissions targets through a system of Nationally Determined Contributions (NDCs)(\cite[pp.\ 9–10]{maslin-nodate}). Therefore, participants can generally integrate their unique obligations within a broader vision of long-term cooperation and shared interests. Today, 194 parties have ratified the Paris Agreement, accounting for roughly 95 percent of global greenhouse-gas emissions, and 153 of them have already submitted at least one strengthened NDC (\cite{unfccc2024ndc}). Overall, the Agreement promotes emissions reduction through long-term cooperation, with its structure suggesting the utilization of instrumental moral reasoning (\cites[pp.\ 5–6]{maslin-nodate}[p.\ 140]{moehler2020climate}{unitednationsclimate2025}). Its decentralized, bottom-up structure demonstrates how moral obligations may arise through rational, interest- based negotiations among diverse agents (\cite{skjaersethetal2021}). Evidently, the Paris Agreement outperforms the Kyoto Protocol in achieving greater participation and producing actionable targets (\cites{zeyrek2021}{tachibana2017}{green2024}).

Key parallels exist between the Paris Agreement and multilevel social contract theory. First, the Paris Agreement accepts the possibility of moral pluralism among its signatories. Second, the Agreement articulates the goal of long-term peaceful cooperation as its underlying rationale.\footnote{This contrasts with the Kyoto Protocol, which prioritizes adherence with certain moral preferences over long-term peaceful cooperation.} Third, the Agreement operationalizes the WPU by mandating that every state formulate and successively tighten emissions pledges it might reasonably expect from a peer with like capacity, thereby aligning parties with the homo prudens model. Given the moral fragmentation and limited compliance reminiscent of first-level morality in the Kyoto Protocol, the Paris Agreement’s flexible and pragmatic design better addresses global climate cooperation.\footnote{During the drafting process of this paper, the United States under the new Trump administration withdrew from the Paris Agreement. However, this does not undermine the efficaciousness of the Paris Agreement. Rather, since the United States under the new administration has changed its focus from one of long-term peaceful cooperation to what it terms “America First.” As such, its withdrawal merely indicates that the United States, at least under this administration, has chosen to step outside the cooperative stance demanded by the WPU. It does not reveal any inherent defect in the bargaining framework itself. For more information regarding “America First,” see The White House, “Putting America First In International Environmental Agreements”; Bukhari et al., “America First 2.0: Assessing the Global Implications of Donald Trump's Second Term.}

\section{Conclusion}

Where moral prescriptivism fails, multilevel social contract theory provides a coherent basis for intergenerational moral obligations, avoiding the normative pitfalls and inconsistencies of alternative theories. According to a multilevel approach, although we owe obligations to the living primarily through traditional morality, our moral obligations toward future people rest exclusively upon second-level morality, which we ground in instrumental rationality and prudential reasoning.


\printbibliography

\end{document}