\documentclass[12pt, a4paper, twoside]{article}
\usepackage{format}
\usepackage{tikz}
\usetikzlibrary{intersections, calc}
% Do not alter above

% Metadata: put your article information here 
\newcommand{\jtitle}{A Study of Ticket Scalping}
\newcommand{\jauthor}{Suresh Raina}
\newcommand{\jaffiliation}{Bishop O'Dowd High School}

% Editors will change these fields after acceptance 
\newcommand{\jvolume}{2}
\newcommand{\jyear}{2025}
\newcommand{\jdoi}{10.17613/rzve8-ck797}  

% References should be placed in refs.bib and cited with \autocite{<source>}
% Quotations can be placed in quote environments: \begin{quote}<your quote>\end{quote}
% Footnotes can be added with \footnote{<your footnote>}

% Your Content

\begin{document}

\maketitle{}

Ticket scalping, the practice of purchasing then reselling the just-purchased tickets at an inflated price has recently been a discussion filled with controversy. Many consider this system of buying then selling at an inflated price to be immoral – taking advantage of “true fans” and selling the tickets instead to those who seem to easily be able to afford these outrageous prices. The reality is that ticket scalping is not only economically efficient, but also equitable. By enacting regulations against this practice, we height those negative parts of this practice while doing away with those beneficial parts of this practice. 

Let me first explore some of the benefits of ticket scalping. One of the most obvious advantages of ticket scalping is its ability to move the market closer to allocative efficiency. Put simply, ticket scalping puts tickets in the hands of people who value the ticket the most. Let me illustrate this with an example first without ticket scalping: 

\begin{quote}
Suppose that a wildly popular rock singer, John, is visiting a city, where there are 1000 fans willing to buy tickets and watch the artist perform. However, there are only 100 tickets for \$100 each available for sale due to the limited size of the venue. To simplify the explanation, let us focus our attention on two people: Bob and Alice. Bob is one of John’s biggest fans; he attends all his concerts and listens to his music constantly. Alice, on the other hand, enjoys country music more, although she does enjoy listening to rock occasionally. As a result of the differing levels of enjoyment they gain from attending John’s concert, they value a ticket to John’s concert differently. Bob would be willing to pay \$500 and Alice only \$200. In this case, both Bob and Alice (as well as the 998 other fans all of whom assign a value to the ticket between \$100 and \$500) rush to purchase their tickets. In the end, the tickets are randomly distributed among fans through the highly competitive and indiscriminate online ticketing process. In the end, both Bob and Alice would have an equal chance of getting a ticket.  
\end{quote}

Though they may not seem like a disparaging scenario, let us see what might happen when we introduce the practice of ticket scalping into the system. 

\begin{quote}
This time, the moment the tickets go on sale, they are all immediately snatched up by the ticket scalpers (this is unlikely to occur in real life; however, it serves to strengthen the explanation). The tickets are then resold by the various ticket scalpers at a higher price, decided by competition between the various scalpers. Let’s say that the new price is around \$450. With the new price in place, Alice certainly would not purchase the tickets anymore, though Bob certainly would, their self-assigned value for the concert being lower and higher than the price, respectively.  
\end{quote}

Through this thought experiment, we see that with scalping, tickets would be moved from fans who valued the concert less to the fans who valued the concert more. Better yet, because raising the price reduced the demand, fans are now able to purchase tickets long after they would have sold out on the primary market without ticket scalpers. If the ticket still has not been sold as the event draws near, their price will eventually be lowered, even below the original price, something that the artist would not being willing to do considering do such a thing may compromise their reputation. 

Ticket scalping also has many benefits for the producer of the tickets. One example of such a benefit could occur in a scenario where a fan is uncertain about whether they would be able to attend a concert. In a case with ticket scalping, these fans would purchase more tickets, knowing that they would be able to resell them if they are unable to attend. Ticket scalpers also provide the producer with an “insurance” in the sense that it protects the producer from any potentially negative fluctuations in the ticket market. In 2016, as the Bloomberg states: “Tickets to Desert Trip, the concert featuring music icons such as The Rolling Stones and Bob Dylan, [were] selling for less than half their original price on the secondary market, as speculators appear to have overestimated demand. A three-day pass for the first weekend, Oct. 7 to 9, was selling for as little as \$188 on StubHub, \$167 on Vivid Seats and \$165 on Ticket City — a steep drop from their original price of \$399.” The last major advantage for the producers is that through the secondary market, they gain a better sense of that they should be pricing their own tickets at. According to the New York Times: “Jeffrey Seller, the lead producer of ‘Hamilton,’ told the Times that he arrived at eight hundred and forty-nine dollars by ‘continually monitoring the secondary market and finding out where the average is.’” By giving this information to the producers, it allows them to make more informed decisions about what to price their tickets for. 

There are some legitimate concerns regarding the practice of ticket scalping. First, in some cases where ticket scalping drastically reduces the attendance of an event, it may cause a loss of profits due to the reduction in sales of refreshments and snacks. Other related practices such as what is commonly known as “ice” involve insider trading where scalpers receive a large amount of tickets before the sale of said tickets begins in the market in exchange for a fee. This is and should be illegal in all markets.  

To end our inquiry into the practice of ticket scalping, I what us to think about some potential solutions. First, as many ticket producers have already attempted, the institution of some kind of authentication feature with the tickets (such as including the name of the ticket recipient or having websites automatically flag suspicious behavior). This solution introduces inefficiencies into the system such as misidentifications by bots, new workarounds created by scalpers necessitating constant updates, or simply having to bear the extra cost of verifying the identity of concertgoers at the physical location. Second, although it may seem like a fantastic solution to this problem, banning ticket scalping is in fact quite an unideal solution. In banning ticket scalping, we have not addressed the root issue – that mismatch between supply and demand. The banning of ticket scalping will only drive the market underground and away from trusted institutions, causing an increase in the sale of counterfeit and invalid tickets. Lastly, the most obvious solutions would be to raise the price of tickets, increasing the supply of tickets, or a combination of both. With information about the apt price for a ticket, producers can “safely” raise the price of their tickets without running the risk of endangering their public opinion – Afterall, who would want those greedy ticket scalpers to profit at the expense of the artist?  


\nocite{*}
\printbibliography[title=References]

\end{document}
