\documentclass[12pt, a4paper, openany]{book}
\usepackage{format}

\newcommand{\jissue}{2}
\newcommand{\jyear}{2025}

\begin{document}

\frontmatter

\vspace*{\fill}
\begin{center}
Copyright © 2025 by the authors.

All articles published in Academus are open access under the Creative Commons Attribution-NonCommercial 4.0 International License (CC BY-NC 4.0).

Authors retain copyright and grant the journal a non-exclusive license to publish and distribute their work.
\end{center}

\vspace*{\fill}

\chapter{Editor's Note}

Welcome to the second issue of Academus. This issue circles a common concern: how institutions and ideas shape the lives we actually live. From election rules to classroom pressures, from classic debates in moral philosophy to the early American presidency, our authors test arguments not only for elegance, but for impact. We’re grateful to the writers and reviewers who made this volume possible, and we hope these pages invite you to argue back: with care, with courage, and in writing.

We open with Alexander Lian’s “Can Ranked Choice Voting Deliver Meaningful Electoral Reform in the U.S.?” A set of case studies (Alaska statewide races, San Francisco 2018, Oakland 2010) shows RCV reliably blunting the spoiler effect and nudging campaigns toward broader coalitions—even as it reconfigures, rather than erases, strategic behavior.

Charles Schäfer’s “The American Dream’s Flawed Promise of Education: The Issue of Academic Stress” traces how rising academic competition burdens adolescents—psychologically, physically, and socially—and how supportive school cultures, mental-health resources, and resilience practices can convert pressure into growth. The argument insists that access and ambition mean little without belonging and well-being.

In “Against Moral Intuitionism,” Celine Wang challenges our confidence in “self-evident” moral seemings. Drawing on reliability worries, deep disagreement, and Mackie’s queerness objection, the paper presses readers to ask not just \emph{what} we feel is right, but \emph{why} those feelings should count as knowledge.

Sarah Kim’s “The Political Economy of Farm Subsidies” argues that U.S. agricultural supports are regressive at home and distortive abroad—socializing costs, concentrating benefits, and weakening developing-country advantages—while proposing a freer, fairer alternative that takes both consumers and farmers seriously.

Sam Cao’s “The Asymmetry of Moral Obligations” uses Michael Moehler’s multilevel social-contract framework to separate duties to contemporaries (grounded partly in shared norms) from duties to future people (grounded in prudence and bargaining). The piece contrasts Kyoto’s prescriptivism with Paris’s pragmatic architecture to show how cooperation can scale under moral pluralism.

Jeff Cao’s “Rejection of Cognitivism” (after T.\ Nguyen) reframes trust not as belief but as an \emph{unquestioning attitude} that rationally delegates deliberation under cognitive limits. By disentangling trust from belief, the paper explains how evidentialism can survive those everyday moments when trust and surface evidence collide.

With “Washington’s Neutrality Proclamation and the Birth of Executive Authority,” John Wang revisits 1793. Through language, law, and politics, the essay shows how Washington’s carefully ambiguous “friendly and impartial” stance both expanded the practical reach of the presidency and forged a public-facing model of executive leadership.

We close with Suresh Raina’s “A Study of Ticket Scalping,” which defends resale markets as allocatively efficient and—counter to intuition—often equitable. By walking through simple models and real frictions, the piece invites readers to separate outrage at bots from the economics of getting scarce seats to those who value them most.

Finally, I want to mention that we’re indebted to the teachers, mentors, and reviewers who made this second issue possible, and to the authors who trusted us with their work. If something here makes you want to argue, write it down, and send it in. May these pages invite careful thinking, charitable disagreement, and scholarship that carries beyond these covers.

\medskip
\emph{Veritas · Ratio · Sapientia.}

\medskip

\noindent Sincerely,

 Sam Cao, Editor-in-Chief

\tableofcontents

\mainmatter

\begin{refsection}[refs/rcv]
\input{source/rcv}
\nocite{*}
\printbibliography[title={References}, heading=subbibliography]
\end{refsection}

\begin{refsection}[refs/american]

\makechapter{The American Dream's Flawed Promise of Education}{The American Dream’s Flawed Promise of Education:\\ The Issue of Academic Stress}{Charles Schäfer}{Stuyvesant High School}{10.17613/82c3s-nac44}


At the center of the American dream lies the promise of education: the idea that with hard work comes reward. Unfortunately, in such a rapidly shifting society, established standards have quickly begun to shift. The issue of academic stress and its impact on adolescents has become increasingly important in recent years. Increased competition for college and jobs has driven the academic expectations for adolescents higher and higher. High school students take more college-level classes, fill out their resumes with more extracurriculars, and score higher on standardized tests. Unfortunately, as academic demands increase, so does academic-related stress. Academic stress encompasses various stressors, such as intimidating workloads, performance pressure, and even self-inflicted stressors like time management issues. These stressors can have major long and short-term effects on one’s physical and mental health, potentially resulting in sickness for a week or even leading to long-term mental health concerns that extend into college and beyond. Despite its many detrimental effects, in the correct environment, academic stress can be turned into resilience and strength. The importance of the impact of academic stress is creating educational institutions that support students’ personal growth, rather than enforce detrimental stress. Addressing this issue is vital for improving students' academic performance, well-being, and future. While academic stress can have severe short-term consequences and devastating long-term effects, supportive educational environments can alleviate stress and promote resilience. 

Excessive academic stress in challenging environments can cause significant short-term repercussions, including anxiety and depression, feelings of inadequacy, and even physical consequences. Periods of peak stress in academically rigorous environments, particularly distance learning, correlate with increased depression and anxiety. In academically stressful environments, students are typically juggling tests, homework, and high grades. This balancing act puts extreme strain on the overworked student: “Sudden bursts of academic stress significantly undermine emotional health and well-being” (\cite{claney2023}). In addition, stressful environments heavily contribute to increasing stress and its repercussions. One of the most stressful environments for students was the COVID-19 pandemic, where distance learning saw increased depression among college students due to feelings of isolation and academic pressure (\cite{chen2024}). Locked in their rooms for the majority of the day, academic stress only amplified growing feelings of depression. Unfortunately, academic stress during the pandemic was pervasive, with 90 million cases of anxiety and 70 million cases of depression being newly measured (\cite{cordovaolivera2023}). According to a study conducted on 1200+ university students, 35\% reported having experienced depression from academic stress during the pandemic (\cite{cordovaolivera2023}). While stressful environments and moments of maximum stress can cause severe depression and anxiety, many other stressors can cause different psychological challenges. Self-imposed stressors, such as time management and perfectionism, exacerbate feelings of inadequacy and emotional instability. Self-imposed stressors are traits people have that often cause feelings of insecurity. Typically, these stressors are based on students' subjective beliefs about their ability to meet their academic goals, creating continuous pressure and a sense of failure if one’s high expectations are unmet (\cite{cordovaolivera2023}). The most common stressor, poor time management, was identified as the “most significant stressor for university students, leading to feelings of emotional instability” (\cite{cordovaolivera2023}). Other self-imposed stressors, such as maladaptive perfectionism or “hypercritical judgment upon failure to meet extreme goals”, are also associated with negative self-perception (\cite{almroth2019}). These toxic forms of introspection can reinforce feelings of inadequacy, whereas fear of failure can hinder one’s ability to confidently approach schoolwork and life as a whole. Research from the National Longitudinal Survey of Adolescent Health reveals that self-perception of inadequacy contributes heavily to emotional instability and fear of failure (\cite{mcleod2012}). This feedback loop of self-inflicted stressors increasing negative self-perception and emotional instability is exceedingly unhealthy, often manifesting itself in physical effects. Prolonged stress results in physical consequences including sleeplessness, changes in appetite, and a weakened immune system. One of the most common physical effects of stress is sleeplessness. This is caused when acute stress attacks increase heart rate and dilate blood vessels to direct blood to large muscles and the heart, elevating blood pressure in a "fight or flight" response (\cite{sha2023}). These attacks come unpredictably, resulting in sleeplessness as the body attempts to return to normal. Unfortunately, once stressed individuals are asleep, they experience “sleep disturbances”, where stress activates neurons in the hippocampus region causing interruptions or low-quality sleep (\cite{mcleod2012}). Another physical consequence of stress is an altered appetite, causing an increase or decrease in food intake, resulting in unhealthy eating patterns and weight fluctuations (\cite{sha2023}). Finally, extreme stress can lead to prolonged cortisol release, weakening the immune system and making the body more susceptible to illness (\cite{cordovaolivera2023}). While periods of extreme stress can have many short-term repercussions, long-term exposure to stress can have devastating permanent effects.

Chronic academic stress can have the long-term negative effects of creating poor stress management habits, increasing the chances of long-term physical weakness, and fostering permanent emotional and psychological damage. Unresolved stress from a high academic workload in adolescence weakens one’s ability to manage future stressors, leading to unhealthy coping strategies in adulthood. Adolescents under constant academic pressure are less likely to develop effective coping mechanisms, as stress often overrides cognitive strategies for stress management (\cite{chen2024}). Lowered self-worth caused by chronic academic stress can cause individuals to perceive future stressors as insurmountable. Over time, this causes students to develop maladaptive coping mechanisms or self-destructive methods for treating stress (\cite{cordovaolivera2023}). Repeated stress responses during the formative years of adolescence can influence how people respond to such stress in adulthood. Such stress-resolving strategies that carry over from adolescence can include “avoidance, denial, or substance abuse” (\cite{claney2023}). This can have massive effects on one’s professional and personal life, disrupting one’s career and relationships. In addition, extreme bursts of academic stress can activate the HPA axis, an anti-stress bodily system that causes sustained cortisol release (\cite{cordovaolivera2023}). This can impair cognitive and emotional regulation, making future responses to stress extremely ineffective. Similar to stress-coping habits, repeated physiological responses also cause permanent bodily harm. Persistent academic stress can permanently increase cardiovascular, immune, and cognitive weakness. Maintaining cardiovascular health is extremely important in living a healthy life. Unfortunately, long-term stress contributes to persistent increases in heart rate and blood pressure, which elevate the risk of hypertension, heart attack, and even stroke. Chronic stress also creates inflammation in coronary arteries, increasing the risk of cardiovascular disease (\cite{almroth2019}). All of these cardiovascular repercussions are associated with long-term chronic stress. Unfortunately, chronic stress can also damage other parts of the body: chronic stress also impairs communication between the immune system and the aforementioned HPA axis, leading to increased vulnerability to “chronic fatigue, metabolic disorders (e.g., diabetes, obesity), and immune dysfunction” (\cite{sha2023}). Beyond weakened cardiovascular and immune systems, academic stress can severely damage an adolescent’s cognitive abilities. Similar to trauma, chronic stress can negatively impact memory, attention, and cognitive function–all essential for learning and development. Stress also makes individuals more susceptible to mental health disorders like PTSD, Major Depressive Disorder, and Bipolar Disorder (\cite{claney2023}). Yet another burden that stress can have on the body is also seen in the continuous activation of the autonomic nervous system and immune system which can lead to long-term wear-and-tear on the body, causing exhaustion and a weakened physical state (\cite{almroth2019}). It is imperative to recognize stress’s wide-reaching consequences on the body, and how such a small part of one’s life can have devastating effects. Beyond physiological consequences, academic stress can also have deeper ramifications by affecting one’s attitude and drive. Excessive academic stress during adolescence increases the likelihood of persistent emotional and psychological challenges into adulthood including lessened motivation, reduced self-esteem, and reliance on external validation. Repeated academic stress can be extremely demoralizing, sapping the determination that one may have had prior. Chronic academic stress during adolescence is linked to long-term emotional exhaustion, reduced motivation, and an increased likelihood of burnout (\cite{cordovaolivera2023}). Extreme stress often causes academic or workplace struggle; repeated failure is not only demoralizing but also detrimental to one’s outlook on life. This is seen when people underestimate their abilities, perceiving themselves as weaker than their aspirations. Long-term academic stress correlates with an increased risk of self-esteem issues and psychological distress later in life (\cite{almroth2019}). With lowered self-esteem in adolescence, those with chronic stress often foster a dependency on external validation for self-worth (\cite{cordovaolivera2023}). In contexts where academic success is heavily emphasized, students frequently equate their value with external achievements and validation (\cite{almroth2019}). Arbitrary statistics such as GPA and test scores can massively impact a student’s mental health. If unresolved, this trend will continue into adulthood, where individuals struggle with identity and self-worth in a more consequential environment. Academic stress has many extremely negative long and short-term consequences on the body and mind; luckily, academic stress, in tandem with rigorous support, can benefit and strengthen adolescents. 

The extreme consequences of academic stress can be alleviated through rigorous support in providing stress-abating resources, promoting strong feelings of school belonging, and fostering resilience to transform stress into growth. Supportive academic environments provide stress-relieving resources and services such as crisis hotlines and mental health centers. With academic stress being so prevalent among students during the pandemic, many high schools and universities integrated crisis hotlines into their mental health resources. 
Institutions such as Wake Forest University promote hotlines such as The Oregon YouthLine, The Trevor Project, and the National Suicide Prevention Lifeline (\cite{wfu2025}). These hotlines, typically open continuously, provide service to those requiring immediate assistance or those simply needing someone to talk to. Such hotlines are extremely effective, with a 2020 study finding a 43\% decrease in overall caller distress, and a 56\% caller follow-up rate (\cite{boness2021}). Other stress-relieving resources have also found great success in supporting student mental health. Following the pandemic, the National Association of Student Personnel Administrators (NASPA) published research regarding strategies for helping students facing academic stress, citing on-campus counseling and support groups as the most effective methods to counteract academic stress (\cite{wfu2025}). School-provided support groups are beneficial due to their communal element. Interacting with students in similar situations promotes connection associated with “enhanced well-being and reduced stress” (\cite{duan2016}). Beyond school-based mental health centers, academically stressful environments can also alleviate stress by promoting a culture of school connectedness. A strong sense of school belonging correlates with lower symptoms of depression and anxiety, promoting resilience to stress. With lowered self-esteem being a major symptom of academic stress, feeling socially valued in one’s community can combat these negative feelings. This feeling of acceptance can largely reduce anxiety and depression symptoms in adolescents, seeing a marked drop after 12-18 months (\cite{allen2024}). Even long-term effects of chronic stress, including predisposition to mental health disorders, can be mitigated through feelings of school belonging. In a longitudinal study conducted on Australian high school and college students, those studying in socially welcoming environments saw a gradual decline in mental health symptoms over time, whereas those studying in purely stressful environments saw sharp increases in depression and anxiety (\cite{allen2024}). Social factors play a major role in influencing how adolescents approach schoolwork and stress. If supported by one’s peers, students are often pushed to work harder to compete with their peers. This can result in increased academic engagement and, by extension, resilience to stress. A 2022 study conducted on Dutch university students determined that engagement was a key pathway to resilience, with 78\% of engaged students reporting resilience to stress (\cite{versteeg2022}). Supportive environments prevent academic stress from hindering their students and transform experienced stress into strength. Supportive school environments cultivate mental resilience, stress-mitigating skills, and strength. Resilience plays a major role in turning stress into strength. Defined as the “ability to bounce back from adversity”, resilience is critical for adolescents in academically challenging environments (\cite{claney2023}). Supportive academic environments utilize stress management activities that promote resilience, such as mindfulness and physical activity. Mindfulness, the practice of “focusing on the present moment non-judgmentally”, fosters relaxation and a sense of calm (\cite{claney2023}). This can help teens cope with the pressures of school by disconnecting from external stressors to reduce anxiety. Physical activity can similarly act as a stress reliever: producing endorphins to improve mood, allowing the individual to detach themselves from their problems, and improving bodily health. In a study conducted by the American Psychological Association, 62\% of adults reported physical activity as “extremely effective” in reducing stress (\cite{apa2014stress}). These stress-relieving techniques can increase one’s resilience to stress, allowing their experience to grow into strength. By fostering resilience through mindfulness and physical activity, schools help students manage stress to build strength. With the correct support, stress becomes a tool for growth rather than a burden, fostering a healthy mindset and academic success. 

The American Dream has long been the golden standard that all strive for. Traditionally, the dream was defined as the promise that hard work is rewarded by material wealth and other classic markers of success: owning a house with a white picket fence, going to a good college, and living patriotically. These dreams have brought millions of immigrants to America and motivated countless generations. As our society’s standards and goals change, the American Dream for many has shifted to incorporate new aspirations, often centered around personal freedom: being financially independent, finding meaningful relationships, and, most importantly, being healthy, happy, and emotionally fulfilled. For adolescents, traditional American values prioritizing strong education can take precedence over core human values of being healthy and fulfilled. Chronic academic stress can force a reliance on external validation from markers like GPA and test scores. Luckily, the shift from traditional American values to more relevant human ones promotes support for academic stress and a greater focus on the more important aspects of life: happiness, health, and fulfillment.



\nocite{*}
\printbibliography[title=References,
heading=subbibliography]
\end{refsection}

\begin{refsection}[refs/moralint]

\makechapter{Against Moral Intuitionism}{Against Moral Intuitionism}{Celine Wang}{Farmington High School}{10.17613/n9xdf-8dh92}

\section{Introduction}

In this paper, I contend that we do not have any reason to trust in our moral intuitions on the grounds that moral intuitions are epistemically unreliable, suffer from persistent disagreement, and are subject to debate regarding its objectivity. 

To limit the scope of our discussion, I will be focusing on a foundationalist interpretation of intuition as a means of knowledge production (\cite{britannica2024intuition}). As such, an intuition shall thereby be defined as a self-evident “intellectual seeming” (\cites{audi1999self}[p.\ 10]{bealer1998intuition}[p.\ 102]{huemer2005ethical}[p.\ 1]{ogar2016critique}). Consequently, moral intuitionism would entail a non-natural, realist stance on the objectivity of morality, and would be the utilization of intuition as an epistemic tool concerning moral matters; i.e., what is right and what is wrong (\cites{ethics2024moral}{stratton2014intuitionism}).  

\section{Argument From Unreliability}

As opposed to perception, intuition is notoriously unreliable. While not infallible, perceptual seemings (i.e., the visual analog of intellectual seemings) are typically considered reliable because they detect potential inaccuracies \emph{vis-à-vis} context. In other words, when weather phenomena obscures vision or when physically impaired, we are less apt to trust our perceptual seemings (\cite[p.\ 8]{cecchini2024reliability}). Intellectual seemings, on the other hand, lack the capacity to receive reliable feedback contextually (\cite[pp.\ 17–18]{destefano2014reliability}).

Additionally, the non-naturalness of moral intuition implies that they are causally inert, i.e., they can neither affect nor be affected by other objects. That is to say, moral intuitions cannot be caused by their corresponding moral truths (\cites[p.\ 1]{hayward2019immoral}{stratton2014intuitionism}).

Moral intuitions are also subject to various distorting factors.  

First, phraseology. Studies have concluded that when the trolley problem was presented with “saving language”, people overwhelmingly choose to flip the switch; the opposite occurred when “killing language” was employed (\cite[p.\ 6]{bengson2013experimental}).

Second, partiality. One might have ulterior motives for experiencing certain intuitions (\cite[pp.\ 343–346]{sinnott2006moral}). For example, in judging two piano performances, the father of one of the contestants would, intuitively, think that his daughter played better, though this might not be justified without familial association (\cite[p.\ 343]{sinnott2006moral}).

Third, disgust. Disgust elicited by physical or chemical stimuli and disgust regarding some moral prospects, respectively, have been shown to influence the moral intuitive process (\cite[p.\ 8]{tao2022effects}).

Moreover, other factors also influence intuition. These include the ordering of cases involving intuition, hunger level of intuitive moral agents, and the “Knobe effect,” i.e., the tendency for people to judge intentional actions as negative rather than positive (\cites{swain2008instability}[p.\ 1]{danziger2011extraneous}).

An argument which has been proposed by Jonathan Smith states that partiality is imprecise. Concerning the example mentioned previously, Smith argues that a piano performance is incomparable with moral intuition. There are nuances and particularities present in musical performances which are not present in the fundamental claims of basic moral intuitions (e.g., murder is bad). He presents a counterexample: two contestants play a single note and are judged based on proximity with a third note. Thereby, if the father judges that the note his daughters played was closer, then he would be justified in believing that his daughter won, regardless of his familial predisposition towards her (\cite[p.\ 77]{smith2010sinnott}).

Though I agree that the father would be justified in the second instance, this argument assumes a moral realist conclusion. A better reformulation of the piano competition would be one in which the two contestants are simply told to simply play one note. The father is then asked simply, “Who played better?” By eliminating the intended note (or, when applied to the topic of moral intuitionism, some accepted moral principle), partiality comes back into play. 

Considering another possibility, Dario Cecchini suggests that the strength of intuition may be a method of tracking the potential reliability of intuition in a context-sensitive way. This would only be the case if one truly accepts moral intuitions proportional to confidence in said intuitions, and that such confidence is epistemically reliable (\cite[p.\ 11]{cecchini2024reliability}). One’s confidence would then accordingly be used to determine the presence of biases, and as such, eliminating the possibility of distorted intuitions. 

Nonetheless, there are inconsistencies in this argument. The idea that intuitive strength is epistemically reliable remains hotly contested, as it currently lacks sufficient empirical study (\cite[p.\ 22]{cecchini2024reliability}). Furthermore, it seems unlikely that one would be able to prove the epistemic reliability of moral intuitions, given that intuitive strength itself is subject to distorting factors such as the Dunning-Kruger effect or the “consensuality principle” (\cites[pp.\ 23–24]{cecchini2024reliability}[pp.\ 259–262]{dunning2011kruger}{koriat2012self}). Such an argument is also subject to the implication that self-evidence is strength dependent, which undermines the very definition of the term itself. 

\section{Argument from Disagreement}

One might assume that since moral intuition is “self-evident” and “non-inferential,” that there would be harmony between proponents of moral ideas. Yet, this is not the case. When ten diverse societies were examined for their views regarding moral principles, each “exhibited substantial variation not only in the degree to which such [moral] factors were viewed as excusing, but also in the kinds of [moral] factors taken to provide exculpatory excuses, and in the types of norm violations for which such [moral] factors were seen as relevant” (\cite[p.\ 4692]{barrett2016small}). If the moral intuitions of two unprejudiced individuals persistently disagree on some moral issue, then the realist basis for intuitionism would be compromised (\cites[pp.\ 208–210]{sidgwick1907methods}[p.\ 44]{shaferlandau2004good}).

In addition, what seems “moral” to us may be evolutionarily favored biological responses that have emerged over time. We have evolved to react to different moral acts with either approval or disapproval based on biological necessity, preventing us from attaining positive knowledge of moral predicates (\cites[p.\ 348]{singer2009ethics}[p.\ 72]{joyce2007evolution}{morton2016new}). Differences in moral beliefs have also been observed across ages and between modern and traditional culture groups (\cite{nisan1987moral}). One especially notable example of moral disagreement is the disagreement between moral collectivists and individualists, as many common moral realist objections directed towards the idea that their disagreements will not remain persistent, have been disproven (\cite[pp.\ 855–864]{goldman2022right}).

Nevertheless, this is a shallow argument, as non-moral disagreements are not the true point of contention in a moral issue. In comparison with the previous example, we would certainly object if a living person were to be boiled, even without experiencing pain. The question again returns to a moral one: Are the lives of a lobster and a person worth the same? If one continues to argue that “worth” could be determined via some empirical measure, e.g., brain capacity, then a new moral issue emerges: is it morally acceptable to boil some organism simply based on the metric of brain capacity? This line of argument suggests the infinite regressive loop fallacy.  

Russ Shafer-Landau proposes that the argument from moral disagreement is self-defeating. He summarizes said argument in terms of moral realism, a key property of intuitionism: 
\begin{enumerate}
	\item[P1:] \textit{Moral realism is subject to persistent disagreement;}

	\item[P2:] \textit{Any theory(s) subject to persistent disagreement is false;}

	\item[C:] \textit{Therefore, moral realism is false (\cite[pp.\ 44–48]{shaferlandau2004good})}.
\end{enumerate}

To this, he claims that the same argument could be applied in reverse. Substituting “moral realism” for “theories against moral realism” results in the conclusion that theories against moral realism are false. 

As we will see, though Shafer-Landau’s argument is valid, it is not sound. Concealed by Shafer-Landau’s summary of the argument from disagreement is what I will term the agreement premise. The premise states that if moral realism is true, then there would be widespread agreement regarding moral realism. 

\begin{enumerate}

	\item[P1:] \textit{The agreement premiss is true;}

	\item[P2:] \textit{Moral realism is subject to persistent disagreement;}

	\item[P3:] \textit{Any theory(s) subject to persistent disagreement is false;}

	\item[C:] \textit{Therefore, either (a) moral intuitionism is false, or (b) the agreement premise is false.}

\end{enumerate}

Accepting either (a) or (b) would be difficult for the moral realist. Conversely, if one were to now substitute “moral realism” for “theories against moral realism,” then (b) would be acceptable to opponents of moral realism, as it implies that only some would agree with moral realism, perfectly. In other words, Shafer-Landau’s argument, though compelling, is more so \emph{vis-à-vis} the moral realist. 

\section{Argument from Queerness}

Moral Intuitionism is a theory that is built upon a foundation of self-evident, non-inferential moral objectivity. However, the very existence of such values can be challenged. In J. L. Mackie’s Argument from Queerness, Mackie reasons such that if objective values were to exist, then objectively good values would have an inherent “to-be-pursuedness,” and objectively bad values would have an inherent “not-to-be-pursuedness” (\cites[p.\ 40]{mackie1978ethics}[p.\ 104]{olson2014moral}). This motivational property is one that is unique to objective moral values, and one which cannot be described in naturalistic terms; in other words, objective values are “queer” (\cites[pp.\ 2–3]{scholl2015defense}[p.\ 33]{mackie1978ethics}) To claim that such a property is not queer is to invite contradiction with the non-natural property of moral intuitions (\cite[p.\ 41]{mackie1978ethics}). The argument can be represented in both metaphysical and epistemological versions, however, only the epistemological is necessary here:  

\begin{enumerate}

	\item[P1:] \textit{If objective moral values exist, they will have to be epistemologically queer, and require unconventional, bizarre epistemological faculties;}

	\item[P2:] \textit{We are unable to utilize such bizarre epistemological faculties;}

	\item[C:] \textit{We should not believe that objective moral values exist (\cites[pp.\ 38–41]{mackie1978ethics}[pp.\ 1–2]{lillehammer2019queerness}).}

\end{enumerate}

The parity argument, proposed by Terence Cuneo, attempts to prove the existence of objective values. His argument has two main premises: the parity premise ($P_P$) and the epistemological realism premise ($P_E$). The parity premise states that moral facts and epistemic facts are analogous to one another since both are categorical imperatives, i.e., both tell us what we ought to do and believe regardless of our subjective desires. The epistemic realism premise describes the idea that belief in the existence of epistemic facts seems natural; otherwise, one could theoretically have misconstrued beliefs based only on whims and compulsions (\cite[pp.\ 52–62]{cuneo2010normative}). His argument is outlined as the following, with an extra “modus tollens” ($P_{MT}$) premise added for clarity:
\begin{enumerate}

	\item[P1:] \textit{If objective moral values exist, they will have to be epistemologically queer, and require unconventional, bizarre epistemological faculties;}

	\item[P2:] \textit{We are unable to utilize such bizarre epistemological faculties;}

	\item[C:] \textit{We should not believe that objective moral values exist (\cites[pp.\ 38–41]{mackie1978ethics}[pp.\ 1–2]{lillehammer2019queerness}).}

\end{enumerate}

The parity argument, proposed by Terence Cuneo, attempts to prove the existence of objective values. His argument has two main premises: the parity premise ($P_P$) and the epistemological realism premise ($P_E$). The parity premise states that moral facts and epistemic facts are analogous to one another since both are categorical imperatives, i.e., both tell us what we ought to do and believe regardless of our subjective desires. The epistemic realism premise describes the idea that belief in the existence of epistemic facts seems natural; otherwise, one could theoretically have misconstrued beliefs based only on whims and compulsions (\cite[pp.\ 52–62]{cuneo2010normative}). His argument is outlined as the following, with an extra “modus tollens” ($P_{MT}$) premise added for clarity: 
\begin{enumerate}

\item[$P_P:$] \textit{There can be no epistemic facts if there are no moral facts;}

\item[$P_E$:] \textit{There are epistemic facts;}

\item[$P_{MT}$:] \textit{If there are epistemic facts, then there are moral facts;}

\item[C:] \textit{If there are moral facts, then moral realism is true (\cite[p.\ 1]{rutten2010parity}).}

\end{enumerate}

Believing in moral realism, by definition, entails believing in objective moral values, and in this case, moral intuitionism. 

Regardless, though the argument is valid, the parity premise lacks argumentative power. As Richard Joyce notes, there are many ways in which moral facts and epistemic facts are incompatible. Examples include how unlike with moral facts: 

\begin{enumerate}

	\item We acquire epistemic facts involuntarily.

	\item Epsitemological facts lack the concept of desert, i.e., "deserved reward or punishment (\cite{merriam2024desert})."

	\item We are not reluctant to rely on the judgement of experts when faced with epistemological facts (\cite[pp.\ 14–16]{joyce2007evolution}).

\end{enumerate}

As a result of these discrepancies, we can safely disregard the parity argument and regard the argument unsound.  

Hallvard Lillehammer expresses another objection, claiming that knowledge of objective moral values does not entail the utilization of epistemologically queer faculties (\cite{balaguer2016platonism}). He argues that moral facts can be equated with logical facts, since both are causally inert (\cite{balaguer2016platonism}). Hence, as Lillehammer points out, if logical facts rely only on an “unproblematic sense … which we are able to grasp” and not some bizarre epistemological faculty, then the same should be the case for moral facts. If one were to deny this argument, the implication would be that we are unable to access logical facts, a highly contentious and counterintuitive claim (\cite[p.\ 6]{lillehammer2019queerness}). 

Whilst claiming that logical facts are epistemologically inaccessible is certainly absurd, we can nonetheless dispute the notion that logical facts and moral facts are analogous. One key disanalogy between the two lies in that logical facts lack the “to-be-pursuedness” of moral facts. For example, while the moral fact “giving to charity is good” carries an inherent tendency to act, a tautology such as “all humans are mammals” does not (\cite{britannica2024tautology}). 

\section{Conclusion}

I have presented three arguments against moral intuitionism in this paper, including how:

\begin{enumerate}
\item Moral intuitions are fallible.

\item There are disagreements between supposedly ``self-evident'' moral intuitions.

\item The objective nature of moral intuitions can be challenged.
\end{enumerate}

Thus, I reject the notion that we have any good reason to trust our moral intuition.


\nocite{*}
\printbibliography[title=References, heading=subbibliography]
\end{refsection}

\begin{refsection}[refs/farm]
\makechapter{The Political Economy of Farm Subsidies}{The Political Economy of Farm Subsidies}{Sarah Kim}{Las Lomas High School}

“America, land of the free”—or so they say. In reality, the United States is far from being a genuinely free nation. The American government has a well-documented history of abusing its considerable power, including in the protection of slavery, restriction of voting rights, and suppression of dissent. However, what is less well-known but hugely significant is the government’s intervention in agriculture. The roots of this extensive involvement lie in the Agricultural Adjustment Act of 1933, which was enacted to stabilize farm prices during the Great Depression. Decades after the Depression, however, the government has continued to subsidize agriculture. Furthermore, while aimed at aiding farmers and lowering prices, government intervention has been highly detrimental, impacting both the U.S. and the international community. In essence, the American government’s agricultural intervention exemplifies the coercive nature of government and underscores the need to limit such practices in society.  

Although the original intention of subsidies was to alleviate the plight of struggling farmers during the Great Depression, farmers are no longer in the same desperate straits. They are thriving, in fact. Paradoxically, despite farmers’ average incomes being 52 percent above the national average and their net worth reaching eight times the average American’s net worth, the government continues to subsidize their income. Indeed, nearly 40 percent of farmers’ income is derived from government aid. Furthermore, subsidies are regressive in nature. In 2019, only around 3 percent of American farmers collected nearly 70 percent of subsidies; on average, from 1995 to 2021, while the bottom 80 percent of farmers received around 9 percent of all federal allocations, the top decile received nearly 78 percent of the total. Since many farm subsidies are distributed based on a certain farm’s historical production volume, and larger farms have higher production volumes, they are given the majority of federal subsidies. Additionally, there is the problem of subsidy-induced overproduction, which leads to artificially suppressed food prices. However, to maintain prices, the government has set price floors for various farm products. The sugar price floor, for instance, results in an annual cost of approximately \$4 billion for consumers. Beyond this, consumers also face a tax burden of roughly \$206 per household per year to finance the government subsidy programs. It is also ironic that the government, after creating a supply glut through subsidies, then purchases and resells excess goods internationally at a steep discount. In this way, the government incurs significant financial losses, effectively subsidizing foreign consumers at the expense of U.S. taxpayers. As evidenced, farm subsidies produce inequality, long-term market distortions, and increased costs for consumers, compromising the overall health of the U.S. economy. 

Beyond these domestic repercussions, the impact of American farm subsidies is felt worldwide. Typically, developing countries have a comparative advantage in producing agricultural goods. However, this advantage is nullified by the subsidization of products produced by more developed nations. The ramifications of this are substantial for developing economies. Each year, such subsidies cost developing nations around US \$24 billion, not accounting for any spillover effects; the inclusion of such effects would yield a significantly higher total, given that agriculture often comprises a large portion of developing nations’ economies. Studies have shown that a 1 percent increase in Africa’s total agricultural exports would lift its GDP by US \$70 billion per year, roughly five times the amount of foreign aid received in the same period. Clearly, even beyond the U.S., farm subsidies have had devastating effects. 

Having illustrated the failings of subsidies, we should naturally ask ourselves, \emph{why do they still exist?}  The answer to this question lies in politics. Just like any other government program, the recipients of aid create interest groups that fiercely defend their handouts, relentlessly perpetuating the myth of the struggling family farmer whose woes require overwhelming federal subsidies to remedy. Nonetheless, such is a narrative detached from reality. Another method of maintaining farm subsidies is by aligning them with food security, that is, presenting a bill to Congress that includes both farm subsidies and food security proposals. By linking farm subsidies to food security, opposition to subsidy programs is framed as an assault on the food-insecure, transforming a discussion of economic policy into a morally condemnatory criticism of political opponents.  

In summary, farm subsidies exemplify government coercion in two primary ways. First, farm subsidies erode public trust in the government by fostering an environment in which politically connected agricultural interests benefit at the expense of both domestic consumers and vulnerable populations abroad. Second, farm subsidies undermine the individual freedom of farmers and consumers. Consumers are unwittingly coerced into funding the livelihoods of wealthy farmers, while smaller farmers struggle to survive under government-imposed disadvantages. Overall, farm subsidies represent a clear example of government overreach. Where the free market would have been able to allocate resources efficiently, the government has stepped in to forcefully disrupt the equilibrium. 

When the government is unchecked in its power, it no longer acts with the interests of the people in mind, leading to policies that demand compliance rather than reflect consent. In addition, the government is unable to adapt to changing circumstances. What was once a measure intended to buttress the agrarian base of the U.S. economy has now become a means of entrenching wealth inequality. Although government intervention can have benevolent intentions, it usually erodes individual liberty, distorts the free market, and often reverses the intended effects of wealth redistribution, with taxpayers coerced to fund the endeavor. Ultimately, it is vital to uphold the principle of limited government, ensuring that policies do not become the means by which the state undermines the foundations of a free society. As economist Henry George once observed, “Government should be repressive no further than is necessary to secure liberty… and the moment governmental prohibitions extend beyond this line they are in danger of defeating the very ends they are intended to serve.” It is imperative that the government’s function be that of an impartial umpire, rather than an active stakeholder. Markets, not governments, create prosperity. 


\nocite{*}
\printbibliography[title=References,heading=subbibliography]
\end{refsection}

\begin{refsection}[refs/asymmetry]
\input{source/asymmetry}
\nocite{*}
\printbibliography[title=References,heading=subbibliography]
\end{refsection}

\begin{refsection}[refs/rejection]

% Metadata: put your article information here 
\makechapter{Rejection of Cognitivism}{Rejection of Cognitivism}{Jeff Cao}{Campolindo High School}


Cognitivism is the belief that trust essentially involves believing. Evidentialism is the belief that rational beliefs must be based on evidence. Whereas, conflict is the belief that rational trust sometimes conflicts with evidence. For trust to be rational, only two beliefs can be true. T. Nguyen rejects the belief about cognitivism.

According to Nguyen, trust is not a belief. To trust is to adopt an unquestioning attitude. To support his claim, Nguyen essentially argues that “to trust X as an informational source in domain Z is to have an attitude of not questioning X’s deliverances concerning Z”. By rejecting cognitivism and instead focusing on how conflict and evidentialism coincide to make trust rational, we align with Nguyen’s theory that trust is an unquestioning attitude, not a belief.

Nguyen rejects cognitivism by explaining that trust is not a belief but rather an unquestioning attitude. Based on this view, to trust someone is not necessarily to believe that they will act in a certain way. Instead, you are suspending doubt without continuous evaluation. Implicit trust allows Nguyen to contend the view that when you are trusting them, you are not believing that they are trustworthy, but you still treat them as if they are trustworthy.

There are three examples that Nguyen gives to support trust as an unquestioning attitude. First, when someone walks on the ground. You are trusting the ground to support your weight as you walk. It becomes natural, you are not constantly thinking about whether you will fall because the ground does not support you. Second, when you trust your memory to remind you to get the missing ingredients at the supermarket. Third, when you trust your doctor to give appropriate medical advice. 

A real-life example of implicit trust is when you are in class listening to your instructor. When you are in class, you do not know what type of person your professor is. It does not really matter, either. What matters is that you are trusting the professor to be an informational source in the classroom. You are not questioning whether they know about the course material and are capable of teaching it. When you are in a classroom setting, you are implicitly trusting that your professor is capable of teaching.

Nguyen firmly supports evidentialism—the idea that rational beliefs must be grounded in evidence, although with a caveat. According to Nguyen, trust does not replace belief, but rather enables us to form beliefs by integrating trustworthy agents and non-agents into our lives. In his proposition, the concept of trust functions as a way to increase cognitive efficiency. With this attitude of trust, one functions more effectively without constantly reassessing all evidence. He states, “when I accept testimony through trust without deliberation, it isn’t that I have failed to go through a proper practical deliberation. I am deferring to deliberation that was run elsewhere.” This essentially means that trust involves belief in the judgments of others, specifically if agents/non-agents have evidenced their trustworthiness.

Nguyen makes a key distinction between trust and belief to clarify how evidentialism is preserved. Trust, he argues, is not a belief about someone’s reliability. Instead, it’s an unquestioning attitude that temporarily suspends personal evaluation, giving external sources (like memory, doctors, or technology) direct input into one’s reasoning. Beliefs, by contrast, are still subject to evidential standards. An example of this dichotomy is trusting your doctor without constantly questioning them; you are not believing your doctor is correct in every moment—you are simply not re-evaluating. However, this trust can support belief by providing evidence through testimony, assuming your doctor’s advice is proven by their medical expertise. Thus, evidentialism is preserved not through internal reevaluation of every piece of evidence, but through epistemic dependence on trusted agents and non-agents.

While some argue that Nguyen’s idea of trust is based more on functional integration rather than genuine trust, Nguyen importantly distinguishes between trust and reliability. Unlike reliability, trust as an unquestioning attitude stipulates that true trust must be subconscious or fully clear of doubt. Yes, one can reason and reject objections to trusting in something, but never believe and accept counterpoints. This creates a deeper emotional form of trust that is illustrated by the idea of betrayal, or the breaking of trust. When people, objects, or internal functions fail us, the personal nature of betrayal illustrates the deeper bond of trust beyond that of mere reliability.

Nguyen defends the conflict thesis by arguing that trust can remain rational even when it contradicts available evidence. According to the distinction established earlier in rejecting cognitivism, the conflict between trust and evidence no longer implies irrationality. Instead, it reveals the cognitive limitations of humans. As Nguyen argues, we are cognitively limited beings who cannot constantly re-evaluate every piece of information or every relationship we rely on. To solve this limitation, we develop trust mechanisms that help to conserve cognitive resources. Functionally, trust operates like crossing one thing off our to-do list. Once trust is established, we do not continuously re-justify it. Even when counter evidence emerges, we are not necessarily required to immediately withdraw our trust. This temporary suspension of evaluation is precisely what gives trust its value. Trust is not irrational; it is a rational strategy that allows us to act effectively under conditions of limited attention and time.

In real-life examples, a surgeon may trust the anesthesiologist during a procedure, not because of recent reaffirming data, but based on the default assumption that they will fulfill their role. That trust may temporarily conflict with weak new signals of risk. But still, if the surgeon were to pause and re-evaluate in the middle of the operation, it could compromise team coordination and the safety of the patient. In this context, trust that contradicts minor evidence is not only rational and necessary.

A possible objection to Nguyen’s view is that suspending doubt might lead to blind trust. If trust persists despite counterevidence, how can it remain rational? Nguyen would respond that the suspension of evaluation is neither unconditional nor permanent. It is a temporary, pragmatic strategy that helps us act efficiently. When strong disconfirming evidence arises, we naturally reassess our trust. In this way, trust does not reject evidence; it simply postpones constant reevaluation until it is necessary. Moreover, Nguyen’s notion of “delegated deliberation” ensures that trust is still indirectly evidence-based. We rely on others’ judgments not out of ignorance, but because it is rational to do so under cognitive constraints. Thus, trust remains a rational attitude, not a lapse in reasoning.

By rejecting cognitivism, it is then possible to keep evidentialism and conflict to make trust a rational belief. Nguyen’s view reframes trust not as a belief, but as an unquestioning attitude that suspends active evaluation in order to manage our cognitive limitations. This makes space for a form of rational trust that can sometimes persist despite conflicting evidence. Trust, then, is not a deviation from rationality, but an adaptive response to the practical demands of reasoning and action in a complex world.

\nocite{*}
\printbibliography[title=References,heading=subbibliography]
\end{refsection}

\begin{refsection}[refs/washington]

\makechapter{Washington's Neutrality Proclaimation}{Washington's Neutrality Proclamation and the Birth of Executive Authority}{John Wang}{BASIS Fremont High School}{10.17613/xdy8g-g1p48}

In 1793, the nascent French Republic launched preemptive military campaigns against its encircling monarchical neighbors to ensure its own survival \autocite{holland1911}. The result was a war of unprecedented scope and devastation. However, while primarily focused on Europe, France also inadvertently embroiled the young American republic in its first major foreign policy crisis. Caught between treaty obligations to France and the threat of British retaliation, the United States faced mounting pressure to choose sides, risking entanglement in a conflict that could tear the fragile union apart. It was under such uncertainty that President George Washington issued the Neutrality Proclamation. On the surface, it was nothing special: the short statement ostensibly asserted America’s intention to remain uninvolved in European conflicts. However, a close reading of the Proclamation reveals a deliberate effort to silence factional clamor in the national interest, an assertion of sharp Federalist assumptions about the extent of executive power, and a dismissal of revolutionary ideology as irrelevant to American interests. 

Although drafted by Edmund Randolph, the Proclamation bore Washington’s unmistakable influence, a fact most evident in its deliberate word choice \autocite[p.\ 59]{moats2021}. Strikingly, the term “neutrality” does not appear a single time; instead, the text uses “friendly” and “impartial” (\cite{washington1793}). This signaled a strategic ambiguity designed by Washington to soften the political implications of a policy that seemed to clearly align with Federalist interests (\cite[p.\ 71]{reinstein2011}). The words “friendly” and “impartial,” unlike “neutral,” carry the connotation of fairness without formal renunciation; it enabled the administration to distance itself from the French cause without appearing to betray it (\cite[p.\ 65]{moats2021}). At a time when the 1778 alliance with France remained technically binding, and when pro-French sentiments among Republicans ran high, the Proclamation’s language carefully tempered partisan divisions between Federalists and Republicans (\cite[p.\ 223, 215]{reinstein2011}). Nonetheless, this was a minor concession; the Proclamation’s effect remained distinctly Federalist, as it universalized the American stance as one firmly opposed to revolutionary fervor. At heart, Washington feared that open partisanship would entangle the young republic in foreign wars for which it was neither militarily nor politically prepared (\cite[pp.\ 470–471]{sheridan1994}). Such pragmatism, reflected in both the wording and intent of the Proclamation, forged a rare point of unity between rival factions in the cabinet (\cite[pp.\ 689–691]{chernow2011}). In doing so, Washington alleviated internal discord regarding the Proclamation’s issuance. 

Beyond foreign affairs, the Neutrality Proclamation also subtly redefined the constitutional boundaries of executive power. The 1778 Treaty of Alliance with France had pledged reciprocal military support should either nation be attacked, and it obligated the United States to defend French possessions in the Americas (\cite[p.\ 11–12]{reinstein2011}). By proclaiming neutrality, however, Washington effectively suspended this commitment, all without explicit congressional approval and absent any formal treaty abrogation. As such, many Republicans argued that Washington’s declaration overstepped the executive’s power (\cite[p.\ 126]{schmitt2000}). Madison, in particular, argued that Washington had issued the Proclamation in “unqualified terms” that ignored established treaty obligations to France (\cite[p.\ 126]{schmitt2000}). Under the Constitution, Congress holds the power to declare war; therefore, as Republicans reasoned, they possessed the authority to declare neutrality (\cites[p.\ 691]{chernow2011}[p.\ 127]{schmitt2000}[pp.\ 329–331]{prakash2001}). This controversy over presidential authority culminated in the pivotal Pacificus-Helvidius debates. In the Pacificus essays, Hamilton championed Washington’s Proclamation, arguing that Article II of the Constitution vested the president with all-encompassing “executive power,” including the “federative” power of managing foreign affairs (\cite[§\textsc{xi}, p.\ 12]{hamilton2007}). Under such a view, congressional powers to declare war and ratify treaties were exceptions—not limitations—on otherwise broad executive discretion. In response, Madison challenged Hamilton’s claims through the Helvidius essays, contending that foreign affairs, especially decisions with treaty implications, required legislative consent (\cite[p.\ 64]{hamilton2007}). The executive’s authority to create treaties, after all, requires a vote from the legislature; if the executive sought to withdraw the U.S. from treaty obligations, it ought also to consult the legislature (\cite[§2, cl. 2]{constitution}). Undoubtedly, Washington’s move occupied constitutional gray space: Article II granted the president executive authority, although without precise demarcation (\cite[pp.\ 445–446]{young2011}). Nevertheless, opponents of this expansion of power, though dismayed, found reassurance in the Proclamation’s ambiguity, ultimately valuing national unity and peace over ideological purism (\cite[p.\ 65, 75]{moats2021}). Thus, the assertion of power was tolerated. In this way, Washington used deliberate ambiguity to preserve presidential discretion without provoking accusations of despotism (\cite[p.\ 68]{moats2021}).

Beyond its legal implications, the Proclamation’s greatest statement may have been what it refused to say. On April 8, 1793, Edmond-Charles Genêt, a French diplomat charged with enlisting American support for the French Revolution, arrived in South Carolina with the mission and radical fervor of Jacobinism—fervor that Washington feared might entangle American republicanism with radicalism (\cites[p.\ 464]{sheridan1994}[pp.\ 444–445]{mccullough2008}{hamilton1793}). However, the Proclamation issued two weeks later never mentioned France, revolution, or Genêt. This refusal to address Genêt’s republican evangelism in explicit terms was deliberate (\cite{washington1793}). By distancing itself from direct confrontation, the document shrewdly redefined American identity in contrast to the excesses of the French Revolution. Accordingly, neutrality emerged as a method of national self-definition: America would not be like France (\cites[p.\ 470]{sheridan1994}[p.\ 465]{young2011}). It is rather ironic, then, that the Proclamation that sought to rise above politics anchored the presidency to public opinion. Faced with Genêt’s popularity and Republican criticism, Federalists mobilized mass meetings and coordinated pro-administration resolutions, all to rally popular support behind Washington’s Proclamation (\cite[p.\ 454–456]{young2011}). The president responded in kind, formalizing a new interactive relationship between the executive and the public (\cite[pp.\ 437, 456–58, 462]{young2011}). In the face of popular outcry, Washington did not retreat from the public arena; instead, he embraced a public-facing presidency—one that cultivated mass approval to maintain legitimacy, even as it sought to curtail popular radicalism (\cite[p.\ 465]{sheridan1994}). This has long-term consequences. The Proclamation set the stage for the public politicization of the presidency, not only in function but also in perception (\cite[p.\ 435]{young2011}).

The Neutrality Proclamation of 1793 did more than simply outline American neutrality. Through the strategic use of ambiguous language, careful assertion of executive authority, and refusal to indulge in partisan fervor, Washington elevated neutrality to a foundational principle of American foreign policy. The Neutrality Proclamation established a model for foreign policy that persisted over the next century and a half. Furthermore, it instituted a new norm in foreign policymaking—one defined by a powerful yet measured executive branch capable of navigating both domestic and international challenges (\cite[p.\ 107]{reinstein2011}).



\nocite{*}
\printbibliography[title=References,heading=subbibliography]
\end{refsection}

\begin{refsection}[refs/tickets]
\input{source/tickets}
\nocite{*}
\printbibliography[title=References,heading=subbibliography]
\end{refsection}

\end{document}
