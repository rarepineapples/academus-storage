\makechapter{The Political Economy of Farm Subsidies}{The Political Economy of Farm Subsidies}{Sarah Kim}{Las Lomas High School}

“America, land of the free”—or so they say. In reality, the United States is far from being a genuinely free nation. The American government has a well-documented history of abusing its considerable power, including in the protection of slavery, restriction of voting rights, and suppression of dissent. However, what is less well-known but hugely significant is the government’s intervention in agriculture. The roots of this extensive involvement lie in the Agricultural Adjustment Act of 1933, which was enacted to stabilize farm prices during the Great Depression. Decades after the Depression, however, the government has continued to subsidize agriculture. Furthermore, while aimed at aiding farmers and lowering prices, government intervention has been highly detrimental, impacting both the U.S. and the international community. In essence, the American government’s agricultural intervention exemplifies the coercive nature of government and underscores the need to limit such practices in society.  

Although the original intention of subsidies was to alleviate the plight of struggling farmers during the Great Depression, farmers are no longer in the same desperate straits. They are thriving, in fact. Paradoxically, despite farmers’ average incomes being 52 percent above the national average and their net worth reaching eight times the average American’s net worth, the government continues to subsidize their income. Indeed, nearly 40 percent of farmers’ income is derived from government aid. Furthermore, subsidies are regressive in nature. In 2019, only around 3 percent of American farmers collected nearly 70 percent of subsidies; on average, from 1995 to 2021, while the bottom 80 percent of farmers received around 9 percent of all federal allocations, the top decile received nearly 78 percent of the total. Since many farm subsidies are distributed based on a certain farm’s historical production volume, and larger farms have higher production volumes, they are given the majority of federal subsidies. Additionally, there is the problem of subsidy-induced overproduction, which leads to artificially suppressed food prices. However, to maintain prices, the government has set price floors for various farm products. The sugar price floor, for instance, results in an annual cost of approximately \$4 billion for consumers. Beyond this, consumers also face a tax burden of roughly \$206 per household per year to finance the government subsidy programs. It is also ironic that the government, after creating a supply glut through subsidies, then purchases and resells excess goods internationally at a steep discount. In this way, the government incurs significant financial losses, effectively subsidizing foreign consumers at the expense of U.S. taxpayers. As evidenced, farm subsidies produce inequality, long-term market distortions, and increased costs for consumers, compromising the overall health of the U.S. economy. 

Beyond these domestic repercussions, the impact of American farm subsidies is felt worldwide. Typically, developing countries have a comparative advantage in producing agricultural goods. However, this advantage is nullified by the subsidization of products produced by more developed nations. The ramifications of this are substantial for developing economies. Each year, such subsidies cost developing nations around US \$24 billion, not accounting for any spillover effects; the inclusion of such effects would yield a significantly higher total, given that agriculture often comprises a large portion of developing nations’ economies. Studies have shown that a 1 percent increase in Africa’s total agricultural exports would lift its GDP by US \$70 billion per year, roughly five times the amount of foreign aid received in the same period. Clearly, even beyond the U.S., farm subsidies have had devastating effects. 

Having illustrated the failings of subsidies, we should naturally ask ourselves, \emph{why do they still exist?}  The answer to this question lies in politics. Just like any other government program, the recipients of aid create interest groups that fiercely defend their handouts, relentlessly perpetuating the myth of the struggling family farmer whose woes require overwhelming federal subsidies to remedy. Nonetheless, such is a narrative detached from reality. Another method of maintaining farm subsidies is by aligning them with food security, that is, presenting a bill to Congress that includes both farm subsidies and food security proposals. By linking farm subsidies to food security, opposition to subsidy programs is framed as an assault on the food-insecure, transforming a discussion of economic policy into a morally condemnatory criticism of political opponents.  

In summary, farm subsidies exemplify government coercion in two primary ways. First, farm subsidies erode public trust in the government by fostering an environment in which politically connected agricultural interests benefit at the expense of both domestic consumers and vulnerable populations abroad. Second, farm subsidies undermine the individual freedom of farmers and consumers. Consumers are unwittingly coerced into funding the livelihoods of wealthy farmers, while smaller farmers struggle to survive under government-imposed disadvantages. Overall, farm subsidies represent a clear example of government overreach. Where the free market would have been able to allocate resources efficiently, the government has stepped in to forcefully disrupt the equilibrium. 

When the government is unchecked in its power, it no longer acts with the interests of the people in mind, leading to policies that demand compliance rather than reflect consent. In addition, the government is unable to adapt to changing circumstances. What was once a measure intended to buttress the agrarian base of the U.S. economy has now become a means of entrenching wealth inequality. Although government intervention can have benevolent intentions, it usually erodes individual liberty, distorts the free market, and often reverses the intended effects of wealth redistribution, with taxpayers coerced to fund the endeavor. Ultimately, it is vital to uphold the principle of limited government, ensuring that policies do not become the means by which the state undermines the foundations of a free society. As economist Henry George once observed, “Government should be repressive no further than is necessary to secure liberty… and the moment governmental prohibitions extend beyond this line they are in danger of defeating the very ends they are intended to serve.” It is imperative that the government’s function be that of an impartial umpire, rather than an active stakeholder. Markets, not governments, create prosperity. 

