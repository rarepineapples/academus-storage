
\makechapter{The American Dream's Flawed Promise of Education}{The American Dream’s Flawed Promise of Education:\\ The Issue of Academic Stress}{Charles Schäfer}{Stuyvesant High School}


At the center of the American dream lies the promise of education: the idea that with hard work comes reward. Unfortunately, in such a rapidly shifting society, established standards have quickly begun to shift. The issue of academic stress and its impact on adolescents has become increasingly important in recent years. Increased competition for college and jobs has driven the academic expectations for adolescents higher and higher. High school students take more college-level classes, fill out their resumes with more extracurriculars, and score higher on standardized tests. Unfortunately, as academic demands increase, so does academic-related stress. Academic stress encompasses various stressors, such as intimidating workloads, performance pressure, and even self-inflicted stressors like time management issues. These stressors can have major long and short-term effects on one’s physical and mental health, potentially resulting in sickness for a week or even leading to long-term mental health concerns that extend into college and beyond. Despite its many detrimental effects, in the correct environment, academic stress can be turned into resilience and strength. The importance of the impact of academic stress is creating educational institutions that support students’ personal growth, rather than enforce detrimental stress. Addressing this issue is vital for improving students' academic performance, well-being, and future. While academic stress can have severe short-term consequences and devastating long-term effects, supportive educational environments can alleviate stress and promote resilience. 

Excessive academic stress in challenging environments can cause significant short-term repercussions, including anxiety and depression, feelings of inadequacy, and even physical consequences. Periods of peak stress in academically rigorous environments, particularly distance learning, correlate with increased depression and anxiety. In academically stressful environments, students are typically juggling tests, homework, and high grades. This balancing act puts extreme strain on the overworked student: “Sudden bursts of academic stress significantly undermine emotional health and well-being” (\cite{claney2023}). In addition, stressful environments heavily contribute to increasing stress and its repercussions. One of the most stressful environments for students was the COVID-19 pandemic, where distance learning saw increased depression among college students due to feelings of isolation and academic pressure (\cite{chen2024}). Locked in their rooms for the majority of the day, academic stress only amplified growing feelings of depression. Unfortunately, academic stress during the pandemic was pervasive, with 90 million cases of anxiety and 70 million cases of depression being newly measured (\cite{cordovaolivera2023}). According to a study conducted on 1200+ university students, 35\% reported having experienced depression from academic stress during the pandemic (\cite{cordovaolivera2023}). While stressful environments and moments of maximum stress can cause severe depression and anxiety, many other stressors can cause different psychological challenges. Self-imposed stressors, such as time management and perfectionism, exacerbate feelings of inadequacy and emotional instability. Self-imposed stressors are traits people have that often cause feelings of insecurity. Typically, these stressors are based on students' subjective beliefs about their ability to meet their academic goals, creating continuous pressure and a sense of failure if one’s high expectations are unmet (\cite{cordovaolivera2023}). The most common stressor, poor time management, was identified as the “most significant stressor for university students, leading to feelings of emotional instability” (\cite{cordovaolivera2023}). Other self-imposed stressors, such as maladaptive perfectionism or “hypercritical judgment upon failure to meet extreme goals”, are also associated with negative self-perception (\cite{almroth2019}). These toxic forms of introspection can reinforce feelings of inadequacy, whereas fear of failure can hinder one’s ability to confidently approach schoolwork and life as a whole. Research from the National Longitudinal Survey of Adolescent Health reveals that self-perception of inadequacy contributes heavily to emotional instability and fear of failure (\cite{mcleod2012}). This feedback loop of self-inflicted stressors increasing negative self-perception and emotional instability is exceedingly unhealthy, often manifesting itself in physical effects. Prolonged stress results in physical consequences including sleeplessness, changes in appetite, and a weakened immune system. One of the most common physical effects of stress is sleeplessness. This is caused when acute stress attacks increase heart rate and dilate blood vessels to direct blood to large muscles and the heart, elevating blood pressure in a "fight or flight" response (\cite{sha2023}). These attacks come unpredictably, resulting in sleeplessness as the body attempts to return to normal. Unfortunately, once stressed individuals are asleep, they experience “sleep disturbances”, where stress activates neurons in the hippocampus region causing interruptions or low-quality sleep (\cite{mcleod2012}). Another physical consequence of stress is an altered appetite, causing an increase or decrease in food intake, resulting in unhealthy eating patterns and weight fluctuations (\cite{sha2023}). Finally, extreme stress can lead to prolonged cortisol release, weakening the immune system and making the body more susceptible to illness (\cite{cordovaolivera2023}). While periods of extreme stress can have many short-term repercussions, long-term exposure to stress can have devastating permanent effects.

Chronic academic stress can have the long-term negative effects of creating poor stress management habits, increasing the chances of long-term physical weakness, and fostering permanent emotional and psychological damage. Unresolved stress from a high academic workload in adolescence weakens one’s ability to manage future stressors, leading to unhealthy coping strategies in adulthood. Adolescents under constant academic pressure are less likely to develop effective coping mechanisms, as stress often overrides cognitive strategies for stress management (\cite{chen2024}). Lowered self-worth caused by chronic academic stress can cause individuals to perceive future stressors as insurmountable. Over time, this causes students to develop maladaptive coping mechanisms or self-destructive methods for treating stress (\cite{cordovaolivera2023}). Repeated stress responses during the formative years of adolescence can influence how people respond to such stress in adulthood. Such stress-resolving strategies that carry over from adolescence can include “avoidance, denial, or substance abuse” (\cite{claney2023}). This can have massive effects on one’s professional and personal life, disrupting one’s career and relationships. In addition, extreme bursts of academic stress can activate the HPA axis, an anti-stress bodily system that causes sustained cortisol release (\cite{cordovaolivera2023}). This can impair cognitive and emotional regulation, making future responses to stress extremely ineffective. Similar to stress-coping habits, repeated physiological responses also cause permanent bodily harm. Persistent academic stress can permanently increase cardiovascular, immune, and cognitive weakness. Maintaining cardiovascular health is extremely important in living a healthy life. Unfortunately, long-term stress contributes to persistent increases in heart rate and blood pressure, which elevate the risk of hypertension, heart attack, and even stroke. Chronic stress also creates inflammation in coronary arteries, increasing the risk of cardiovascular disease (\cite{almroth2019}). All of these cardiovascular repercussions are associated with long-term chronic stress. Unfortunately, chronic stress can also damage other parts of the body: chronic stress also impairs communication between the immune system and the aforementioned HPA axis, leading to increased vulnerability to “chronic fatigue, metabolic disorders (e.g., diabetes, obesity), and immune dysfunction” (\cite{sha2023}). Beyond weakened cardiovascular and immune systems, academic stress can severely damage an adolescent’s cognitive abilities. Similar to trauma, chronic stress can negatively impact memory, attention, and cognitive function–all essential for learning and development. Stress also makes individuals more susceptible to mental health disorders like PTSD, Major Depressive Disorder, and Bipolar Disorder (\cite{claney2023}). Yet another burden that stress can have on the body is also seen in the continuous activation of the autonomic nervous system and immune system which can lead to long-term wear-and-tear on the body, causing exhaustion and a weakened physical state (\cite{almroth2019}). It is imperative to recognize stress’s wide-reaching consequences on the body, and how such a small part of one’s life can have devastating effects. Beyond physiological consequences, academic stress can also have deeper ramifications by affecting one’s attitude and drive. Excessive academic stress during adolescence increases the likelihood of persistent emotional and psychological challenges into adulthood including lessened motivation, reduced self-esteem, and reliance on external validation. Repeated academic stress can be extremely demoralizing, sapping the determination that one may have had prior. Chronic academic stress during adolescence is linked to long-term emotional exhaustion, reduced motivation, and an increased likelihood of burnout (\cite{cordovaolivera2023}). Extreme stress often causes academic or workplace struggle; repeated failure is not only demoralizing but also detrimental to one’s outlook on life. This is seen when people underestimate their abilities, perceiving themselves as weaker than their aspirations. Long-term academic stress correlates with an increased risk of self-esteem issues and psychological distress later in life (\cite{almroth2019}). With lowered self-esteem in adolescence, those with chronic stress often foster a dependency on external validation for self-worth (\cite{cordovaolivera2023}). In contexts where academic success is heavily emphasized, students frequently equate their value with external achievements and validation (\cite{almroth2019}). Arbitrary statistics such as GPA and test scores can massively impact a student’s mental health. If unresolved, this trend will continue into adulthood, where individuals struggle with identity and self-worth in a more consequential environment. Academic stress has many extremely negative long and short-term consequences on the body and mind; luckily, academic stress, in tandem with rigorous support, can benefit and strengthen adolescents. 

The extreme consequences of academic stress can be alleviated through rigorous support in providing stress-abating resources, promoting strong feelings of school belonging, and fostering resilience to transform stress into growth. Supportive academic environments provide stress-relieving resources and services such as crisis hotlines and mental health centers. With academic stress being so prevalent among students during the pandemic, many high schools and universities integrated crisis hotlines into their mental health resources. 
Institutions such as Wake Forest University promote hotlines such as The Oregon YouthLine, The Trevor Project, and the National Suicide Prevention Lifeline (\cite{wfu2025}). These hotlines, typically open continuously, provide service to those requiring immediate assistance or those simply needing someone to talk to. Such hotlines are extremely effective, with a 2020 study finding a 43\% decrease in overall caller distress, and a 56\% caller follow-up rate (\cite{boness2021}). Other stress-relieving resources have also found great success in supporting student mental health. Following the pandemic, the National Association of Student Personnel Administrators (NASPA) published research regarding strategies for helping students facing academic stress, citing on-campus counseling and support groups as the most effective methods to counteract academic stress (\cite{wfu2025}). School-provided support groups are beneficial due to their communal element. Interacting with students in similar situations promotes connection associated with “enhanced well-being and reduced stress” (\cite{duan2016}). Beyond school-based mental health centers, academically stressful environments can also alleviate stress by promoting a culture of school connectedness. A strong sense of school belonging correlates with lower symptoms of depression and anxiety, promoting resilience to stress. With lowered self-esteem being a major symptom of academic stress, feeling socially valued in one’s community can combat these negative feelings. This feeling of acceptance can largely reduce anxiety and depression symptoms in adolescents, seeing a marked drop after 12-18 months (\cite{allen2024}). Even long-term effects of chronic stress, including predisposition to mental health disorders, can be mitigated through feelings of school belonging. In a longitudinal study conducted on Australian high school and college students, those studying in socially welcoming environments saw a gradual decline in mental health symptoms over time, whereas those studying in purely stressful environments saw sharp increases in depression and anxiety (\cite{allen2024}). Social factors play a major role in influencing how adolescents approach schoolwork and stress. If supported by one’s peers, students are often pushed to work harder to compete with their peers. This can result in increased academic engagement and, by extension, resilience to stress. A 2022 study conducted on Dutch university students determined that engagement was a key pathway to resilience, with 78\% of engaged students reporting resilience to stress (\cite{versteeg2022}). Supportive environments prevent academic stress from hindering their students and transform experienced stress into strength. Supportive school environments cultivate mental resilience, stress-mitigating skills, and strength. Resilience plays a major role in turning stress into strength. Defined as the “ability to bounce back from adversity”, resilience is critical for adolescents in academically challenging environments (\cite{claney2023}). Supportive academic environments utilize stress management activities that promote resilience, such as mindfulness and physical activity. Mindfulness, the practice of “focusing on the present moment non-judgmentally”, fosters relaxation and a sense of calm (\cite{claney2023}). This can help teens cope with the pressures of school by disconnecting from external stressors to reduce anxiety. Physical activity can similarly act as a stress reliever: producing endorphins to improve mood, allowing the individual to detach themselves from their problems, and improving bodily health. In a study conducted by the American Psychological Association, 62\% of adults reported physical activity as “extremely effective” in reducing stress (\cite{apa2014stress}). These stress-relieving techniques can increase one’s resilience to stress, allowing their experience to grow into strength. By fostering resilience through mindfulness and physical activity, schools help students manage stress to build strength. With the correct support, stress becomes a tool for growth rather than a burden, fostering a healthy mindset and academic success. 

The American Dream has long been the golden standard that all strive for. Traditionally, the dream was defined as the promise that hard work is rewarded by material wealth and other classic markers of success: owning a house with a white picket fence, going to a good college, and living patriotically. These dreams have brought millions of immigrants to America and motivated countless generations. As our society’s standards and goals change, the American Dream for many has shifted to incorporate new aspirations, often centered around personal freedom: being financially independent, finding meaningful relationships, and, most importantly, being healthy, happy, and emotionally fulfilled. For adolescents, traditional American values prioritizing strong education can take precedence over core human values of being healthy and fulfilled. Chronic academic stress can force a reliance on external validation from markers like GPA and test scores. Luckily, the shift from traditional American values to more relevant human ones promotes support for academic stress and a greater focus on the more important aspects of life: happiness, health, and fulfillment.


