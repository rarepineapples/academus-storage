
\makechapter{Washington's Neutrality Proclaimation}{Washington's Neutrality Proclamation and the Birth of Executive Authority}{John Wang}{BASIS Fremont High School}

In 1793, the nascent French Republic launched preemptive military campaigns against its encircling monarchical neighbors to ensure its own survival \autocite{holland1911}. The result was a war of unprecedented scope and devastation. However, while primarily focused on Europe, France also inadvertently embroiled the young American republic in its first major foreign policy crisis. Caught between treaty obligations to France and the threat of British retaliation, the United States faced mounting pressure to choose sides, risking entanglement in a conflict that could tear the fragile union apart. It was under such uncertainty that President George Washington issued the Neutrality Proclamation. On the surface, it was nothing special: the short statement ostensibly asserted America’s intention to remain uninvolved in European conflicts. However, a close reading of the Proclamation reveals a deliberate effort to silence factional clamor in the national interest, an assertion of sharp Federalist assumptions about the extent of executive power, and a dismissal of revolutionary ideology as irrelevant to American interests. 

Although drafted by Edmund Randolph, the Proclamation bore Washington’s unmistakable influence, a fact most evident in its deliberate word choice \autocite[p.\ 59]{moats2021}. Strikingly, the term “neutrality” does not appear a single time; instead, the text uses “friendly” and “impartial” (\cite{washington1793}). This signaled a strategic ambiguity designed by Washington to soften the political implications of a policy that seemed to clearly align with Federalist interests (\cite[p.\ 71]{reinstein2011}). The words “friendly” and “impartial,” unlike “neutral,” carry the connotation of fairness without formal renunciation; it enabled the administration to distance itself from the French cause without appearing to betray it (\cite[p.\ 65]{moats2021}). At a time when the 1778 alliance with France remained technically binding, and when pro-French sentiments among Republicans ran high, the Proclamation’s language carefully tempered partisan divisions between Federalists and Republicans (\cite[p.\ 223, 215]{reinstein2011}). Nonetheless, this was a minor concession; the Proclamation’s effect remained distinctly Federalist, as it universalized the American stance as one firmly opposed to revolutionary fervor. At heart, Washington feared that open partisanship would entangle the young republic in foreign wars for which it was neither militarily nor politically prepared (\cite[pp.\ 470–471]{sheridan1994}). Such pragmatism, reflected in both the wording and intent of the Proclamation, forged a rare point of unity between rival factions in the cabinet (\cite[pp.\ 689–691]{chernow2011}). In doing so, Washington alleviated internal discord regarding the Proclamation’s issuance. 

Beyond foreign affairs, the Neutrality Proclamation also subtly redefined the constitutional boundaries of executive power. The 1778 Treaty of Alliance with France had pledged reciprocal military support should either nation be attacked, and it obligated the United States to defend French possessions in the Americas (\cite[p.\ 11–12]{reinstein2011}). By proclaiming neutrality, however, Washington effectively suspended this commitment, all without explicit congressional approval and absent any formal treaty abrogation. As such, many Republicans argued that Washington’s declaration overstepped the executive’s power (\cite[p.\ 126]{schmitt2000}). Madison, in particular, argued that Washington had issued the Proclamation in “unqualified terms” that ignored established treaty obligations to France (\cite[p.\ 126]{schmitt2000}). Under the Constitution, Congress holds the power to declare war; therefore, as Republicans reasoned, they possessed the authority to declare neutrality (\cites[p.\ 691]{chernow2011}[p.\ 127]{schmitt2000}[pp.\ 329–331]{prakash2001}). This controversy over presidential authority culminated in the pivotal Pacificus-Helvidius debates. In the Pacificus essays, Hamilton championed Washington’s Proclamation, arguing that Article II of the Constitution vested the president with all-encompassing “executive power,” including the “federative” power of managing foreign affairs (\cite[§\textsc{xi}, p.\ 12]{hamilton2007}). Under such a view, congressional powers to declare war and ratify treaties were exceptions—not limitations—on otherwise broad executive discretion. In response, Madison challenged Hamilton’s claims through the Helvidius essays, contending that foreign affairs, especially decisions with treaty implications, required legislative consent (\cite[p.\ 64]{hamilton2007}). The executive’s authority to create treaties, after all, requires a vote from the legislature; if the executive sought to withdraw the U.S. from treaty obligations, it ought also to consult the legislature (\cite[§2, cl. 2]{constitution}). Undoubtedly, Washington’s move occupied constitutional gray space: Article II granted the president executive authority, although without precise demarcation (\cite[pp.\ 445–446]{young2011}). Nevertheless, opponents of this expansion of power, though dismayed, found reassurance in the Proclamation’s ambiguity, ultimately valuing national unity and peace over ideological purism (\cite[p.\ 65, 75]{moats2021}). Thus, the assertion of power was tolerated. In this way, Washington used deliberate ambiguity to preserve presidential discretion without provoking accusations of despotism (\cite[p.\ 68]{moats2021}).

Beyond its legal implications, the Proclamation’s greatest statement may have been what it refused to say. On April 8, 1793, Edmond-Charles Genêt, a French diplomat charged with enlisting American support for the French Revolution, arrived in South Carolina with the mission and radical fervor of Jacobinism—fervor that Washington feared might entangle American republicanism with radicalism (\cites[p.\ 464]{sheridan1994}[pp.\ 444–445]{mccullough2008}{hamilton1793}). However, the Proclamation issued two weeks later never mentioned France, revolution, or Genêt. This refusal to address Genêt’s republican evangelism in explicit terms was deliberate (\cite{washington1793}). By distancing itself from direct confrontation, the document shrewdly redefined American identity in contrast to the excesses of the French Revolution. Accordingly, neutrality emerged as a method of national self-definition: America would not be like France (\cites[p.\ 470]{sheridan1994}[p.\ 465]{young2011}). It is rather ironic, then, that the Proclamation that sought to rise above politics anchored the presidency to public opinion. Faced with Genêt’s popularity and Republican criticism, Federalists mobilized mass meetings and coordinated pro-administration resolutions, all to rally popular support behind Washington’s Proclamation (\cite[p.\ 454–456]{young2011}). The president responded in kind, formalizing a new interactive relationship between the executive and the public (\cite[pp.\ 437, 456–58, 462]{young2011}). In the face of popular outcry, Washington did not retreat from the public arena; instead, he embraced a public-facing presidency—one that cultivated mass approval to maintain legitimacy, even as it sought to curtail popular radicalism (\cite[p.\ 465]{sheridan1994}). This has long-term consequences. The Proclamation set the stage for the public politicization of the presidency, not only in function but also in perception (\cite[p.\ 435]{young2011}).

The Neutrality Proclamation of 1793 did more than simply outline American neutrality. Through the strategic use of ambiguous language, careful assertion of executive authority, and refusal to indulge in partisan fervor, Washington elevated neutrality to a foundational principle of American foreign policy. The Neutrality Proclamation established a model for foreign policy that persisted over the next century and a half. Furthermore, it instituted a new norm in foreign policymaking—one defined by a powerful yet measured executive branch capable of navigating both domestic and international challenges (\cite[p.\ 107]{reinstein2011}).


