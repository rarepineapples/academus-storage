\documentclass[12pt, a4paper, twoside]{article}
\usepackage{format}
\usepackage{tikz}
\usepackage{pgfplots}
\pgfplotsset{compat=1.18} 
% Do not alter above

% Metadata: put your article information here 
\newcommand{\jtitle}{Rejection of Cognitivism}
\newcommand{\jauthor}{Jeff Cao}
\newcommand{\jaffiliation}{Campolindo High School}

% Editors will change these fields after acceptance 
\newcommand{\jvolume}{2}
\newcommand{\jyear}{2025}
\newcommand{\jdoi}{10.17613/aa688-7tw25}  

% References should be placed in refs.bib and cited with \autocite{<source>}
% Quotations can be placed in quote environments: \begin{quote}<your quote>\end{quote}
% Footnotes can be added with \footnote{<your footnote>}

% Your Content

\begin{document}

\maketitle{}

Cognitivism is the belief that trust essentially involves believing. Evidentialism is the belief that rational beliefs must be based on evidence. Whereas, conflict is the belief that rational trust sometimes conflicts with evidence. For trust to be rational, only two beliefs can be true. T. Nguyen rejects the belief about cognitivism.

According to Nguyen, trust is not a belief. To trust is to adopt an unquestioning attitude. To support his claim, Nguyen essentially argues that “to trust X as an informational source in domain Z is to have an attitude of not questioning X’s deliverances concerning Z”. By rejecting cognitivism and instead focusing on how conflict and evidentialism coincide to make trust rational, we align with Nguyen’s theory that trust is an unquestioning attitude, not a belief.

Nguyen rejects cognitivism by explaining that trust is not a belief but rather an unquestioning attitude. Based on this view, to trust someone is not necessarily to believe that they will act in a certain way. Instead, you are suspending doubt without continuous evaluation. Implicit trust allows Nguyen to contend the view that when you are trusting them, you are not believing that they are trustworthy, but you still treat them as if they are trustworthy.

There are three examples that Nguyen gives to support trust as an unquestioning attitude. First, when someone walks on the ground. You are trusting the ground to support your weight as you walk. It becomes natural, you are not constantly thinking about whether you will fall because the ground does not support you. Second, when you trust your memory to remind you to get the missing ingredients at the supermarket. Third, when you trust your doctor to give appropriate medical advice. 

A real-life example of implicit trust is when you are in class listening to your instructor. When you are in class, you do not know what type of person your professor is. It does not really matter, either. What matters is that you are trusting the professor to be an informational source in the classroom. You are not questioning whether they know about the course material and are capable of teaching it. When you are in a classroom setting, you are implicitly trusting that your professor is capable of teaching.

Nguyen firmly supports evidentialism—the idea that rational beliefs must be grounded in evidence, although with a caveat. According to Nguyen, trust does not replace belief, but rather enables us to form beliefs by integrating trustworthy agents and non-agents into our lives. In his proposition, the concept of trust functions as a way to increase cognitive efficiency. With this attitude of trust, one functions more effectively without constantly reassessing all evidence. He states, “when I accept testimony through trust without deliberation, it isn’t that I have failed to go through a proper practical deliberation. I am deferring to deliberation that was run elsewhere.” This essentially means that trust involves belief in the judgments of others, specifically if agents/non-agents have evidenced their trustworthiness.

Nguyen makes a key distinction between trust and belief to clarify how evidentialism is preserved. Trust, he argues, is not a belief about someone’s reliability. Instead, it’s an unquestioning attitude that temporarily suspends personal evaluation, giving external sources (like memory, doctors, or technology) direct input into one’s reasoning. Beliefs, by contrast, are still subject to evidential standards. An example of this dichotomy is trusting your doctor without constantly questioning them; you are not believing your doctor is correct in every moment—you are simply not re-evaluating. However, this trust can support belief by providing evidence through testimony, assuming your doctor’s advice is proven by their medical expertise. Thus, evidentialism is preserved not through internal reevaluation of every piece of evidence, but through epistemic dependence on trusted agents and non-agents.

While some argue that Nguyen’s idea of trust is based more on functional integration rather than genuine trust, Nguyen importantly distinguishes between trust and reliability. Unlike reliability, trust as an unquestioning attitude stipulates that true trust must be subconscious or fully clear of doubt. Yes, one can reason and reject objections to trusting in something, but never believe and accept counterpoints. This creates a deeper emotional form of trust that is illustrated by the idea of betrayal, or the breaking of trust. When people, objects, or internal functions fail us, the personal nature of betrayal illustrates the deeper bond of trust beyond that of mere reliability.

Nguyen defends the conflict thesis by arguing that trust can remain rational even when it contradicts available evidence. According to the distinction established earlier in rejecting cognitivism, the conflict between trust and evidence no longer implies irrationality. Instead, it reveals the cognitive limitations of humans. As Nguyen argues, we are cognitively limited beings who cannot constantly re-evaluate every piece of information or every relationship we rely on. To solve this limitation, we develop trust mechanisms that help to conserve cognitive resources. Functionally, trust operates like crossing one thing off our to-do list. Once trust is established, we do not continuously re-justify it. Even when counter evidence emerges, we are not necessarily required to immediately withdraw our trust. This temporary suspension of evaluation is precisely what gives trust its value. Trust is not irrational; it is a rational strategy that allows us to act effectively under conditions of limited attention and time.

In real-life examples, a surgeon may trust the anesthesiologist during a procedure, not because of recent reaffirming data, but based on the default assumption that they will fulfill their role. That trust may temporarily conflict with weak new signals of risk. But still, if the surgeon were to pause and re-evaluate in the middle of the operation, it could compromise team coordination and the safety of the patient. In this context, trust that contradicts minor evidence is not only rational and necessary.

A possible objection to Nguyen’s view is that suspending doubt might lead to blind trust. If trust persists despite counterevidence, how can it remain rational? Nguyen would respond that the suspension of evaluation is neither unconditional nor permanent. It is a temporary, pragmatic strategy that helps us act efficiently. When strong disconfirming evidence arises, we naturally reassess our trust. In this way, trust does not reject evidence; it simply postpones constant reevaluation until it is necessary. Moreover, Nguyen’s notion of “delegated deliberation” ensures that trust is still indirectly evidence-based. We rely on others’ judgments not out of ignorance, but because it is rational to do so under cognitive constraints. Thus, trust remains a rational attitude, not a lapse in reasoning.

By rejecting cognitivism, it is then possible to keep evidentialism and conflict to make trust a rational belief. Nguyen’s view reframes trust not as a belief, but as an unquestioning attitude that suspends active evaluation in order to manage our cognitive limitations. This makes space for a form of rational trust that can sometimes persist despite conflicting evidence. Trust, then, is not a deviation from rationality, but an adaptive response to the practical demands of reasoning and action in a complex world.





\printbibliography

\end{document}