\documentclass[12pt, a4paper, twoside]{article}
\usepackage{format}
% Do not alter above

% Metadata: put your article information here 
\newcommand{\jtitle}{Against Moral Intuitionism}
\newcommand{\jauthor}{Celine Wang}
\newcommand{\jaffiliation}{Farmington High School}

% Editors will change these fields after acceptance 
\newcommand{\jvolume}{1}
\newcommand{\jyear}{2025}
% \newcommand{\jdoi}{N/A}  


% References should be placed in refs.bib and cited with \autocite{<source>}
% Quotations can be placed in quote environments: \begin{quote}<your quote>\end{quote}
% Footnotes can be added with \footnote{<your footnote>}

% Your Content

\begin{document}

\maketitle{}

\section{Introduction}

In this paper, I contend that we do not have any reason to trust in our moral intuitions on the grounds that moral intuitions are epistemically unreliable, suffer from persistent disagreement, and are subject to debate regarding its objectivity. 

To limit the scope of our discussion, I will be focusing on a foundationalist interpretation of intuition as a means of knowledge production (\cite{britannica2024intuition}). As such, an intuition shall thereby be defined as a self-evident “intellectual seeming” (\cites{audi1999self}[p.\ 10]{bealer1998intuition}[p.\ 102]{huemer2005ethical}[p.\ 1]{ogar2016critique}). Consequently, moral intuitionism would entail a non-natural, realist stance on the objectivity of morality, and would be the utilization of intuition as an epistemic tool concerning moral matters; i.e., what is right and what is wrong (\cites{ethics2024moral}{stratton2014intuitionism}).  

\section{Argument From Unreliability}

As opposed to perception, intuition is notoriously unreliable. While not infallible, perceptual seemings (i.e., the visual analog of intellectual seemings) are typically considered reliable because they detect potential inaccuracies \emph{vis-à-vis} context. In other words, when weather phenomena obscures vision or when physically impaired, we are less apt to trust our perceptual seemings (\cite[p.\ 8]{cecchini2024reliability}). Intellectual seemings, on the other hand, lack the capacity to receive reliable feedback contextually (\cite[pp.\ 17–18]{destefano2014reliability}).

Additionally, the non-naturalness of moral intuition implies that they are causally inert, i.e., they can neither affect nor be affected by other objects. That is to say, moral intuitions cannot be caused by their corresponding moral truths (\cites[p.\ 1]{hayward2019immoral}{stratton2014intuitionism}).

Moral intuitions are also subject to various distorting factors.  

First, phraseology. Studies have concluded that when the trolley problem was presented with “saving language”, people overwhelmingly choose to flip the switch; the opposite occurred when “killing language” was employed (\cite[p.\ 6]{bengson2013experimental}).

Second, partiality. One might have ulterior motives for experiencing certain intuitions (\cite[pp.\ 343–346]{sinnott2006moral}). For example, in judging two piano performances, the father of one of the contestants would, intuitively, think that his daughter played better, though this might not be justified without familial association (\cite[p.\ 343]{sinnott2006moral}).

Third, disgust. Disgust elicited by physical or chemical stimuli and disgust regarding some moral prospects, respectively, have been shown to influence the moral intuitive process (\cite[p.\ 8]{tao2022effects}).

Moreover, other factors also influence intuition. These include the ordering of cases involving intuition, hunger level of intuitive moral agents, and the “Knobe effect,” i.e., the tendency for people to judge intentional actions as negative rather than positive (\cites{swain2008instability}[p.\ 1]{danziger2011extraneous}).

An argument which has been proposed by Jonathan Smith states that partiality is imprecise. Concerning the example mentioned previously, Smith argues that a piano performance is incomparable with moral intuition. There are nuances and particularities present in musical performances which are not present in the fundamental claims of basic moral intuitions (e.g., murder is bad). He presents a counterexample: two contestants play a single note and are judged based on proximity with a third note. Thereby, if the father judges that the note his daughters played was closer, then he would be justified in believing that his daughter won, regardless of his familial predisposition towards her (\cite[p.\ 77]{smith2010on}).

Though I agree that the father would be justified in the second instance, this argument assumes a moral realist conclusion. A better reformulation of the piano competition would be one in which the two contestants are simply told to simply play one note. The father is then asked simply, “Who played better?” By eliminating the intended note (or, when applied to the topic of moral intuitionism, some accepted moral principle), partiality comes back into play. 

Considering another possibility, Dario Cecchini suggests that the strength of intuition may be a method of tracking the potential reliability of intuition in a context-sensitive way. This would only be the case if one truly accepts moral intuitions proportional to confidence in said intuitions, and that such confidence is epistemically reliable (\cite[p.\ 11]{cecchini2024reliability}). One’s confidence would then accordingly be used to determine the presence of biases, and as such, eliminating the possibility of distorted intuitions. 

Nonetheless, there are inconsistencies in this argument. The idea that intuitive strength is epistemically reliability remains hotly contested, as it currently lacks sufficient empirical study (\cite[p.\ 22]{cecchini2024reliability}). Furthermore, it seems unlikely that one would be able to prove the epistemic reliability of moral intuitions, given that intuitive strength itself is subject to distorting factors such as the Dunning-Kruger effect or the “consensuality principle” (\cites[pp.\ 23–24]{cecchini2024reliability}[pp.\ 259–262]{dunning2011kruger}{koriat2012self}). Such an argument is also subject to the implication that self-evidence is strength dependent, which undermines the very definition of the term itself. 

\section{Argument from Diagreement}

One might assume that since moral intuition is “self-evident” and “non-inferential,” that there would be harmony between proponents of moral ideas. Yet, this is not the case. When ten diverse societies were examined for their views regarding moral principles, each “exhibited substantial variation not only in the degree to which such [moral] factors were viewed as excusing, but also in the kinds of [moral] factors taken to provide exculpatory excuses, and in the types of norm violations for which such [moral] factors were seen as relevant” (\cite[p.\ 4692]{barrett2016small}). If the moral intuitions of two unprejudiced individuals persistently disagree on some moral issue, then the realist basis for intuitionism would be compromised (\cites[pp.\ 208–210]{sidgwick1907methods}[p.\ 44]{shaferlandau2004good}).

In addition, what seems “moral” to us may be evolutionarily favored biological responses that have emerged over time. We have evolved to react to different moral acts with either approval or disapproval based on biological necessity, preventing us from attaining positive knowledge of moral predicates (\cites[p.\ 348]{singer2009ethics}[p.\ 72]{joyce2007evolution}{morton2016new}). Differences in moral beliefs have also been observed across ages and between modern and traditional culture groups (\cite{nisan1987moral}). One especially notable example of moral disagreement is the disagreement between moral collectivists and individualists, as many common moral realist objections directed towards the idea that their disagreements will not remain persistent, have been disproven (\cite[pp.\ 855–864]{goldman2022right}).

Nevertheless, this is a shallow argument, as non-moral disagreements are not the true point of contention in a moral issue. In comparison with the previous example, we would certainly object if a living person were to be boiled, even without experiencing pain. The question again returns to a moral one: Are the lives of a lobster and a person worth the same? If one continues to argue that “worth” could be determined via some empirical measure, e.g., brain capacity, then a new moral issue emerges: is it morally acceptable to boil some organism simply based on the metric of brain capacity? This line of argument suggests the infinite regressive loop fallacy.  

Russ Shafer-Landau proposes that the argument from moral disagreement is self-defeating. He summarizes said argument in terms of moral realism, a key property of intuitionism: 
\begin{enumerate}
	\item[P1:] \textit{Moral realism is subject to persistent disagreement;}

	\item[P2:] \textit{Any theory(s) subject to persistent disagreement is false;}

	\item[C:] \textit{Therefore, moral realism is false (\cite[pp.\ 44–48]{shaferlandau2004good})}.
\end{enumerate}

To this, he claims that the same argument could be applied in reverse. Substituting “moral realism” for “theories against moral realism” results in the conclusion that theories against moral realism are false. 

As we will see, though Shafer-Landau’s argument is valid, it is not sound. Concealed by Shafer-Landau’s summary of the argument from disagreement is what I will term the agreement premise. The premise states that if moral realism is true, then there would be widespread agreement regarding moral realism. 

\begin{enumerate}

	\item[P1:] \textit{The agreement premiss is true;}

	\item[P2:] \textit{Moral realism is subject to persistent disagreement;}

	\item[P3:] \textit{Any theory(s) subject to persistent disagreement is false;}

	\item[C:] \textit{Therefore, either (a) moral intuitionism is false, or (b) the agreement premise is false.}

\end{enumerate}

Accepting either (a) or (b) would be difficult for the moral realist. Conversely, if one were to now substitute “moral realism” for “theories against moral realism,” then (b) would be acceptable to opponents of moral realism, as it implies that only some would agree with moral realism, perfectly. In other words, Shafer-Landau’s argument, though compelling, is more so \emph{vis-à-vis} the moral realist. 

\section{Argument from Queerness}

Moral Intuitionism is a theory that is built upon a foundation of self-evident, non-inferential moral objectivity. However, the very existence of such values can be challenged. In J. L. Mackie’s Argument from Queerness, Mackie reasons such that if objective values were to exist, then objectively good values would have an inherent “to-be-pursuedness,” and objectively bad values would have an inherent “not-to-be-pursuedness” (\cites[p.\ 40]{mackie1978ethics}[p.\ 104]{olson2014moral}). This motivational property is one that is unique to objective moral values, and one which cannot be described in naturalistic terms; in other words, objective values are “queer” (\cites[pp.\ 2–3]{scholl2015defense}[p.\ 33]{mackie1978ethics}) To claim that such a property is not queer is to invite contradiction with the non-natural property of moral intuitions (\cite[p.\ 41]{mackie1978ethics}). The argument can be represented in both metaphysical and epistemological versions, however, only the epistemological is necessary here:  

\begin{enumerate}

	\item[P1:] \textit{If objective moral values exist, they will have to be epistemologically queer, and require unconventional, bizarre epistemological faculties;}

	\item[P2:] \textit{We are unable to utilize such bizarre epistemological faculties;}

	\item[C:] \textit{We should not believe that objective moral values exist (\cites[pp.\ 38–41]{mackie1978ethics}[pp.\ 1–2]{lillehammer2019queerness}).}

\end{enumerate}

The parity argument, proposed by Terence Cuneo, attempts to prove the existence of objective values. His argument has two main premises: the parity premise ($P_P$) and the epistemological realism premise ($P_E$). The parity premise states that moral facts and epistemic facts are analogous to one another since both are categorical imperatives, i.e., both tell us what we ought to do and believe regardless of our subjective desires. The epistemic realism premise describes the idea that belief in the existence of epistemic facts seems natural; otherwise, one could theoretically have misconstrued beliefs based only on whims and compulsions (\cite[pp.\ 52–62]{cuneo2010normative}). His argument is outlined as the following, with an extra “modus tollens” ($P_{MT}$) premise added for clarity:
\begin{enumerate}

	\item[P1:] \textit{If objective moral values exist, they will have to be epistemologically queer, and require unconventional, bizarre epistemological faculties;}

	\item[P2:] \textit{We are unable to utilize such bizarre epistemological faculties;}

	\item[C:] \textit{We should not believe that objective moral values exist (\cites[pp.\ 38–41]{mackie1978ethics}[pp.\ 1–2]{lillehammer2019queerness}).}

\end{enumerate}

The parity argument, proposed by Terence Cuneo, attempts to prove the existence of objective values. His argument has two main premises: the parity premise ($P_P$) and the epistemological realism premise ($P_E$). The parity premise states that moral facts and epistemic facts are analogous to one another since both are categorical imperatives, i.e., both tell us what we ought to do and believe regardless of our subjective desires. The epistemic realism premise describes the idea that belief in the existence of epistemic facts seems natural; otherwise, one could theoretically have misconstrued beliefs based only on whims and compulsions (\cite[pp.\ 52–62]{cuneo2010normative}). His argument is outlined as the following, with an extra “modus tollens” ($P_{MT}$) premise added for clarity: 
\begin{enumerate}

\item[$P_P:$] \textit{There can be no epistemic facts if there are no moral facts;}

\item[$P_E$:] \textit{There are epistemic facts;}

\item[$P_{MT}$:] \textit{If there are epistemic facts, then there are moral facts;}

\item[C:] \textit{If there are moral facts, then moral realism is true (\cites[p.\ 1]{{}rutten2010parity}[p.\ 6]).}

\end{enumerate}

Believing in moral realism, by definition, entails believing in objective moral values, and in this case, moral intuitionism. 

Regardless, though the argument is valid, the parity premise lacks argumentative power. As Richard Joyce notes, there are many ways in which moral facts and epistemic facts are incompatible. Examples include how unlike with moral facts: 

\begin{enumerate}

	\item We acquire epistemic facts involuntarily.

	\item Epsitemological facts lack the concept of desert, i.e., "deserved reward or punishment (\cite{merriam2024desert})."

	\item We are not reluctant to rely on the judgement of experts when faced with epistemological facts (\cite[pp.\ 14–16]{joyce2007evolution}).

\end{enumerate}

As a result of these discrepancies, we can safely disregard the parity argument and regard the argument unsound.  

Hallvard Lillehammer expresses another objection, claiming that knowledge of objective moral values does not entail the utilization of epistemologically queer faculties (\cite{balaguer2016platonism}). He argues that moral facts can be equated with logical facts, since both are causally inert.xxxii Hence, as Lillehammer points out, if logical facts rely only on an “unproblematic sense … which we are able to grasp” and not some bizarre epistemological faculty, then the same should be the case for moral facts. If one were to deny this argument, the implication would be that we are unable to access logical facts, a highly contentious and counterintuitive claim (\cite[p.\ 6]{lillehammer2019queerness}). 

Whilst claiming that logical facts are epistemologically inaccessible is certainly absurd, we can nonetheless dispute the notion that logical facts and moral facts are analogous. One key disanalogy between the two lies in that logical facts lack the “to-be-pursuedness” of moral facts. For example, while the moral fact “giving to charity is good” carries an inherent tendency to act, a tautology such as “all humans are mammals” does not (\cite{britannica2024tautology}). 

\section{I have presented three arguments against moral intuitionism in this paper, including how: }

\begin{enumerate}
\item Moral intuitions are fallible.

\item There are disagreements between supposedly ``self-evident'' moral intuitions.

\item The objective nature of moral intuitions can be challenged.
\end{enumerate}

Thus, I reject the notion that we have any good reason to trust our moral intuition.

\printbibliography
\end{document}
